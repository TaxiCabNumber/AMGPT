\documentclass[10pt]{article}
\usepackage[utf8]{inputenc}
\usepackage[T1]{fontenc}
\usepackage{amsmath}
\usepackage{amsfonts}
\usepackage{amssymb}
\usepackage[version=4]{mhchem}
\usepackage{stmaryrd}
\usepackage{graphicx}
\usepackage[export]{adjustbox}
\graphicspath{ {./images/} }
\usepackage{multirow}
\usepackage{hyperref}
\hypersetup{colorlinks=true, linkcolor=blue, filecolor=magenta, urlcolor=cyan,}
\urlstyle{same}

\title{Material data sheet }


\author{John J. Lewandowski and Mohsen Seifi\\
Department of Materials Science and Engineering, Case Western Reserve University (CWRU),\\
Cleveland, Ohio 44106; email: JJL3@case.edu, mohsen.seifi@case.edu}
\date{}


\begin{document}
\maketitle
\section*{EOS Aluminium AISi10Mg}
EOS Aluminium AISi10Mg is an aluminium alloy in fine powder form which has been specially optimised for processing on EOSINT M systems

This document provides information and data for parts built using EOS Aluminium AISi10Mg powder (EOS art.-no. 9011-0024) on the following system specifications:

\begin{itemize}
  \item EOSINT M 280 with PSW 3.6 and Original EOS Parameterset AISi10Mg\_Speed 1.0
  \item EOS M 290 400Watt with EOSPRINT 1.0 and Original EOS Parameterset AISi10Mg\_Speed 1.0
\end{itemize}

\section*{Description}
AlSi10Mg is a typical casting alloy with good casting properties and is typically used for cast parts with thin walls and complex geometry. It offers good strength, hardness and dynamic properties and is therefore also used for parts subject to high loads. Parts in EOS Aluminium AISi10Mg are ideal for applications which require a combination of good thermal properties and low weight. They can be machined, spark-eroded, welded, micro shot-peened, polished and coated if required.

Conventionally cast components in this type of aluminium alloy are often heat treated to improve the mechanical properties, for example using the T6 cycle of solution annealing, quenching and age hardening. The laser-sintering process is characterized by extremely rapid melting and re-solidification. This produces a metallurgy and corresponding mechanical properties in the as-built condition which is similar to T6 heat-treated cast parts. Therefore such hardening heat treatments are not recommended for laser-sintered parts, but rather a stress relieving cycle of 2 hours at $300^{\circ} \mathrm{C}\left(572^{\circ} \mathrm{F}\right)$. Due to the layerwise building method, the parts have a certain anisotropy, which can be reduced or removed by appropriate heat treatment - see Technical Data for examples.

\section*{Material data sheet}
\section*{Technical data}
\section*{General process and geometrical data}
\begin{center}
\begin{tabular}{|c|c|}
\hline
Typical achievable part accuracy [1] [2] & $100 \mu \mathrm{m}$ \\
\hline
Smallest wall thickness [1] [3] & \begin{tabular}{c}
approx. $0.3-0.4 \mathrm{~mm}$ \\
approx. $0.012-0.016$ inch \\
\end{tabular} \\
\hline
Surface roughness, as built, cleaned [1] [4] & \begin{tabular}{l}
Ra $6-10 \mu \mathrm{m}, R_{z} 30-40 \mu \mathrm{m}$ \\
Ra $0.24-0.39 \times 10^{3} \mathrm{inch}$ \\
Rz $1.18-1.57 \times 10^{3} \mathrm{inch}$ \\
\end{tabular} \\
\hline
- after micro shot-peening & \begin{tabular}{l}
Ra $7-10 \mu \mathrm{m}, \mathrm{Rz}_{z} 50-60 \mu \mathrm{m}$ \\
Ra $0.28-0.39 \times 10^{3}$ inch \\
Rz $1.97-2.36 \times 10^{3}$ inch \\
\end{tabular} \\
\hline
Volume rate [5] & \begin{tabular}{c}
$7.4 \mathrm{~mm}^{3} / \mathrm{s}\left(26.6 \mathrm{~cm}^{3} / \mathrm{h}\right)$ \\
$1.6 \mathrm{in}^{3} / \mathrm{h}$ \\
\end{tabular} \\
\hline
\end{tabular}
\end{center}

[1] These properties were determined on an EOSINT M 270.

[2] Based on users' experience of dimensional accuracy for typical geometries. Part accuracy is subject to appropriate data preparation and post-processing, in accordance with EOS training.

[3] Mechanical stability dependent on the geometry (wall height etc.) and application

[4] Due to the layerwise building, the surface structure depends strongly on the orientation of the surface, for example sloping and curved surfaces exhibit a stair-step effect. The values also depend on the measurement method used. The values quoted here given an indication of what can be expected for horizontal (up-facing) or vertical surfaces.

[5] The volume rate is a measure of the building speed during laser exposure. The overall building speed is dependent on the average volume rate, the time required for coating (depends on the number of layers) and other factors, e.g. DMLS settings.

\section*{Material data sheet}
Physical and chemical properties of the parts

\begin{center}
\begin{tabular}{lc}
\hline
Material composition & $\mathrm{Al}$ (balance) \\
 & $\mathrm{Si}(9.0-11.0 \mathrm{wt}-\%)$ \\
$\mathrm{Fe}(0.55 \mathrm{wt}-\%)$ &  \\
$\mathrm{Cu}(0.05 \mathrm{wt}-\%)$ &  \\
$\mathrm{Mn}(0.45 \mathrm{wt}-\%)$ &  \\
$\mathrm{Mg}(0.2-0.45 \mathrm{wt}-\%)$ &  \\
$\mathrm{Ni}(0.05 \mathrm{wt}-\%)$ &  \\
$\mathrm{Zn}(0.10 \mathrm{wt}-\%)$ &  \\
$\mathrm{Pb}(0.05 \mathrm{wt}-\%)$ &  \\
$\mathrm{Sn} .(0.05 \mathrm{wt}-\%)$ &  \\
$\mathrm{Ti}(0.15 \mathrm{wt}-\%)$ &  \\
\hline
Relative density & $\mathrm{approx.99.85 \%}$ \\
\hline
Density & $2.67 \mathrm{~g} / \mathrm{cm}^{3}$ \\
\hline
\end{tabular}
\end{center}

\section*{Material data sheet}
\section*{Mechanical properties of the parts}
\begin{center}
\begin{tabular}{|c|c|c|}
\hline
 & As built & Heat treated [9] \\
\hline
\multicolumn{3}{|l|}{Tensile strength [6]} \\
\hline
- in horizontal direction (XY) & \begin{tabular}{l}
$460 \pm 20 \mathrm{MPa}$ \\
$66.7 \pm 2.9 \mathrm{ksi}$ \\
\end{tabular} & \begin{tabular}{c}
$345 \pm 10 \mathrm{MPA}$ \\
$50.0 \pm 1.5 \mathrm{ksi}$ \\
\end{tabular} \\
\hline
- in vertical direction (Z) & \begin{tabular}{c}
$460 \pm 20 \mathrm{MPa}$ \\
$66.7 \pm 2.9 \mathrm{ksi}$ \\
\end{tabular} & \begin{tabular}{c}
$350 \pm 10 \mathrm{MPa}$ \\
$50.8 \pm 1.5 \mathrm{ksi}$ \\
\end{tabular} \\
\hline
\multicolumn{3}{|l|}{Yield strength (Rp $0.2 \%$ ) [6]} \\
\hline
- in horizontal direction (XY) & \begin{tabular}{l}
$270 \pm 10 \mathrm{MPa}$ \\
$39.2 \pm 1.5 \mathrm{ksi}$ \\
\end{tabular} & \begin{tabular}{l}
$230 \pm 15 \mathrm{MPa}$ \\
$33.4 \pm 2.2 \mathrm{ksi}$ \\
\end{tabular} \\
\hline
- in vertical direction (Z) & \begin{tabular}{l}
$240 \pm 10 \mathrm{MPa}$ \\
$34.8 \pm 1.5 \mathrm{ksi}$ \\
\end{tabular} & \begin{tabular}{l}
$230 \pm 15 \mathrm{MPa}$ \\
$33.4 \pm 2.2 \mathrm{ksi}$ \\
\end{tabular} \\
\hline
\multicolumn{3}{|l|}{Modulus of elasticity} \\
\hline
- in horizontal direction (XY) & \begin{tabular}{c}
$75 \pm 10 \mathrm{GPa}$ \\
$10.9 \pm 0.7 \mathrm{Msi}$ \\
\end{tabular} & \begin{tabular}{c}
$70 \pm 10 \mathrm{GPa}$ \\
$10.2 \pm 0.7 \mathrm{Msi}$ \\
\end{tabular} \\
\hline
- in vertical direction (Z) & \begin{tabular}{c}
$70 \pm 10 \mathrm{GPa}$ \\
$10.2 \pm 0.7 \mathrm{Msi}$ \\
\end{tabular} & \begin{tabular}{l}
$60 \pm 10 \mathrm{GPa}$ \\
$8.7 \pm 0.7 \mathrm{Msi}$ \\
\end{tabular} \\
\hline
\multicolumn{3}{|l|}{Elongation at break [6]} \\
\hline
- in horizontal direction (XY) & $(9 \pm 2) \%$ & $12 \pm 2 \%$ \\
\hline
- in vertical direction (Z) & $(6 \pm 2) \%$ & $11 \pm 2 \%$ \\
\hline
Hardness [7] & approx.119 $\pm 5$ HBW &  \\
\hline
\multicolumn{3}{|l|}{Fatigue strength [1] [8]} \\
\hline
- in vertical direction (Z) & \begin{tabular}{c}
approx. $97 \pm 7 \mathrm{MPa}$ \\
approx. $14.1 \pm 1.0 \mathrm{ksi}$ \\
\end{tabular} &  \\
\hline
\end{tabular}
\end{center}

[6] Mechanical strength tested as per ISO 6892-1:2009 (B) annex D, proportional specimens, specimen diameter $5 \mathrm{~mm}$, original gauge length $25 \mathrm{~mm}$ (1 inch).

[7] Hardness test in accordance with Brinell (HBW 2.5/62.5) as per DIN EN ISO 6506-1. Note that measured hardness can vary significantly depending on how the specimen has been prepared.

[8] Fatigue test with test frequency of $50 \mathrm{~Hz}, \mathrm{R}=-1$, measurement stopped on reaching 5 million cycles without fracture.

[9] Stress relieve: anneal for $2 \mathrm{~h}$ at $300^{\circ} \mathrm{C}\left(572^{\circ} \mathrm{F}\right)$.

[10] These properties were determined on an EOSINT M 280-400W. Test parts from following machine type EOS M 290-400W correspond with these data.

\section*{Material data sheet}
Thermal properties of parts

\begin{center}
\begin{tabular}{lll}
\hline
 & \multicolumn{1}{c}{As built [1]} & Heat treated [1] [9] \\
\hline
Thermal conductivity (at $\left.20^{\circ} \mathrm{C}\right)$ &  &  \\
- in horizontal direction $(X Y)$ & approx. $103 \pm 5 \mathrm{~W} / \mathrm{m}^{\circ} \mathrm{C}$ & approx. $173 \pm 10 \mathrm{~W} / \mathrm{m}^{\circ} \mathrm{C}$ \\
\cline { 2 - 3 }
- in vertical direction $(\mathrm{Z})$ & approx. $119 \pm 5 \mathrm{~W} / \mathrm{m}^{\circ} \mathrm{C}$ & approx. $173 \pm 10 \mathrm{~W} / \mathrm{m}^{\circ} \mathrm{C}$ \\
\hline
Specific heat capacity &  &  \\
- in horizontal direction (XY) & approx. $920 \pm 50 \mathrm{~J} / \mathrm{kg}^{\circ} \mathrm{C}$ & approx. $890 \pm 50 \mathrm{~J} / \mathrm{kg}^{\circ} \mathrm{C}$ \\
- in vertical direction $(\mathrm{Z})$ & approx. $910 \pm 50 \mathrm{~J} / \mathrm{kg}^{\circ} \mathrm{C}$ & approx. $890 \pm 50 \mathrm{~J} / \mathrm{kg}^{\circ} \mathrm{C}$ \\
\hline
\end{tabular}
\end{center}

\section*{Abbreviations}
approx. approximately

wt weight

\section*{Notes}
The data are valid for the combinations of powder material machine and parameter sets referred to on page 1 , when used in accordance with the relevant Operating Instructions (including Installation Requirements and Maintenance) and Parameter Sheet. Part properties are measured using defined test procedures. Further details of the test procedures used by EOS are available on request.

The data correspond to our knowledge and experience at the time of publication. They do not on their own provide a sufficient basis for designing parts. Neither do they provide any agreement or guarantee about the specific properties of a part or the suitability of a part for a specific application. The producer or the purchaser of a part is responsible for checking the properties and the suitability of a part for a particular application. This also applies regarding any rights of protection as well as laws and regulations. The data are subject to change without notice as part of EOS' continuous development and improvement processes.

EOS , EOSINT and DMLS are registered trademarks of EOS GmbH.

2014 EOS GmbH - Electro Optical Systems. All rights reserved.

\section*{EOS CobaltChrome MP1}
EOS CobaltChrome MP1 is a cobalt-chrome-molybdenum-based superalloy powder which has been optimized especially for processing on EOSINT M systems.

This document provides information and data for parts built using EOS CobaltChrome MP1 powder (EOS art.-no. 9011-0012) on the following system specifications:

$\boxtimes$ EOSINT M 270 Installation Mode Standard with PSW 3.3 or 3.4 and default job CC20\_MP1\_020\_default.job or CC20\_MP1\_040\_default.job

$\boxtimes$ EOSINT M 270 Dual-Mode with PSW 3.5 and EOS Original Parameter Set MP1\_Surface 1.0 or MP1\_Performance 1.0

$\boxtimes$ EOSINT M 280

with PSW 3.5 and EOS Original Parameter Set MP1\_Surface 1.0, MP1\_Performance 1.0 or MP1\_Speed 1.0

\section*{Description}
Parts built from EOS CobaltChrome MP1 conform to the chemical composition UNS R31538 of high carbon CoCrMo alloy. They are nickel-free ( $<0.1 \%$ nickel content) and are characterized by a fine, uniform crystal grain structure. As built EOS CobaltChrome MP1 meets the chemical and mechanical specifications of ISO 5832-4 and ASTM F75 for cast CoCrMo implant alloys, as well as the specifications of ISO 5832-12 and ASTM F1537 for wrought CoCrMo implants alloys except remaining elongation. The remaining elongation can be increased to fulfil even these standards by high temperature stress relieving or hot isostatic pressing (HIP).

Parts made from EOS CobaltChrome MP1 can be machined, spark-eroded, welded, micro shotpeened, polished and coated if required. They are suitable for biomedical applications (note: subject to fulfilment of statutory validation requirements where appropriate), and for parts requiring high mechanical properties in elevated temperatures $\left(500-1000^{\circ} \mathrm{C}\right)$ and with good corrosion resistance. Due to the layerwise building method, the parts have a certain anisotropy, which can be reduced or removed by appropriate heat treatment - see Technical Data for examples.

\section*{Material data sheet}
\section*{Technical data}
\section*{General process data}
\begin{center}
\begin{tabular}{|c|c|}
\hline
\multicolumn{2}{|l|}{Typical achievable part accuracy [1]} \\
\hline
- small parts & \begin{tabular}{l}
approx. $\pm 20-50 \mu \mathrm{m}$ \\
approx. $\pm 0.8-2 \times 10^{.3}$ inch \\
\end{tabular} \\
\hline
- large parts & \begin{tabular}{l}
approx. $\pm 50-200 \mu \mathrm{m}$ \\
approx. $\pm 2-8 \times 10^{.3}$ inch \\
\end{tabular} \\
\hline
Min. wall thickness [2] & \begin{tabular}{l}
approx. $0.3 \mathrm{~mm}$ \\
approx. 0.012 inch \\
\end{tabular} \\
\hline
\multicolumn{2}{|l|}{Surface roughness [3]} \\
\hline
\multicolumn{2}{|l|}{- as built} \\
\hline
MP1 Surface $(20 \mu \mathrm{m})$ & \begin{tabular}{l}
Ra $4-10 \mu \mathrm{m} ; R_{z} 20-40 \mu \mathrm{m}$ \\
Ra $0.16-0.39 \times 10^{-3}$ inch, \\
Rz $0.79-1.57 \times 10^{-3}$ inch \\
\end{tabular} \\
\hline
MP1 Performance (40 $\mu \mathrm{m})$ & \begin{tabular}{l}
Ra $7-10 \mu \mathrm{m} ; \mathrm{R}$ z $35-50 \mu \mathrm{m}$ \\
$\mathrm{R}_{\mathrm{a}} 0.28-0.39 \times 10^{-3} \mathrm{inch}$ \\
Rz $1.37-1.96 \times 10^{-3}$ inch \\
\end{tabular} \\
\hline
MP1 Speed (50 $\mu \mathrm{m})$ & \begin{tabular}{l}
Ra $8-12 \mu m ; R$ z $38-50 \mu m$ \\
Ra $0.31-0.47 \times 10^{.3}$ inch, \\
Rz $1.49-1.96 \times 10^{.3}$ inch \\
\end{tabular} \\
\hline
- after polishing & \begin{tabular}{c}
$\mathrm{R}_{\mathrm{z}}$ up to $<1 \mu \mathrm{m}$ \\
$\mathrm{R}_{z}$ up to $<0.04 \times 10^{3}$ inch \\
\end{tabular} \\
\hline
\multicolumn{2}{|l|}{Volume rate [4]} \\
\hline
\begin{tabular}{l}
- Parameter set MP1\_Surface 1.0 / default job \\
CC20\_MP1\_020\_default.job (20 $\mu$ m layer thickness) \\
\end{tabular} & \begin{tabular}{l}
$1.6 \mathrm{~mm}^{3} / \mathrm{s}\left(5.1 \mathrm{~cm}^{3} / \mathrm{h}\right)$ \\
$0.35 \mathrm{in}^{3} / \mathrm{h}$ \\
\end{tabular} \\
\hline
\begin{tabular}{l}
- Parameter set MP1\_Perfømance 1.0 / default job \\
CC20\_MP1\_040\_default.job (40 $\mu \mathrm{m}$ layer thickness) \\
\end{tabular} & \begin{tabular}{c}
$3.2 \mathrm{~mm}^{3} / \mathrm{s}\left(11.5 \mathrm{~cm}^{3} / \mathrm{h}\right)$ \\
$0.70 \mathrm{in}^{3} / \mathrm{h}$ \\
\end{tabular} \\
\hline
\begin{tabular}{l}
- Parameter set MP1\_Performance 1.0 for M 280 / 400 W \\
(40 um layer thickness) \\
\end{tabular} & \begin{tabular}{c}
$4.2 \mathrm{~mm}^{3} / \mathrm{s}\left(15.1 \mathrm{~cm}^{3} / \mathrm{h}\right)$ \\
$0.92 \mathrm{in}^{3} / \mathrm{h}$ \\
\end{tabular} \\
\hline
\begin{tabular}{l}
- Parameter set MP1\_Speed 1.0 / for M 280 / 400 W \\
(50 um layer thickness) \\
\end{tabular} & \begin{tabular}{c}
$5.5 \mathrm{~mm}^{3} / \mathrm{s}\left(19.8 \mathrm{~cm}^{3} / \mathrm{h}\right)$ \\
$1.21 \mathrm{in}^{3} / \mathrm{h}$ \\
\end{tabular} \\
\hline
\end{tabular}
\end{center}

\section*{Material data sheet}
[1] Based on users' experience of dimensional accuracy for typical geometries, e.g. $\pm 20 \mu \mathrm{m}$ ( $0.8 \times 10^{3}$ inch) when parameters can be optimized for a certain class of parts or $\pm 50 \mu \mathrm{m}\left(2 \times 10^{3}\right.$ inch) when building a new kind of geometry for the first time. For larger parts the accuracy can be improved by post-process stressrelieving at $1150^{\circ} \mathrm{C}\left(2100^{\circ} \mathrm{F}\right)$ for 6 hours. Part accuracy is subject to appropriate data preparation and postprocessing, in accordance with EOS training.

[2] Mechanical stability is dependent on geometry (wall height etc.) and application

[3] Due to the layerwise building, the surface structure depends strongly on the orientation of the surface, for example sloping and curved surfaces exhibit a stair-step effect. The values also depend on the measurement method used. The values quoted here given an indication of what can be expected for horizontal (up-facing) or vertical surfaces.

[4] Volume rate is a measure of build speed during laser exposure. The total build speed depends on the average volume rate, the recoating time (related to number of layers) and other factors such as DMLS-Start settings.

\section*{Physical and chemical properties of parts}
\begin{center}
\begin{tabular}{lc}
\hline
Material composition & $\mathrm{Co}(60-65 \mathrm{wt}-\%)$ \\
 & $\mathrm{Cr}(26-30 \mathrm{wt}-\%)$ \\
 & $\mathrm{Mo}(5-7 \mathrm{wt}-\%)$ \\
 & $\mathrm{Si}(\leq 1.0 \mathrm{wt}-\%)$ \\
 & $\mathrm{Mn}(\leq 1.0 \mathrm{wt}-\%)$ \\
 & $\mathrm{Fe}(<0.75 \mathrm{wt}-\%)$ \\
$\mathrm{C}(\leq 0.16 \mathrm{wt}-\%)$ &  \\
Relative density & $\mathrm{Ni}(\leq 0.10 \mathrm{wt}-\%)$ \\
\hline
Density & approx. $100 \%$ \\
\hline
 & approx. $8.3 \mathrm{~g} / \mathrm{cm}^{3}$ \\
 & approx. $0.30 \mathrm{lb} / \mathrm{in}^{3}$ \\
\hline
\end{tabular}
\end{center}

\section*{Material data sheet}
Mechanical properties of parts at $20^{\circ} \mathrm{C}\left(68^{\circ} \mathrm{F}\right)$

\begin{center}
\begin{tabular}{|c|c|c|}
\hline
 & As built & Stress relieved [5] \\
\hline
\multicolumn{3}{|l|}{Tensile strength [6]} \\
\hline
- in horizontal direction (XY) & \begin{tabular}{c}
$1350 \pm 100 \mathrm{MPa}$ \\
$196 \pm 15 \mathrm{ksi}$ \\
\end{tabular} & \begin{tabular}{c}
$1100 \pm 100 \mathrm{MPa}$ \\
$160 \pm 15 \mathrm{ksi}$ \\
\end{tabular} \\
\hline
- in vertical direction $(Z)$ & \begin{tabular}{c}
$1200 \pm 150 \mathrm{MPa}$ \\
$174 \pm 22 \mathrm{ksi}$ \\
\end{tabular} & \begin{tabular}{c}
$1100 \pm 100 \mathrm{MPa}$ \\
$160 \pm 15 \mathrm{ksi}$ \\
\end{tabular} \\
\hline
\multicolumn{3}{|l|}{Yield strength (Rp 0.2 \%) [6]} \\
\hline
- in horizontal direction (XY) & \begin{tabular}{c}
$1060 \pm 100 \mathrm{MPa}$ \\
$154 \pm 15 \mathrm{ksi}$ \\
\end{tabular} & \begin{tabular}{c}
$600 \pm 50 \mathrm{MPa}$ \\
$87 \pm 7 \mathrm{ksi}$ \\
\end{tabular} \\
\hline
- in vertical direction (Z) & \begin{tabular}{c}
$800 \pm 100 \mathrm{MPa}$ \\
$116 \pm 15 \mathrm{ksi}$ \\
\end{tabular} & \begin{tabular}{c}
$600 \pm 50 \mathrm{MPa}$ \\
$87 \pm 7 \mathrm{ksi}$ \\
\end{tabular} \\
\hline
\multicolumn{3}{|l|}{Elongation at break [6]} \\
\hline
- in horizontal direction (XY) & $(11 \pm 3) \%$ & $\min .20 \%$ \\
\hline
- in vertical direction (Z) & $(24 \pm 4) \%$ & $\min .20 \%$ \\
\hline
\multicolumn{3}{|l|}{Modulus of elasticity [6]} \\
\hline
- in horizontal direction (XY) & \begin{tabular}{l}
$200 \pm 20 \mathrm{GPa}$ \\
$29 \pm 3 \mathrm{Msi}$ \\
\end{tabular} & \begin{tabular}{c}
$200 \pm 20 \mathrm{GPa}$ \\
$29 \pm 3 \mathrm{Msi}$ \\
\end{tabular} \\
\hline
- in vertical direction (Z) & \begin{tabular}{c}
$190 \pm 20 \mathrm{GPa}$ \\
$28 \pm 3 \mathrm{Msi}$ \\
\end{tabular} & \begin{tabular}{c}
$200 \pm 20 \mathrm{GPa}$ \\
$29 \pm 3 \mathrm{Msi}$ \\
\end{tabular} \\
\hline
\multicolumn{3}{|l|}{Fatigue life [7]} \\
\hline
- max. stress to reach 10 million cycles & \multicolumn{2}{|c|}{approx. $560 \mathrm{MPa}, 81$ ksi} \\
\hline
- max. stress to reach 1 million cycles & \multicolumn{2}{|c|}{approx. $660 \mathrm{MPa}, 96$ ksi} \\
\hline
Hardness [8] & \multicolumn{2}{|c|}{approx. 35 - 45 HRC} \\
\hline
\end{tabular}
\end{center}

[5] High temperature stress relieved, 6 hours at $1150^{\circ} \mathrm{C}\left(2100{ }^{\circ} \mathrm{F}\right)$ under inert argon atmosphere

[6] Tensile testing according to ISO 6892-1:2009 (B) Annex D, proportional test pieces, diameter of the neck area $5 \mathrm{~mm}$ ( 0.2 inch), original gauge length $25 \mathrm{~mm}$ ( 1 inch).

[7] Testing according to ASTM E466:1996, using vertical samples, as built, under $250 \mathrm{MPa}$ ( $36.3 \mathrm{ksi}$ ) stress amplitude and $44 \mathrm{~Hz}$ testing frequency

[8] Rockwell C (HRC) hardness measurement according to EN ISO 6508-1 on polished surface. Note that measured hardness can vary significantly depending on how the specimen has been prepared.

\section*{Material data sheet}
Thermal properties of parts

\begin{center}
\begin{tabular}{|c|c|}
\hline
 & As built \\
\hline
\multicolumn{2}{|l|}{Coefficient of thermal expansion} \\
\hline
- over 20 - $500^{\circ} \mathrm{C}\left(68-932^{\circ} \mathrm{F}\right)$ & \begin{tabular}{c}
typ. $13.6 \times 10^{-6} \mathrm{~m} / \mathrm{m}{ }^{\circ} \mathrm{C}$ \\
typ. $7.6 \times 10^{-6} \mathrm{in} / \mathrm{in}{ }^{\circ} \mathrm{F}$ \\
\end{tabular} \\
\hline
- over $500-1000^{\circ} \mathrm{C}\left(932-1832^{\circ} \mathrm{F}\right)$ & \begin{tabular}{c}
typ. $15.1 \times 10^{-6} \mathrm{~m} / \mathrm{m}^{\circ} \mathrm{C}$ \\
typ. $8.4 \times 10^{-6} \mathrm{in} / \mathrm{in}{ }^{\circ} \mathrm{F}$ \\
\end{tabular} \\
\hline
\multicolumn{2}{|l|}{Thermal conductivity} \\
\hline
- at $20^{\circ} \mathrm{C}\left(68^{\circ} \mathrm{F}\right)$ & \begin{tabular}{c}
typ. $13 \mathrm{~W} / \mathrm{m}^{\circ} \mathrm{C}$ \\
typ. $90 \mathrm{Btu} \mathrm{in} /\left(\mathrm{h} \mathrm{ft}^{2}{ }^{\circ} \mathrm{F}\right)$ \\
\end{tabular} \\
\hline
- at $300^{\circ} \mathrm{C}\left(572^{\circ} \mathrm{F}\right)$ & \begin{tabular}{c}
typ. $18 \mathrm{~W} / \mathrm{m}^{\circ} \mathrm{C}$ \\
typ. $125 \mathrm{Btu}$ in $/\left(\mathrm{h} \mathrm{ft}^{2}{ }^{\circ} \mathrm{F}\right)$ \\
\end{tabular} \\
\hline
- at $500^{\circ} \mathrm{C}\left(932^{\circ} \mathrm{F}\right)$ & \begin{tabular}{c}
typ. $22 \mathrm{~W} / \mathrm{m}^{\circ} \mathrm{C}$ \\
typ. $153 \mathrm{Btu} \mathrm{in} /\left(\mathrm{h} \mathrm{ft}^{2}{ }^{\circ} \mathrm{F}\right)$ \\
\end{tabular} \\
\hline
- at $1000^{\circ} \mathrm{C}\left(1832^{\circ} \mathrm{F}\right)$ & \begin{tabular}{c}
typ. $33 \mathrm{~W} / \mathrm{m}^{\circ} \mathrm{C}$ \\
typ. $229 \mathrm{Btu} \mathrm{in} /\left(\mathrm{h} \mathrm{ft}^{2}{ }^{\circ} \mathrm{F}\right)$ \\
\end{tabular} \\
\hline
Maximum operating temperature & \begin{tabular}{l}
approx. $1150^{\circ} \mathrm{C}$ \\
approx. $2100^{\circ} \mathrm{F}$ \\
\end{tabular} \\
\hline
Melting range & \begin{tabular}{l}
$1350-1430^{\circ} \mathrm{C}$ \\
$2460-2600^{\circ} \mathrm{F}$ \\
\end{tabular} \\
\hline
\end{tabular}
\end{center}

\section*{Abbreviations}
typ. typical

min. minimum

approx. approximately

wt weight

\section*{Material data sheet}
\section*{Notes}
The data are valid for the combinations of powder material machine and parameter sets referred to on page 1 , when used in accordance with the relevant Operating Instructions (including Installation Requirements and Maintenance) and Parameter Sheet. Part properties are measured using defined test procedures. Further details of the test procedures used by EOS are available on request. Unless otherwise specified, the data refer to the parameter set MP1\_Surface 1.0 or the equivalent default job CC20\_MP1\_020\_default.job. The corresponding data for the parameter set MP1\_Performance 1.0 or the equivalent default job CC20\_MP1\_040\_default.job are approximately the same except where otherwise specified.

The data correspond to our knowledge and experience at the time of publication. They do not on their own provide a sufficient basis for designing parts. Neither do they provide any agreement or guarantee about the specific properties of a part or the suitability of a part for a specific application. The producer or the purchaser of a part is responsible for checking the properties and the suitability of a part for a particular application. This also applies regarding any rights of protection as well as laws and regulations. The data are subject to change without notice as part of EOS' continuous development and improvement processes.

EOS , EOSINT and DMLS are registered trademarks of EOS GmbH.

2011 EOS GmbH - Electro Optical Systems. All rights reserved.

\section*{Copper C18150}
\section*{DIRECT METAL LASER SINTERING MATERIAL SPECIFICATIONS}
Highlights

\begin{itemize}
  \item Chromium Zirconium Copper (CuCr1Zr)

  \item Excellent electrical and thermal conductivity

  \item Mechanical and material properties stable up to $700^{\circ} \mathrm{F}$

  \item Age Hardenable

\end{itemize}

\section*{Applications:}
\begin{itemize}
  \item Excellent thermal and electrical conductivity

  \item Conformal induction coils

  \item Regeneratively cooled nozzles

  \item Plastic mold components

\end{itemize}

\section*{TYPICAL PHYSICAL PROPERTIES}
\begin{center}
\begin{tabular}{|c|c|c|c|c|c|}
\hline
MECHANICAL PROPERTIES (Z AXIS) & AS BUILT & SR & $S R+H I P$ & $\mathrm{SR}+\mathrm{HIP}+\mathrm{SHT}$ & \[
\begin{aligned} & $\mathrm{SR}+\mathrm{HIP} \\ + & \mathrm{SHT}+\mathrm{PHT}\end{aligned}
\]\$ \\
\hline
Yield Strength & $22 \mathrm{ksi}$ & 27 ksi & $16 \mathrm{ksi}$ & 17 ksi & 36 ksi \\
\hline
Ultimate Tensile Strength & $30 \mathrm{ksi}$ & $34.6 \mathrm{ksi}$ & $36.3 \mathrm{ksi}$ & $37 \mathrm{ksi}$ & $55 \mathrm{ksi}$ \\
\hline
Modulus & $103 \mathrm{GPa}$ & $96 \mathrm{GPa}$ & $117 \mathrm{GPa}$ & $110 \mathrm{GPa}$ & $124 \mathrm{GPa}$ \\
\hline
Elongation & $31 \%$ & $14 \%$ & $47 \%$ & $48 \%$ & $31 \%$ \\
\hline
Hardness & 90 Rockwell B & 108 Rockwell B & 80 Rockwell B & 94 Rockwell B & 87 Rockwell B \\
\hline
\end{tabular}
\end{center}

HEAT TREATMENT OPTIONS

\begin{itemize}
  \item Stress Relief (SR)
  \item SR + HIP + Solution Anneal (SHT)
  \item SR + Hot Isostatic Press (HIP)
  \item SR + HIP + SHT + Precipitation Strengthening (PHT)
\end{itemize}

\begin{center}
\includegraphics[max width=\textwidth]{2024_03_11_add5292786349c475313g-012}
\end{center}

*UNS C18150, AMS, and/or additional industry standards and designations

The information presented represents typical values intended for reference and comparison purposes only. It should not be used for design specifications or quality control purposes. End-use material performance can be impacted (+/-) by, but not limited to, part design, end-use conditions, test conditions, color etc. Actual values will vary with build conditions. Product specifications are subject to change without notice. ${ }^{*}$ Chemical analysis for specific lots available upon request.

The performance characteristics of these materials may vary according to application, operating conditions, or end use. Each user is responsible for determining that the material is safe, lawful, and technically suitable for the intended application. Stratasys makes no warranties of any kind, express or implied, including, but not limited to, the warranties of merchantability, fitness for a particular use, or warranty against patent infringement.

\section*{Materialdata sheet}
\section*{EOS Maraging Steel MS1}
EOSMaragingSteelMS1 is a steel powder which has been optimized especially for pocessing on EOSINT M systems.

This document provides information and data for parts built usingEOS MaragingSteel MS1 powder (EOS art.-no.9011-0016) on the following systemspecifications:

\begin{itemize}
  \item EOSINTM270Installation ModeStandard with PSW 3.3 or 3.4 and defaulbp MS1\_020\_default.jobor MS1\_040\_default.job
  \item EOSINTM270DualMode with PSW3.5 and EOSOriginalParameterSet MS1\_Surface1.0or MS1\_Performance2.0
  \item EOSINTM280
\end{itemize}

with PSW3.5and EOSOriginalParameterSet MS1\_Performance1.0or MS1\_Speed 1.0

\section*{Description}
Partsbuilt in EOSMaragingSteel MS1 have a chemical composition corresponding to US classification 18\%Ni Maraging 300, European 1.2709and GermanX3NiCoMoT118-9-5.This kind of steel is characterizedby having very good mechanical properties, and being easily heattreatable using a simple thermal age-hardening process to obtain excellent hardness and strength.

Partsbuilt from EOSMaragingSteelMS1 are easily machinable after the building process and can be easily post-hardened to more then $50 \mathrm{HRC}$ by age-hardening at $490^{\circ} \mathrm{C}\left(914^{\circ} \mathrm{F}\right)$ for 6 hours. In both as-built and age-hardened states the parts can be machined, spark-eroded, welded, micro shot-peened, polished and coated if required. Due to the layerwise building method, the parts have a certainanisotropy, which can be reduced or removed by appropriate heat treatment - see Tebnical Data fr examples.

\section*{Material data sheet}
\section*{Technical data}
\section*{General process data}
\begin{center}
\begin{tabular}{|c|c|}
\hline
\multicolumn{2}{|l|}{Typical achievable part accuracy [1]} \\
\hline
- small parts (< $80 \times 80 \mathrm{~mm})$ & \begin{tabular}{c}
approx $\pm 20 \mu \mathrm{m}$ \\
approx $\pm 0.8 \times 10^{3} \mathrm{inch}$ \\
\end{tabular} \\
\hline
- large parts & \begin{tabular}{c}
approx $\pm 50 \mu \mathrm{m}$ \\
approx $\pm 0.002$ inch \\
\end{tabular} \\
\hline
Age hardening shrinkage [2] & approx. $0.08 \%$ \\
\hline
Min. wall thickness [3] & \begin{tabular}{c}
approx. $0.3-0.4 \mathrm{~mm}$ \\
approx. $0.012-0.016$ inch \\
\end{tabular} \\
\hline
\multicolumn{2}{|l|}{Surface roughness (approx.) [4]} \\
\hline
\multicolumn{2}{|l|}{- as manufactured} \\
\hline
MS1 Surface (20 pm) & \begin{tabular}{l}
$\mathrm{Ra} 4 \mu \mathrm{m} ; \mathrm{Rz} 20 \mu \mathrm{m}$ \\
$\mathrm{Ra} 0.16 \times 10^{3}$ inch, \\
Rz $0.78 \times 10^{3}$ inch \\
\end{tabular} \\
\hline
MS1 Performance (40 $\mu \mathrm{m}$ ) & \begin{tabular}{l}
Ra $5 \mu \mathrm{m} ; \mathrm{Rz} 28 \mu \mathrm{m}$ \\
Ra $0.19 \times 10^{3}$ inch, \\
Rz $1.10 \times 10^{3}$ inch \\
\end{tabular} \\
\hline
MS1 Speed (50 um) & \begin{tabular}{l}
Ra $9 \mu \mathrm{m} ; \mathrm{Rz} 50 \mu \mathrm{m}$ \\
Ra $0.47 \times 10^{3}$ inch, \\
Rz $2.36 \times 10^{3}$ inch \\
\end{tabular} \\
\hline
- after shot- peening & \begin{tabular}{l}
$R_{a} 4-6.5 \mu \mathrm{m} ; R_{z} 20-50 \mu \mathrm{m}$ \\
$\quad R_{a} 0.16-0.26 \times 10^{\circ} \mathrm{inch}$ \\
$\quad R_{z} 0.78-1.97 \times 10^{\circ} \mathrm{inch}$ \\
\end{tabular} \\
\hline
- after polishing & \begin{tabular}{c}
$\mathrm{R}_{z}$ up to $<0.5 \mu \mathrm{m}$ \\
$\mathrm{R}_{z}$ up to $<0.02 \times 10^{3}$ inch \\
(can be very finely polished) \\
\end{tabular} \\
\hline
\end{tabular}
\end{center}

\section*{Material data sheet}
\section*{Volume rate [5]}
\begin{itemize}
  \item Parameter set MS1\_Surface 1.0 / default job MS1\_020\_default.job (20 $\mu$ m layer thickness)
\end{itemize}

\begin{center}
\begin{tabular}{c}
\hline
$1.6 \mathrm{~mm}^{3} / \mathrm{s}\left(5.8 \mathrm{~cm}^{3} / \mathrm{h}\right)$ \\
$0.35 \mathrm{in}^{3} / \mathrm{h}$ \\
\hline
$3 \mathrm{~mm}^{3} / \mathrm{s}\left(10.8 \mathrm{~cm}^{3} / \mathrm{h}\right)$ \\
$0.66 \mathrm{in}^{3} / \mathrm{h}$ \\
\hline
$\mathrm{W} \quad 4.2 \mathrm{~mm}^{3} / \mathrm{s}\left(15.1 \mathrm{~cm}^{3} / \mathrm{h}\right)$ \\
$0.92 \mathrm{in}^{3} / \mathrm{h}$ \\
\hline
$5.5 \mathrm{~mm}^{3} / \mathrm{s}\left(19.8 \mathrm{~cm}^{3} / \mathrm{h}\right)$ \\
$1.21 \mathrm{in}^{3} / \mathrm{h}$ \\
\end{tabular}
\end{center}

[1] Based on users' experience of dimensional accuracy for typical geometries, as built. Part accuracy is subject to appropriate data preparation and post-processing, in accordance with EOS training.

[2] Ageing temperature $490^{\circ} \mathrm{C}\left(914^{\circ} \mathrm{F}\right), 6$ hours, air cooling

[3] Mechanical stability is dependent on geometry (wall height etc.) and application

[4] Due to the layerwise building, the surface structure depends strongly on the orientation of the surface, for example sloping and curved surfaces exhibit a stair-step effect. The values also depend on the measurement method used. The values quoted here given an indication of what can be expected for horizontal (up-facing) or vertical surfaces.

[5] Volume rate is a measure of build speed during laser exposure of hatched areas. The total build speed depends on the average volume rate, the recoating time (related to the number of layers) and other geometry- and machine setting-related factors.

\section*{Material data sheet}
\section*{Physical and chemical properties of parts}
\begin{center}
\begin{tabular}{lc}
\hline
Material composition & $\mathrm{Fe}$ (balance) \\
 & $\mathrm{Ni}(17-19 \mathrm{wt}-\%)$ \\
$\mathrm{Co}(8.5-9.5 \mathrm{wt}-\%)$ &  \\
$\mathrm{Mo}(4.5-5.2 \mathrm{wt}-\%)$ &  \\
$\mathrm{Ti}(0.6-0.8 \mathrm{wt}-\%)$ &  \\
$\mathrm{Al}(0.05-0.15 \mathrm{wt}-\%)$ &  \\
$\mathrm{Cr}, \mathrm{Cu}($ each $\leq 0.5 \mathrm{wt}-\%)$ &  \\
$\mathrm{C}(\leq 0.03 \mathrm{wt}-\%)$ &  \\
$\mathrm{Mn}, \mathrm{Si}($ each $\leq 0.1 \mathrm{wt}-\%)$ &  \\
$\mathrm{P}, \mathrm{S}($ each $\leq 0.01 \mathrm{wt}-\%)$ &  \\
\hline
Relative density & approx. $100 \%$ \\
\hline
Density & $8.0-8.1 \mathrm{~g} / \mathrm{cm}^{3}$ \\
 & $0.289-0.293 \mathrm{lb} / \mathrm{in}^{3}$ \\
\hline
\end{tabular}
\end{center}

\section*{Material data sheet}
Mechanical properties of parts at $20^{\circ} \mathrm{C}\left(68^{\circ} \mathrm{F}\right)$

\begin{center}
\begin{tabular}{|c|c|c|}
\hline
 & As built & After age hardening [2] \\
\hline
Tensile strength [6] &  & \begin{tabular}{l}
min. $1930 \mathrm{MPa}$ \\
min. $280 \mathrm{ksi}$ \\
\end{tabular} \\
\hline
- in horizontal direction (XY) & \begin{tabular}{l}
typ. $1100 \pm 100 \mathrm{MPa}$ \\
typ. $160 \pm 15 \mathrm{ksi}$ \\
\end{tabular} & \multirow{2}{*}{}\begin{tabular}{c}
typ. $2050 \pm 100 \mathrm{MPa}$ \\
typ. $297 \pm 15 \mathrm{ksi}$ \\
\end{tabular} \\
\hline
- in vertical direction (Z) & \begin{tabular}{l}
typ. $1100 \pm 100 \mathrm{MPa}$ \\
typ. $160 \pm 15 \mathrm{ksi}$ \\
\end{tabular} &  \\
\hline
Yield strength (Rp $0.2 \%$ ) [6] &  & \begin{tabular}{c}
min. $1862 \mathrm{MPa}$ \\
$\min .270 \mathrm{ksi}$ \\
\end{tabular} \\
\hline
- in horizontal direction (XY) & \begin{tabular}{l}
typ. $1050 \pm 100 \mathrm{MPa}$ \\
typ. $152 \pm 15 \mathrm{ksi}$ \\
\end{tabular} & \multirow{2}{*}{}\begin{tabular}{l}
typ. $1990 \pm 100 \mathrm{MPa}$ \\
typ. $289 \pm 15 \mathrm{ksi}$ \\
\end{tabular} \\
\hline
- in vertical direction (Z) & \begin{tabular}{l}
typ. $1000 \pm 100 \mathrm{MPa}$ \\
typ. $145 \pm 15 \mathrm{ksi}$ \\
\end{tabular} &  \\
\hline
Elongation at break [6] &  & $\min .2 \%$ \\
\hline
- in horizontal direction (XY) & typ. $(10 \pm 4) \%$ & \multirow{2}{*}{typ. $(4 \pm 2) \%$} \\
\hline
- in vertical direction (Z) & typ. $(10 \pm 4) \%$ &  \\
\hline
\multicolumn{3}{|l|}{Modulus of elasticity [6]} \\
\hline
- in horizontal direction (XY) & \begin{tabular}{l}
typ. $160 \pm 25 \mathrm{GPa}$ \\
typ. $23 \pm 4 \mathrm{Msi}$ \\
\end{tabular} & \multirow{2}{*}{}\begin{tabular}{c}
typ. $180 \pm 20 \mathrm{GPa}$ \\
typ. $26 \pm 3 \mathrm{Msi}$ \\
\end{tabular} \\
\hline
- in vertical direction $(Z)$ & \begin{tabular}{l}
typ. $150 \pm 20 \mathrm{GPa}$ \\
typ. $22 \pm 3 \mathrm{Msi}$ \\
\end{tabular} &  \\
\hline
Hardness [7] & typ. 33 - 37 HRC & typ. 50 - 56 HRC \\
\hline
Ductility (Notched Charpy impact test) & typ. $45 \pm 10 \mathrm{~J}$ & typ. $11 \pm 4 \mathrm{~J}$ \\
\hline
\end{tabular}
\end{center}

[6] Tensile testing according to ISO 6891:2009 (B) Annex D, propotional test pieces, dianter of the neck area $5 \mathrm{~mm}$ (0.2 inch), orginal gauge lengt125 $\mathrm{mm}$ ( 1 inch).

[7] Rockwell C (HRC) hardness measurement according to EDD 6508-1 on polished surface. Note that measured hardness can vary significantly dependjion how the specimen has been prepared.

\section*{Material data sheet}
\section*{Thermal properties of parts}
\begin{center}
\begin{tabular}{lcc}
\hline
 & As built & After age hardening [2] \\
\hline
Thermal conductivity & typ. $15 \pm 0.8 \mathrm{~W} / \mathrm{m}^{\circ} \mathrm{C}$ & typ. $20 \pm 1 \mathrm{~W} / \mathrm{m}^{\circ} \mathrm{C}$ \\
 & typ. $104 \pm 6 \mathrm{Btu} \mathrm{in} /\left(\mathrm{h} \mathrm{ft}{ }^{\circ} \mathrm{F}\right)$ & typ. $139 \pm 7 \mathrm{Btu} \mathrm{in} /\left(\mathrm{h} \mathrm{ft}{ }^{\circ} \mathrm{F}\right)$ \\
\hline
Specific heat capacity & typ. $450 \pm 20 \mathrm{~J} / \mathrm{kg}^{\circ} \mathrm{C}$ & typ. $450 \pm 20 \mathrm{~J} / \mathrm{kg}^{\circ} \mathrm{C}$ \\
 & typ. $0.108 \pm 0.005 \mathrm{Btu} /\left(\mathrm{lb}{ }^{\circ} \mathrm{F}\right)$ & typ. $0.108 \pm 0.005 \mathrm{Btu} /\left(\mathrm{lb}{ }^{\circ} \mathrm{F}\right)$ \\
\hline
Maximum operating temperature &  & approx. $400{ }^{\circ} \mathrm{C}$ \\
 &  & approx. $750^{\circ} \mathrm{F}$ \\
\hline
\end{tabular}
\end{center}

\section*{Abbreviations}
typ. typical

min. minimum

approx. approximately

wt weight

\section*{Notes}
The data are valid for the combinations of powder material machine and parameter sets referred to on page 1 , when used in accordance with the relevant Operating Instructions (including Installation Requirements and Maintenance) and Parameter Sheet. Part properties are measured using defined test procedures. Further details of the test procedures used by EOS are available on request. Unless otherwise specified, the data refer to the default job MS1\_040\_default.jobr the equivalent parameter set MS1\_Performance 2.0. The corresponding data for the default job MS1\_020\_default.job or the equivalent parameter set MS1\_Surface 1.0 are approximately the same except where otherwise specified.

The data correspond to our knowledge and experience at the time of publication. They do not on their own provide a sufficient basis for designing parts. Neither do they provide any agreement or guarantee about the specific properties of a part or the suitability of a part for a specific application. The producer or the purchaser of a part is responsible for checking the properties and the suitability of a part for a particular application. This also applies regarding any rights of protection as well as laws and regulations. The data are subject to change without notice as part of EOS' continuous development and improvement processes.

EOS, EOSINTand DMLS are registered trademarks of EOS GmbH.

凶 2011 EOS GmbH - Electro Optical Systems. All rights reserved.

\section*{Material data sheet}
\section*{EOS NickelAlloy IN625}
EOS NickelAlloy IN625 is a heat and corrosion resistant nickel alloy powder which has been optimized especially for processing on EOSINT M systems.

This document provides information and data for parts built using EOS NickelAlloy IN625 powder (EOS art.-no. 9011-0022) on the following system specifications:

EOSINT M 270 Dual Mode

with PSW 3.5 and EOS Original Parameter Set IN625\_Surface 1.0

\section*{Description}
Parts built from EOS NickelAlloy IN625 have chemical composition corresponding to UNS N06625, AMS 5666F, AMS 5599G, \href{http://W.Nr}{W.Nr} 2.4856, DIN NiCr22Mo9Nb. This type of alloy is characterized by having high tensile, creep and rupture strength. Conventionally cast or wrought components in this type of nickel alloy have typically excellent fatigue and thermalfatigue properties combined with good oxidation resistance.

EOS NickelAlloy IN625 is expected to have good corrosion resistance in various corrosive environments. Especially sea-water applications require high pitting and crevice corrosion resistance, stress-corrosion resistance against chloride-ions, high tensile and corrosion-fatigue strength. However, corrosion resistance has not been verified yet and therefore it is recommended to conduct relevant corrosion tests and studies prior to use in specific corrosive environment.

Parts built from EOS NickelAlloy IN625 can be heat treated and material properties can be varied within specified range. Parts can be machined, spark-eroded, welded, micro shot-peened, polished and coated in both as-built and in heat treated conditions. Due to the layerwise building method, the parts have a certain anisotropy - see Technical Data for examples.

\section*{Material data sheet}
\section*{Technical data}
\section*{General process data}
\begin{center}
\begin{tabular}{|c|c|}
\hline
Typical achievable part accuracy [1] &  \\
\hline
- small parts & \begin{tabular}{c}
approx $\pm 40-60 \mu \mathrm{m}$ \\
approx $\pm 1.6-2.4 \times 10^{3}$ inch \\
\end{tabular} \\
\hline
- large parts & $\pm 0.2 \%$ \\
\hline
Min. wall thickness [2] & \begin{tabular}{c}
approx. $0.3-0.4 \mathrm{~mm}$ \\
approx. $0.012-0.016$ inch \\
\end{tabular} \\
\hline
Surface roughness [3] &  \\
\hline
- after shot-peening & \begin{tabular}{l}
Ra $4-6.5 \mu \mathrm{m}, \mathrm{R} 20-50 \mu \mathrm{m}$ \\
Ra $0.16-0.26 \times 10^{3}$ inch, \\
Rz $0.78-1.97 \times 10^{3}$ inch \\
\end{tabular} \\
\hline
- after polishing & \begin{tabular}{l}
$\mathrm{R}_{z}$ up to $<0.5 \mu \mathrm{m}$ \\
$\mathrm{R}_{z}$ up to $<0.02 \times 10^{3}$ inch \\
(can be very finely polished) \\
\end{tabular} \\
\hline
Volume rate [4] & \begin{tabular}{c}
$2 \mathrm{~mm}^{3} / \mathrm{s}\left(7.2 \mathrm{~cm}^{3} / \mathrm{h}\right)$ \\
$0.44 \mathrm{in}^{3} / \mathrm{h}$ \\
\end{tabular} \\
\hline
\end{tabular}
\end{center}

[1] Based on users' experience of dimensional accuracy for typical geometries, e.g. $\pm 40 \mu \mathrm{m}$ ( 0.0016 inch) when parameters can be optimized for a certain class of parts or $\pm 60 \mu \mathrm{m}$ ( 0.0024 inch) when building a new kind of geometry for the first time. Part accuracy is subject to appropriate data preparation and post-processing, in accordance with EOS training.

[2] Mechanical stability is dependent on geometry (wall height etc.) and application

[3] Due to the layerwise building, the surface structure depends strongly on the orientation of the surface, for example sloping and curved surfaces exhibit a stair-step effect. The values also depend on the measurement method used. The values quoted here given an indication of what can be expected for horizontal (up-facing) or vertical surfaces.

[4] Volume rate is a measure of build speed during laser exposure. The total build speed depends on the average volume rate, the recoating time (related to the number of layers) and other factors such as DMLS-Start settings.

\section*{Material data sheet}
Physical and chemical properties of parts

\begin{center}
\begin{tabular}{lc}
\hline
Material composition & $\mathrm{Ni}($ balance $\geq 58.00 \mathrm{wt}-\%)$ \\
$\mathrm{Cr}(20.00-23.00 \mathrm{wt}-\%)$ &  \\
$\mathrm{Mo}(8.00-10.00 \mathrm{wt}-\%)$ &  \\
$\mathrm{Nb}(3.15-4.15 \mathrm{wt}-\%)$ &  \\
$\mathrm{Fe}(\leq 5.00 \mathrm{wt}-\%)$ &  \\
$\mathrm{Ti}(\leq 0.40 \mathrm{wt}-\%)$ &  \\
$\mathrm{Al}(\leq 0.40 \mathrm{wt}-\%)$ &  \\
$\mathrm{Co}(\leq 1.0 \mathrm{wt}-\%)$ &  \\
$\mathrm{C}(\leq 0.10 \mathrm{wt}-\%)$ &  \\
$\mathrm{Ta}(\leq 0.05 \mathrm{wt}-\%)$ &  \\
$\mathrm{Si}, \mathrm{Mn}(\mathrm{each} \leq 0.50 \mathrm{wt}-\%)$ &  \\
$\mathrm{P}, \mathrm{S}($ each $\leq 0.015 \mathrm{wt}-\%)$ &  \\
\hline
Relative density & approx. $100 \%$ \\
\hline
Density & min. $8.4 \mathrm{~g} / \mathrm{cm}^{3}$ \\
 & min. $0.30 \mathrm{lb} / \mathrm{in}^{3}$ \\
\hline
\end{tabular}
\end{center}

\section*{Material data sheet}
Mechanical properties of parts at $20^{\circ} \mathrm{C}\left(68^{\circ} \mathrm{F}\right)$

\begin{center}
\begin{tabular}{|c|c|c|}
\hline
 & As built & Stress relieved [6] \\
\hline
\multicolumn{3}{|l|}{Tensile strength [5]} \\
\hline
- in horizontal direction (XY) & \begin{tabular}{l}
typ. $990 \pm 50 \mathrm{MPa}$ \\
typ. $144 \pm 7 \mathrm{ksi}$ \\
\end{tabular} & \begin{tabular}{c}
$\min .827 \mathrm{MPa}(120 \mathrm{ksi})$ \\
typ. $1040 \pm 100 \mathrm{MPa}(151 \pm 15 \mathrm{ksi})$ \\
\end{tabular} \\
\hline
- in vertical direction (Z) & \begin{tabular}{l}
typ. $900 \pm 50 \mathrm{MPa}$ \\
typ. $131 \pm 7 \mathrm{ksi}$ \\
\end{tabular} & \begin{tabular}{c}
$\min .827 \mathrm{MPa}(120 \mathrm{ksi})$ \\
typ. $930 \pm 100 \mathrm{MPa}(135 \pm 15 \mathrm{ksi})$ \\
\end{tabular} \\
\hline
\multicolumn{3}{|l|}{Yield strength, Rp0.2\% [5]} \\
\hline
- in horizontal direction (XY) & \begin{tabular}{l}
typ. $725 \pm 50 \mathrm{MPa}$ \\
typ. $105 \pm 7 \mathrm{ksi}$ \\
\end{tabular} & \begin{tabular}{c}
$\min .414 \mathrm{MPa}(60 \mathrm{ksi})$ \\
typ. $720 \pm 100 \mathrm{MPa}(104 \pm 15 \mathrm{ksi})$ \\
\end{tabular} \\
\hline
- in vertical direction (Z) & \begin{tabular}{l}
typ. $615 \pm 50 \mathrm{MPa}$ \\
typ. $89 \pm 7 \mathrm{ksi}$ \\
\end{tabular} & \begin{tabular}{c}
$\min .414 \mathrm{MPa}(60 \mathrm{ksi})$ \\
typ. $650 \pm 100 \mathrm{MPa}(94 \pm 15 \mathrm{ksi})$ \\
\end{tabular} \\
\hline
\multicolumn{3}{|l|}{Modulus of elasticity [5]} \\
\hline
- in horizontal direction (XY) & \begin{tabular}{l}
typ. $170 \pm 20 \mathrm{GPa}$ \\
typ. $25 \pm 3 \mathrm{Msi}$ \\
\end{tabular} & \begin{tabular}{l}
typ. $170 \pm 20 \mathrm{GPa}$ \\
typ. $25 \pm 3 \mathrm{Msi}$ \\
\end{tabular} \\
\hline
- in vertical direction (Z) & \begin{tabular}{l}
typ. $140 \pm 20 \mathrm{GPa}$ \\
typ. $20 \pm 3 \mathrm{Msi}$ \\
\end{tabular} & \begin{tabular}{l}
typ. $160 \pm 20 \mathrm{GPa}$ \\
typ. $23 \pm 3 \mathrm{Msi}$ \\
\end{tabular} \\
\hline
\multicolumn{3}{|l|}{Elongation at break [5]} \\
\hline
- in horizontal direction (XY) & $\operatorname{typ}(35 \pm 5) \%$ & min. $30 \%$, typ. $(35 \pm 5) \%$ \\
\hline
- in vertical direction (Z) & $(42 \pm 5) \%$ & $\min .30 \%$, typ. $(44 \pm 5) \%$ \\
\hline
Hardness [7] &  & \begin{tabular}{l}
approx. $30 \mathrm{HRC}$ \\
$(287 \mathrm{HB})$ \\
\end{tabular} \\
\hline
\end{tabular}
\end{center}

[5] Tensile testing according to ISO 6892-1:2009 (B) Annex D, proportional test pieces, diameter of the neck area $5 \mathrm{~mm}$ ( 0.2 inch), original gauge length $25 \mathrm{~mm}$ ( 1 inch).

[6] Stress relieve: anneal at $870^{\circ} \mathrm{C}\left(1600^{\circ} \mathrm{F}\right)$ for 1 hour, rapid cooling

[7] Rockwell C (HRC) hardness measurement according to EN ISO 6508-1 on polished surface. Note that measured hardness can vary significantly depending on how the specimen has been prepared.

\section*{Material data sheet}
\section*{Thermal properties of parts}
\begin{center}
\begin{tabular}{lc}
\hline
Maximum operating temperature for parts under & approx. $650^{\circ} \mathrm{C}$ \\
load & approx. $1200^{\circ} \mathrm{F}$ \\
\hline
Oxidation resistance to [8] & $980^{\circ} \mathrm{C}$ \\
 & $1800^{\circ} \mathrm{F}$ \\
\hline
\end{tabular}
\end{center}

[8] Based on literature of conventional Ni-alloy with identical chemistry

\section*{Abbreviations}
typ. typical

min. minimum

approx. approximately

wt weight

\section*{Notes}
The data are valid for the combinations of powder material machine and parameter sets referred to on page 1 , when used in accordance with the relevant Operating Instructions (including Installation Requirements and Maintenance) and Parameter Sheet. Part properties are measured using defined test procedures. Further details of the test procedures used by EOS are available on request.

The data correspond to our knowledge and experience at the time of publication. They do not on their own provide a sufficient basis for designing parts. Neither do they provide any agreement or guarantee about the specific properties of a part or the suitability of a part for a specific application. The producer or the purchaser of a part is responsible for checking the properties and the suitability of a part for a particular application. This also applies regarding any rights of protection as well as laws and regulations. The data are subject to change without notice as part of EOS' continuous development and improvement processes.

EOS, EOSINT and DMLS are registered trademarks of EOS GmbH.

2011 EOS GmbH - Electro Optical Systems. All rights reserved.

\section*{Material data sheet}
\section*{EOS NickelAlloy IN718}
EOS NickelAlloy IN718 is a heat and corrosion resistant nickel alloy powder which has been optimized especially for processing on EOS M systems.

This document provides information and data for parts built using EOS NickelAlloy IN718 powder (EOS art.-no. 9011-0020) on thefollowing system specifications:

EOSINT M280 400W System with PSW3.6 and Parameter Set IN718\_Performace 1.0

EOS M290 400W System with EOSPRINT 1.0 and Parameter Set IN718\_Performace 1.0

\section*{Description}
Parts built from EOS NickelAlloy IN718 have chemical composition corresponding to UNS N07718, AMS 5662, AMS 5664, \href{http://W.Nr}{W.Nr} 2.4668, DIN NiCr19Fe19NbMo3. This kind of precipitation-hardening nickel-chromium alloy is characterized by having good tensile, fatigue, creep and rupture strength at temperatures up to $700^{\circ} \mathrm{C}\left(1290^{\circ} \mathrm{F}\right)$.

This material is ideal for many high temperature applications such as gas turbine parts, instrumentation parts, power and process industry parts etc. It also has excellent potential for cryogenic applications.

Parts built from EOS NickelAlloy IN718 can be easily post-hardened by precipitation-hardening heat treatments. In both as-built and age-hardened states the parts can be machined, sparkeroded, welded, micro shot-peened, polished and coated if required. Due to the layerwise building method, the parts have a certain anisotropy - see Technical Data for examples.

\section*{Material data sheet}
\section*{Technical data}
\section*{General process data}
\section*{Typical achievable part accuracy [1], [11]}
\begin{itemize}
  \item small parts
\end{itemize}

approx. $\pm 40-60 \mu \mathrm{m}$

approx. $\pm 1.6-2.4 \times 10{ }^{.3}$ inch

\begin{itemize}
  \item large parts
\end{itemize}

$\pm 0.2 \%$

Min. wall thickness [2], [11]

typ. $0.3-0.4 \mathrm{~mm}$

typ. $0.012-0.016$ inch

\section*{Surface roughness [3], [11]}
\begin{itemize}
  \item after shot-peening
\end{itemize}

Ra $4-6.5 \mu \mathrm{m}, \mathrm{Rz} 20-50 \mu \mathrm{m}$

Ra $0.16-0.25 \times 10^{3}$ inch,

Rz $0.78-1.97 \times 10^{3}$ inch

\begin{itemize}
  \item after polishing
\end{itemize}

$\mathrm{Rz}_{\mathrm{z}}$ up to $<0.5 \mu \mathrm{m}$

$\mathrm{R}_{z}$ up to $<0.02 \times 10^{-3}$ inch

(can be very finely polished)

\section*{Volume rate [4]}
\begin{itemize}
  \item Parameter Set IN718\_Performance (40 $\mu$ m)
\end{itemize}

$$
\begin{gathered}
4 \mathrm{~mm}^{3} / \mathrm{s}\left(14.4 \mathrm{~cm}^{3} / \mathrm{h}\right) \\
0.88 \mathrm{in}^{3} / \mathrm{h}
\end{gathered}
$$

[1] Based on users' experience of dimensional accuracy for typical geometries, e.g. $\pm 40 \mu \mathrm{m}$ ( $1.6 \times 10^{3}$ inch) when parameters can be optimized for a certain class of parts or $\pm 60 \mu \mathrm{m}$ ( $2.4 \times 10^{3}$ inch) when building a new kind of geometry for the first time. Part accuracy is subject to appropriate data preparation and postprocessing, in accordance with EOS training.

[2] Mechanical stability is dependent on geometry (wall height etc.) and application\#

[3] Due to the layerwise building, the surface structure depends strongly on the orientation of the surface, for example sloping and curved surfaces exhibit a stair-step effect. The values also depend on the measurement method used. The values quoted here given an indication of what can be expected for horizontal (up-facing) or vertical surfaces.

[4] Volume rate is a measure of build speed during laser exposure. The total build speed depends on the average volume rate, the recoating time (related to the number of layers) and other factors such as DMLS-Start settings.

\section*{Material data sheet}
Physical and chemical properties of parts

\begin{center}
\begin{tabular}{lc}
\hline
Material composition & $\mathrm{Ni}(50-55 \mathrm{wt}-\%)$ \\
$\mathrm{Cr}(17.0-21.0 \mathrm{wt}-\%)$ &  \\
$\mathrm{Nb}(4.75-5.5 \mathrm{wt}-\%)$ &  \\
$\mathrm{Mo}(2.8-3.3 \mathrm{wt}-\%)$ &  \\
$\mathrm{Ti}(0.65-1.15 \mathrm{wt}-\%)$ &  \\
$\mathrm{Al}(0.20-0.80 \mathrm{wt}-\%)$ &  \\
$\mathrm{Co}(\leq 1.0 \mathrm{wt}-\%)$ &  \\
$\mathrm{Cu}(\leq 0.3 \mathrm{wt}-\%)$ &  \\
$\mathrm{C}(\leq 0.08 \mathrm{wt}-\%)$ &  \\
 & $\mathrm{Si}, \mathrm{Mn}($ each $\leq 0.35 \mathrm{wt}-\%)$ \\
$\mathrm{P}, \mathrm{S}($ each $\leq 0.015 \mathrm{wt}-\%)$ &  \\
$\mathrm{B}(\leq 0.006 \mathrm{wt}-\%)$ &  \\
$\mathrm{Fe}(\mathrm{balance})$ &  \\
\hline
Relative density & approx. $100 \%$ \\
\hline
Density & min. $8.15 \mathrm{~g} / \mathrm{cm}^{3}$ \\
 & $\min .0 .294 \mathrm{lb} / \mathrm{in}^{3}$ \\
\hline
\end{tabular}
\end{center}

\section*{Material data sheet}
Mechanical properties of parts at $20^{\circ} \mathrm{C}\left(68^{\circ} \mathrm{F}\right)$

\begin{center}
\begin{tabular}{|c|c|c|c|}
\hline
 & As built & \begin{tabular}{l}
Heat treated per \\
AMS $5662[5]$ \\
\end{tabular} & \begin{tabular}{l}
Heat treated per AMS \\
$5664[6]$ \\
\end{tabular} \\
\hline
\multicolumn{4}{|l|}{Tensile strength [7]} \\
\hline
- in horizontal direction (XY) & \begin{tabular}{c}
typ. $1060 \pm 50 \mathrm{MPa}$ \\
$(154 \pm 7 \mathrm{ksi})$ \\
\end{tabular} &  &  \\
\hline
- in vertical direction $(Z)$ & \begin{tabular}{l}
typ. $980 \pm 50 \mathrm{MPa}$ \\
$(142 \pm 7 \mathrm{ksi})$ \\
\end{tabular} & \begin{tabular}{c}
min. $1241 \mathrm{MPa}(180 \mathrm{ksi})$ \\
typ. $1400 \pm 100 \mathrm{MPa}$ \\
$(203 \pm 15 \mathrm{ksi})$ \\
\end{tabular} & \begin{tabular}{c}
min. $1241 \mathrm{MPa}(180 \mathrm{ksi})$ \\
typ. $1380 \pm 100 \mathrm{MPa}$ \\
$(200 \pm 15 \mathrm{ksi})$ \\
\end{tabular} \\
\hline
\end{tabular}
\end{center}

Yield strength (Rp $0.2 \%$ ) [7]

\begin{itemize}
  \item in horizontal direction (XY) typ. $780 \pm 50 \mathrm{MPa}$
\end{itemize}

$$
(113 \pm 7 \mathrm{ksi})
$$

\begin{itemize}
  \item in vertical direction (Z) typ. $634 \pm 50 \mathrm{MPa}$ min. $1034 \mathrm{MPa}(150 \mathrm{ksi}) \min .1034 \mathrm{MPa}$ (150 ksi)
\end{itemize}

$$
(92 \pm 7 \mathrm{ksi}) \quad \text { typ. } 1150 \pm 100 \mathrm{MPa} \quad \text { typ. } 1240 \pm 100 \mathrm{MPa}
$$

$$
(167 \pm 15 \mathrm{ksi}) \quad(180 \pm 15 \mathrm{ksi})
$$

\section*{Elongation at break [7]}
\begin{itemize}
  \item in horizontal direction (XY)
  \item in vertical direction $(Z)$
\end{itemize}

$$
\begin{array}{ccc}
\text { typ. }(27 \pm 5) \% & & \\
\hline \text { typ. }(31 \pm 5) \% & \min .12 \% & \min .12 \% \\
& \text { typ. }(15 \pm 3) \% & \text { typ. }(18 \pm 5) \%
\end{array}
$$

Modulus of elasticity [7]

\begin{itemize}
  \item in horizontal direction (XY) typ. $160 \pm 20 \mathrm{GPa}$
\end{itemize}

$(23 \pm 3 \mathrm{Msi})$

\begin{itemize}
  \item in vertical direction $(Z)$
\end{itemize}

\begin{center}
\begin{tabular}{lll}
\hline
 & $170 \pm 20 \mathrm{GPa}$ & $170 \pm 20 \mathrm{GPa}$ \\
 & $24.7 \pm 3 \mathrm{Msi}$ & $24.7 \pm 3 \mathrm{Msi}$ \\
\hline
approx.30 HRC & approx. 47 HRC & approx. 43 HRC \\
approx. 287 HB & approx.446 HB & approx. 400 HB \\
\hline
\end{tabular}
\end{center}

[5] Heat treatment procedure per AMS 5662:

\begin{enumerate}
  \item Solution Anneal at $980^{\circ} \mathrm{C}\left(1800^{\circ} \mathrm{F}\right)$ for 1 hour, air (/argon) cool.

  \item Ageing treatment; hold at $720^{\circ} \mathrm{C}\left(1330^{\circ} \mathrm{F}\right) 8$ hours, furnace cool to $620^{\circ} \mathrm{C}\left(1150{ }^{\circ} \mathrm{F}\right)$ in 2 hours, hold at $620^{\circ} \mathrm{C}\left(1150^{\circ} \mathrm{F}\right) 8$ hours, air (/argon) cool.

\end{enumerate}

\section*{Material data sheet}
[6] Heat treatment procedure per AMS 5664:

\begin{enumerate}
  \item Solution Anneal at $1065^{\circ} \mathrm{C}\left(1950^{\circ} \mathrm{F}\right)$ for 1 hour, air (/argon) cool.

  \item Ageing treatment; hold at $760^{\circ} \mathrm{C}\left(1400^{\circ} \mathrm{F}\right) 10$ hours, furnace cool to $650^{\circ} \mathrm{C}\left(1200^{\circ} \mathrm{F}\right)$ in 2 hours, hold at $650^{\circ} \mathrm{C}\left(1200^{\circ} \mathrm{F}\right) 8$ hours, air (/argon) cool

\end{enumerate}

[7] Tensile testing according to ISO 6892 -1:2009 (B) Annex D, proportional test pieces, diameter of the neck area $5 \mathrm{~mm}$ ( 0.2 inch), original gauge length $25 \mathrm{~mm}$ ( 1 inch).

[8] Rockwell C (HRC) hardness measurement according to EN ISO 6508-1 on polished surface. Note that measured hardness can vary significantly depending on how the specimen has been prepared.

\section*{Material data sheet}
Mechanical properties of parts at hi gh temperature $\left(649^{\circ} \mathrm{C}, 1200^{\circ} \mathrm{F}\right)[11]$

\begin{center}
\begin{tabular}{|c|c|c|}
\hline
 & \begin{tabular}{l}
Heat treated per \\
AMS $5662[5]$ \\
\end{tabular} & \begin{tabular}{l}
Heat treated per \\
AMS 5664 [6] \\
\end{tabular} \\
\hline
\multicolumn{3}{|l|}{Tensile Strength (Rm) [9]} \\
\hline
- in vertical direction (Z) & \begin{tabular}{c}
$\min .965 \mathrm{MPa}(140 \mathrm{ksi})$ \\
typ. $1170 \pm 50 \mathrm{MPa}$ \\
$(170 \pm 7 \mathrm{ksi})$ \\
\end{tabular} & \begin{tabular}{l}
typ. $1210 \pm 50 \mathrm{MPa}$ \\
$(175 \pm 7 \mathrm{ksi})$ \\
\end{tabular} \\
\hline
\multicolumn{3}{|l|}{Yield strength (Rp $0.2 \%$ \%) [9]} \\
\hline
- in vertical direction (Z) & \begin{tabular}{c}
$\min .862 \mathrm{MPa}$ (125 ksi) \\
typ. $970 \pm 50 \mathrm{MPa}$ \\
$(141 \pm 7 \mathrm{ksi})$ \\
\end{tabular} & \begin{tabular}{l}
typ. $1010 \pm 50 \mathrm{MPa}$ \\
$(146 \pm 7 \mathrm{ksi})$ \\
\end{tabular} \\
\hline
\multicolumn{3}{|l|}{Elongation at break [9]} \\
\hline
- in vertical direction (Z) & \begin{tabular}{c}
$\min .6 \%$ \\
typ. $(16 \pm 3) \%$ \\
\end{tabular} & typ. $(20 \pm 3) \%$ \\
\hline
\multicolumn{3}{|l|}{Stress-Rupture Properties [10]} \\
\hline
\multirow[t]{2}{*}{- in vertical direction (Z)} & \begin{tabular}{c}
min. 23 hours at stress \\
level $689 \mathrm{MPa}$ \\
$(100 \mathrm{ksi})$ \\
\end{tabular} &  \\
\hline
 & \begin{tabular}{l}
$51 \pm 5$ hours \\
(final applied stress to \\
rupture $792.5 \mathrm{MPa} /$ \\
$115 \mathrm{ksi}$ ) \\
\end{tabular} & \begin{tabular}{c}
$81 \pm 10$ hours \\
(final applied stress to \\
rupture $861.5 \mathrm{MPa} /$ \\
$125 \mathrm{ksi}$ ) \\
\end{tabular} \\
\hline
\end{tabular}
\end{center}

[9] Elevated temperature tensile testing at $649^{\circ} \mathrm{C}\left(1200^{\circ} \mathrm{F}\right)$ in accordance with EN 10002-5 (92)

[10] Testing at $649^{\circ} \mathrm{C}\left(1200^{\circ} \mathrm{F}\right)$ in accordance with ASTME139 (2006), smooth specimens. Test method as described in AMS 5662 (3.5.1.2.3.3): "The load required to produce an initial axial stress of $689 \mathrm{MPa}$ (100 ksi) shall be used to rupture or for 23 hours, whichever occurs first. After the 23 hours and at intervals of 8 hours minimum, thereafter, the stress shall beareased in increments of $34.5 \mathrm{MPa}$ ( $5 \mathrm{ksi})$."

[11] Hint: these properties were determined on an EOSINT M 270 IM Xtended and EOSINT M 280-400W. Test parts from following machine types EOSINT M 270 Dual Mode, EOSINT M 280-200W and EOS M 290-400W correspond with these data.

\section*{Material data sheet}
\section*{Thermal properties of parts}
\begin{center}
\begin{tabular}{|c|c|}
\hline
 & Heat treated per AMS $5662[4]$ \\
\hline
\multicolumn{2}{|l|}{Coefficient of thermal expansion} \\
\hline
- over $25-200^{\circ} \mathrm{C}\left(36-390^{\circ} \mathrm{F}\right)$ & \begin{tabular}{c}
approx. $12.5-13.0 \times 10^{-6} \mathrm{~m} / \mathrm{m}^{\circ} \mathrm{C}$ \\
approx. $6.9-7.2 \times 10^{-6} \mathrm{in} / \mathrm{in}^{\circ} \mathrm{F}$ \\
\end{tabular} \\
\hline
- over 25 - $750^{\circ} \mathrm{C}\left(36-930^{\circ} \mathrm{F}\right)$ & \begin{tabular}{c}
approx. $16.6-17.2 \times 10^{-6} \mathrm{~m} / \mathrm{m}^{\circ} \mathrm{C}$ \\
approx. $9.2-9.6 \times 10^{-6} \mathrm{in} / \mathrm{in}^{\circ} \mathrm{F}$ \\
\end{tabular} \\
\hline
Maximum operating temperature for parts under load & \begin{tabular}{l}
approx. $650^{\circ} \mathrm{C}$ \\
approx. $1200^{\circ} \mathrm{F}$ \\
\end{tabular} \\
\hline
Oxidation resistance up to [11] & \begin{tabular}{l}
approx. $980^{\circ} \mathrm{C}$ \\
approx. $1800^{\circ} \mathrm{F}$ \\
\end{tabular} \\
\hline
\end{tabular}
\end{center}

[12] Based on literature of conventional Ni-alloy with identical chemistry

\section*{Abbreviations}
$$
\begin{aligned}
\text { typ. } & \text { typical } \\
\text { min. } & \text { minimum } \\
\text { approx. } & \text { approximately } \\
\text { wt } & \text { weight }
\end{aligned}
$$

\section*{Notes}
The data are valid for the combinations of powder material machine and parameter sets referred to on page 1 , when used in accordance with the relevant Operating Instructions (including Installation Requirements and Maintenance) and Parameter Sheet. Part properties are measured using defined test procedures. Further details of the test procedures used by EOS are available on request.

The data correspond to our knowledge and experience at the time of publication. They do not on their own provide a sufficient basis for designing parts. Neither do they provide any agreement or guarantee about the specific properties of a part or the suitability of a part for a specific application. The producer or the purchaser of a part is responsible for checking the properties and the suitability of a part for a particular application. This also applies regarding any rights of protection as well as laws and regulations. The data are subject to change without notice as part of EOS' continuous development and improvement processes.

EOS , EOSINT and DMLS are registered trademarks of EOS GmbH.

2014 EOS GmbH - Electro OptidaSystems. All rights reserved.

\section*{Material data sheet}
\section*{EOS StainlessSteel 316L}
EOS StainlessSteel $316 \mathrm{~L}$ is a corrosion resistant iron based alloy which has been optimized especially for processing onEOSINT M280 systems.

This document provides information and data for parts built using EOS StainlessSted $316 \mathrm{~L}$ powder (EOS art.-no. 9011-0032) on the following system specifications:

\begin{itemize}
  \item EOSINT M280 400W System with PSW3.6 and Parameter Set 316L\_Surface 1.0
  \item $\quad$ EOSINT M280 200W System with PSW3.6 and Parameter Set 316L\_Surface 1.0
\end{itemize}

\section*{Description}
The parts built from EOS StainlessSteel 316L have chemical composition correspondingto ASTM F138 "Standard Specification for Wrought 18Cr-14Ni-2.5Mo Stainless Steel Bar and Wire for Surgical Implants (UNS S31673)". This kind of stainless steel is characterized having a goodcorrosion resistance and evidence that there are no leachable substances in cytotoxic concentrations.

This material is ideal in

\begin{itemize}
  \item Lifestyle/Consumer - watches, other jewellery, spectacle frames, decorations, functional elements in electronic housing and accessories
  \item Automaive/Industrial - non-corrodingcommon material, foodand chemical plants
  \item Aerospace/Turbine industry - entry-level material for Laser Sintering Technology, mourting parts, brackets, heat exchangers
\end{itemize}

Parts built from EOS StainlessSteel $316 \mathrm{~L}$ can be machined, shot-peened and polshed in as- built or stress relieved (AMS2759) states if required. Solution annealing is not necessary because the mechanical properties of as-built state are showing desired values (ASTM A403). Parts are not ideal in temperature range $427^{\circ} \mathrm{C}-816^{\circ} \mathrm{C}$ where precipitation of chromiumcarbides occurs. Due to layer-wise building method, the parts have a certain anisotropy which could be seen from mechanical properties.

\section*{Material data sheet}
\section*{Technical data}
General process data

EOS StainlessSteel 316L

\begin{center}
\includegraphics[max width=\textwidth]{2024_03_11_add5292786349c475313g-032}
\end{center}

[1] Based on users' experience of dimensional accuracy for typical geometries, e.g. $\pm 40 \mu \mathrm{m}$ when parameters can be optimized for a certain class of parts or $\pm 60 \mu \mathrm{m}$ when building a new kind of geometry for the first time. Part accuracy is subject to appropriate data preparation and postprocessing.

[2] Mechanical stability is dependent on geometry (wall height etc.) and application

[3] Due to the layerwise building, the surface structure depends strongly on the orientation of the surface, for exampk sloping and curved surfaces exhibit a stair-step effect. The values also depend on the measurement method used. The values quoted here given an indication of what can be expected for vertical surfaces.

[4] Volume rate is a measure of build speed during laser exposure. The total build speed depends on the average volume rate, the recoating time (related to the number of layers) and other factors such as contour and Up/DownSkin parameters.

\section*{Material data sheet}
Physical and chemical properties of parts

EOS StainlessSteel 316L

\begin{center}
\begin{tabular}{|c|c|c|c|}
\hline
\multirow{12}{*}{Material compøition} & Element & Min & Max \\
\hline
 & $\mathrm{Fe}$ & \multicolumn{2}{|c|}{balance} \\
\hline
 & $\mathrm{Cr}$ & 17.00 & 19.00 \\
\hline
 & $\mathrm{Ni}$ & 13.00 & 15.00 \\
\hline
 & Mo & 2.25 & 3.00 \\
\hline
 & C &  & 0.030 \\
\hline
 & $M n$ &  & 2.00 \\
\hline
 & $\mathrm{Cu}$ &  & 0.50 \\
\hline
 & P &  & 0.025 \\
\hline
 & $\mathrm{S}$ &  & 0.010 \\
\hline
 & Si &  & 0.75 \\
\hline
 & $\mathrm{N}$ &  & 0.10 \\
\hline
Relative density with standard parameters & \multicolumn{3}{|c|}{approx. $100 \%$} \\
\hline
\multirow[t]{2}{*}{Density with standard parameters} & \multicolumn{3}{|c|}{$\mathrm{min} .7 .9 \mathrm{~g} / \mathrm{cm}^{3}$} \\
\hline
 & \multicolumn{3}{|c|}{$\min .0 .285 \mathrm{lb} / \mathrm{in}^{3}$} \\
\hline
\end{tabular}
\end{center}

\section*{Material data sheet}
Mechanical properties of parts (at room temperature)

\begin{center}
\begin{tabular}{lc}
\hline
 & \multicolumn{1}{c}{As built} \\
\hline
Ultimate tensile strength [5] &  \\
\cline { 2 - 2 }
- in horizontal direction (XY) & $640 \pm 50 \mathrm{MPa}$ \\
\cline { 2 - 2 }
- in vertical direction (Z) & $540 \pm 55 \mathrm{MPa}$ \\
\hline
Yield strength, Rp0.2\% [5] &  \\
\cline { 2 - 2 }
- in horizontal direction (XY) & $530 \pm 60 \mathrm{MPa}$ \\
\cline { 2 - 2 }
- in vertical direction (Z) & $470 \pm 90 \mathrm{MPa}$ \\
\hline
Young's modilus [5] &  \\
\cline { 2 - 2 }
- in horizontal direction (XY) & typ. $185 \mathrm{GPa}$ \\
- in vertical direction (Z) & typ. $180 \mathrm{GPa}$ \\
\hline
Elongation at break [5] & $50 \pm 15 \%$ \\
\hline
- in horizontal direction (XY) & typ. 85 HRB \\
\hline
- in vertical direction (Z) &  \\
\hline
Hardness [6] &  \\
\hline
\end{tabular}
\end{center}

[5] Machining and testing of the test bars according to ISO 6892 / ASTM E8M, propotional test pieces, diameter of the neck area $5 \mathrm{~mm}(0.2$ inch), gauge length $4 \mathrm{D}=20.0 \mathrm{~mm}(0.79$ inch), stress rate $10 \mathrm{MPa} / \mathrm{s}$, strain speed in plastic region $0.3751 / \mathrm{min}$.

[6] Rockwell hardness (HRB) measurement according to ENISO 6508-1 on polished surface.

\section*{Material data sheet }
\section*{Abbreviations}
The quoted values refer to the use of these materials with EOSINT M 280 systems according to current specifications (including the latest released process software PSW and any hardware specified for the relevant material) and operating instructions. All values are approimate. Unless otherwise stated, the quoted mechanical and physical properties refer to standard building parameters and test samples built in vertical orientation. They depend on the building parameters and strategies used, which can be varied by the user according to the application.

The data are based on our latest knowledge and are subject to changes without notice. They are provided as an indication and not as a guarantee of suitability for any specific application.

EOS $^{\circledR}$, EOSINT $^{\circledR}$, DMLS $^{\circledR}$, DirectTool ${ }^{\circledR}$ and DirectPart ${ }^{\circledR}$ are registered trademarks of EOS GmbH.

(C) 2014 EOS GmbH - Electro Optical Systems. All rights reserved.

\section*{Material data sheet}
\section*{EOS StainlessSteel GP1 for EOSINT M 270}
A number of different materials are available for use with EOSINT M systems, offering a broad range of e Manufacturing applications. EOStainlessSteelGP1 is a stainless steel powder which has been optimized especially for EOSINTM 270 systems. Other materials are also available for EOSINTM systems, and further materials are continuously being developedplease refer to the relevantmaterial data sheets for details.

This document provides a brief description of the principle applications, and a table of technical data. For details of the system requirements please refer to the relevant information quote.

\section*{Description, application}
EOSStainlessSteelGP1 is a pre alloyed stainless steel in fine powder form. Its composition corresponds to US classification 17-4 and European 1.4542. This kind of steel is characterized by having good corrosion resistance and mechanical properties, especially excellent ductillty in aser processed state, and is widely used in a variety of engineering applications.

This material is ideal for many part-building applications (DirectPart) such as functional metal prototypes, small series products, individualised products or spare parts. Standard processing parameters use full melting of the entire geometry with $20 \mu$ m layer thickness, but it is also possible to useSkin \& Core building style to increase the build speed. Using standard parameters the mechanical properties are fairly uniform in all directions. Parts made from EOSStainlessSteelGP1 can be machined, spark-eroded, welded, micro shot-peened, polished and coated if required. Unexposed powder can be reused.

Typical applications:

\begin{itemize}
  \item engineering applications including functional prototypes, small series products, individualised products or spare parts.
  \item parts requiring high corrosion resistance, sterilisability, etc.
  \item parts requiring particularly high toughness and ductility.
\end{itemize}

\section*{Material data sheet}
\section*{Technical data}
\section*{General processand geometric data}
\begin{center}
\begin{tabular}{|c|c|}
\hline
Minimum recommended layer thickness & \begin{tabular}{l}
$20 \mu \mathrm{m}$ \\
$0.8 \mathrm{mil}$ \\
\end{tabular} \\
\hline
\multicolumn{2}{|l|}{Typical achievabløart accuracy [1]} \\
\hline
- small parts & \begin{tabular}{l}
$\pm 20-50 \mu \mathrm{m}$ \\
$0.8-2.0 \mathrm{mil}$ \\
\end{tabular} \\
\hline
- largeparts [2] & $\pm 0.2 \%$ \\
\hline
Min. wall thickness [3] & \begin{tabular}{c}
$0.3-0.4 \mathrm{~mm}$ \\
$0.012-0.016 \mathrm{in}$ \\
\end{tabular} \\
\hline
\multicolumn{2}{|l|}{Surface roughness} \\
\hline
- after shot peening & \begin{tabular}{l}
Ra $2.5-4.5 \mu \mathrm{m}, \mathrm{Ry} 15-40 \mu \mathrm{m}$ \\
Ra $0.1-0.2$, Ry $0.6-1.6 \mathrm{mil}$ \\
\end{tabular} \\
\hline
- after pdishing & \begin{tabular}{c}
$\mathrm{R}_{\mathrm{z}}$ up to $<0.5 \mu \mathrm{m}$ \\
(can be very finely polished) \\
\end{tabular} \\
\hline
\multicolumn{2}{|l|}{Volume rate [4]} \\
\hline
- standard parameters ( $20 \mu \mathrm{m}$ layers, full density) & \begin{tabular}{c}
$2 \mathrm{~mm}^{3} / \mathrm{s}$ \\
$0.44 \mathrm{in}^{3} / \mathrm{h}$ \\
\end{tabular} \\
\hline
- Inner core parameters (Skin \& Corestyle, full density) & \begin{tabular}{c}
$4 \mathrm{~mm}^{3} / \mathrm{s}$ \\
$0.88-1.1 \mathrm{in}^{3} / \mathrm{h}$ \\
\end{tabular} \\
\hline
\end{tabular}
\end{center}

[1] Based on users' experience of dimensional accuracy for typical geometries, e.g. $\pm 20 \mu \mathrm{m}$ when parameters can be optimized for a certain class of parts or $\pm 50 \mu \mathrm{m}$ when building a new kind of geometry for the first time.

[2] For larger parts the accuracy can be improved by post-process stressrelieving at $650^{\circ} \mathrm{C}$ for 1 hour

[3] Mechanical stability is dependent on geometry (wall height etc.) and application

[4] Volume rate is a measure of build speed during laser exposure. The total build speed depends on the average volume rate, the recoating time (related to the number of layers) and other factors such as DMLS - Start settings.

\section*{Material data sheet}
Physical and chemical properties of parts

\begin{center}
\includegraphics[max width=\textwidth]{2024_03_11_add5292786349c475313g-038}
\end{center}

\section*{Material data sheet}
\begin{center}
\begin{tabular}{lcc}
\hline
Young's modulus & \begin{tabular}{c}
$170 \pm 30 \mathrm{GPa}$ \\
$(25 \pm 4 \mathrm{msi})$ \\
\end{tabular} & \begin{tabular}{c}
typical 180 GPa \\
$(26 \mathrm{msi})$ \\
\end{tabular} \\
\hline
Elongation at break &  &  \\
- in horizontal direction (XY) & \begin{tabular}{c}
$\min 25 \%$ \\
typical $31 \pm 5 \%$ \\
\end{tabular} & typical 29\% \\
\cline { 2 - 3 }
- in vertical direction (Z) & \begin{tabular}{c}
$\min 25 \%$ \\
typical 35 $\pm 5 \%$ \\
\end{tabular} & typical 31\% \\
\hline
Hardness [6] &  &  \\
- as built & approx. 230 $20 \mathrm{HV} 1$ &  \\
- ground \& polished7] & approx 250 - 400 HV1 &  \\
\hline
\end{tabular}
\end{center}

[5] Mechanical testing according to ISO 6892:1998(E) Annex C, proportional test pieces, Diameter of the neck area $5 \mathrm{~mm}$ original gauge length $25 \mathrm{~mm}$

[6] Vickers hardness measurement (HV) according to DINEN ISO 6507-1. Note that depending on the measurement method used, the measured hardness value can be dependent on the surface roughness and can be lower than the real hardness. To avoid inaccurate results, hardness should be measured on a polished surface.

[7] Due to work-hardening effect

Thermal properties of parts

\begin{center}
\begin{tabular}{lc}
\hline
Coefficient of thermal expansion &  \\
\cline { 2 - 2 }
- over $20-600^{\circ} \mathrm{C}\left(68-1080^{\circ} \mathrm{F}\right)$ & $14 \times 10^{-6} \mathrm{~m} / \mathrm{m}^{\circ} \mathrm{C}$ \\
 & $7.8 \times 10^{-6} \mathrm{in} / \mathrm{in}{ }^{\circ} \mathrm{F}$ \\
\hline
Thermal conductivity & $13 \mathrm{~W} / \mathrm{m}^{\circ} \mathrm{C}$ \\
- at $20^{\circ} \mathrm{C}\left(68^{\circ} \mathrm{F}\right)$ & $90 \mathrm{Btu} /\left(\mathrm{h} \mathrm{ft}^{\circ}{ }^{\circ} \mathrm{F} / \mathrm{in}\right)$ \\
\hline
- at $100^{\circ} \mathrm{C}\left(212^{\circ} \mathrm{F}\right)$ & $14 \mathrm{~W} / \mathrm{m}^{\circ} \mathrm{C}$ \\
 & $97 \mathrm{Btu} /\left(\mathrm{h} \mathrm{ft}{ }^{\circ} \mathrm{F} / \mathrm{in}\right)$ \\
- at $200^{\circ} \mathrm{C}\left(392^{\circ} \mathrm{F}\right)$ & $15 \mathrm{~W} / \mathrm{m}^{\circ} \mathrm{C}$ \\
 & $104 \mathrm{Btu} /\left(\mathrm{h} \mathrm{ft}{ }^{\circ} \mathrm{F} / \mathrm{in}\right)$ \\
\hline
at $300^{\circ} \mathrm{C}\left(572^{\circ} \mathrm{F}\right)$ & $16 \mathrm{~W} / \mathrm{m}^{\circ} \mathrm{C}$ \\
\hline
Maximum operating temperature & $111 \mathrm{Btu} /\left(\mathrm{h} \mathrm{ft}{ }^{\circ} \mathrm{F} / \mathrm{in}\right)$ \\
\hline
\end{tabular}
\end{center}

\section*{Material data sheet}
The quoted values refer to the use of these materials with EOSINT M 270 systems according to current specifications (including the latest released process software PSW and any hardware specified for the relevant material) and operating instructions. All values are approximate. Unless otherwise stated, the quoted mechanical and physical properties refer to standard building parameters and test samples built in horizontal orientation. They depend on the building parameters and strategies used, which can be varied by the user according to the application. Measurements of the same properties using different test methods (e.g. specimen geometries) can give different results. The data are based on our latest knowledge and are subject to changes without notice. They are provided as an indication and not as a guarantee of suitability for any specific application.

EOS $^{\circledR}$, EOSINT $^{\circledR}$, DMLS $^{\circledR}$ and DirectPart ${ }^{\circledR}$ are registered trademarks of EOS GmbH.

(C) 2009 EOS GmbH- Electro OpticalSystems. All rights reserved.

\section*{Material data sheet}
\section*{EOS Titanium Ti64}
EOS Titanium Ti64 is a titanium alloy powder which has been optimized especially for processing on EOSINT M systems.

This document provides information and data for parts built using EOS Titanium Ti64 powder (EOS art.-no. 9011-0014) on the following system specifications:

\begin{itemize}
  \item EOSINT M 270 Installation ModeXtended with PSW 3.4 and default job Ti64\_30\_030\_default.job
  \item EOSINT M 270 Dual Mode with PSW 3.5 and Original EOS Parameter Set Ti64\_Performance 2.0
  \item EOSINT M 280
\end{itemize}

with PSW 3.5 and Original EOS Parameter Set Ti64\_Speed 1.0

\section*{Description}
Parts built in EOS Titanium Ti64 have a chemical composition corresponding to ISO 5832-3, ASTM F1472 and ASTM B348.

This well-known light alloy is characterized by having excellent mechanical properties and corrosion resistance combined with low specific weight and biocompatibility.

This material is ideal for many high-performance engineering applications, for example in aerospace and motor racing, and also for the production of biomedical implants (note: subject to fulfilment of statutory validation requirements where appropriate).

Due to the layerwise building method, the parts have a certain anisotropy, which can be reduced or removed by appropriate heat treatment - see Technical Data for examples.

\section*{Material data sheet}
\section*{Technical data}
\section*{General process and geometric data}
\begin{center}
\begin{tabular}{|c|c|}
\hline
Typical achievable part accuracy [1] & $\pm 50 \mu \mathrm{m}$ \\
\hline
Min. wall thickness [2] & \begin{tabular}{c}
approx. $0.3-0.4 \mathrm{~mm}$ \\
approx. $0.012-0.016$ inch \\
\end{tabular} \\
\hline
\multicolumn{2}{|l|}{Surface roughness, as built [3]} \\
\hline
\begin{tabular}{l}
Ti64\_30\_030\_default.job \\
Ti64 Performance $(30 \mu \mathrm{m})$ \\
\end{tabular} & \begin{tabular}{l}
Ra $9-12 \mu \mathrm{m}, \mathrm{R}_{z} 40-80 \mu \mathrm{m}$ \\
Ra $0.36-0.47 \times 10^{\circ} \mathrm{inch}$ \\
$\quad \mathrm{Rz}_{\mathrm{z}} 1.6-3.2 \times 10^{3} \mathrm{inch}$ \\
\end{tabular} \\
\hline
Ti64 Speed 1.0 (60 $\mu \mathrm{m})$ & \begin{tabular}{l}
$R_{a} 6-10 \mu \mathrm{m}, R_{z} 35-40 \mu \mathrm{m}$ \\
$\mathrm{R}_{\mathrm{a}} 0.23-0.39 \times 10^{\circ} \mathrm{inch}$, \\
Rz $1.37-1.57 \times 10 \mathrm{inch}$ \\
\end{tabular} \\
\hline
\multicolumn{2}{|l|}{Volume rate [4]} \\
\hline
\begin{tabular}{l}
Ti64\_30\_030\_default.job \\
Ti64 Performance $(30 \mu \mathrm{m})$ \\
\end{tabular} & \begin{tabular}{c}
$3.75 \mathrm{~mm}^{3} / \mathrm{s}\left(13.5 \mathrm{~cm}^{3} / \mathrm{h}\right)$ \\
$0.82 \mathrm{in}^{3} / \mathrm{h}$ \\
\end{tabular} \\
\hline
\includegraphics[max width=\textwidth]{2024_03_11_add5292786349c475313g-042}
 & \begin{tabular}{c}
$9 \mathrm{~mm}^{3} / \mathrm{s}\left(32.4 \mathrm{~cm}^{3} / \mathrm{h}\right)$ \\
$1.98 \mathrm{in}^{3} / \mathrm{h}$ \\
\end{tabular} \\
\hline
\end{tabular}
\end{center}

[1] Based on users' experience of dimensional accuracy for typical geometries. Part accuracy is subject to appropriate data preparation and post-processing, in accordance with EOS training.

[2] Mechanical stability is dependent on geometry (wall height etc.) and application

[3] Due to the layerwise building, the surface structure depends strongly on the orientation of the surface, for example sloping and curved surfaces exhibit a stair-step effect. The values also depend on the measurement method used. The values quoted here given an indication of what can be expected for horizontal (up-facing) or vertical surfaces.

[4] Volume rate is a measure of build speed during laser exposure. The total build speed depends on the average volume rate, the recoating time (related to the number of layers) and other factors such as DMLS-Start settings.

\section*{Material data sheet}
\section*{Physical and chemical properties of parts}
\begin{center}
\begin{tabular}{lc}
\hline
Material composition & Ti (balance) \\
 & $\mathrm{Al}(5.5-6.75 \mathrm{wt} .-\%)$ \\
$\mathrm{V}(3.5-4.5 \mathrm{wt} .-\%)$ &  \\
 & $\mathrm{O}(<2000 \mathrm{ppm})$ \\
$\mathrm{N}(<500 \mathrm{ppm})$ &  \\
$\mathrm{C}(<800 \mathrm{ppm})$ &  \\
$\mathrm{H}(<150 \mathrm{ppm})$ &  \\
Relative density & $\mathrm{Fe}(<3000 \mathrm{ppm})$ \\
\hline
Density & $\mathrm{approx.} 100 \%$ \\
 & $4.41 \mathrm{~g} / \mathrm{cm}^{3}$ \\
\hline
\end{tabular}
\end{center}

\section*{Material data sheet}
\section*{Mechanical properties of parts}
\begin{center}
\begin{tabular}{|c|c|c|}
\hline
 & As built & Heat treated [6] \\
\hline
\multicolumn{3}{|l|}{Tensile strength [5]} \\
\hline
- in horizontal direction (XY) & \begin{tabular}{l}
typ. $1230 \pm 50 \mathrm{MPa}$ \\
typ. $178 \pm 7 \mathrm{ksi}$ \\
\end{tabular} & \begin{tabular}{c}
$\min .930 \mathrm{MPa}(134.8 \mathrm{ksi})$ \\
typ. $1050 \pm 20 \mathrm{MPa}(152 \pm 3 \mathrm{ksi})$ \\
\end{tabular} \\
\hline
- in vertical direction (Z) & \begin{tabular}{l}
typ. $1200 \pm 50 \mathrm{MPa}$ \\
typ. $174 \pm 7 \mathrm{ksi}$ \\
\end{tabular} & \begin{tabular}{c}
$\min .930 \mathrm{MPa}(134.8 \mathrm{ksi})$ \\
typ. $1060 \pm 20 \mathrm{MPa}(154 \pm 3 \mathrm{ksi})$ \\
\end{tabular} \\
\hline
\multicolumn{3}{|l|}{Yield strength (Rp0.2) [5]} \\
\hline
- in horizontal direction (XY) & \begin{tabular}{l}
typ. $1060 \pm 50 \mathrm{MPa}$ \\
typ. $154 \pm 7 \mathrm{ksi}$ \\
\end{tabular} & \begin{tabular}{c}
$\min .860 \mathrm{MPa}(124.7 \mathrm{ksi})$ \\
typ. $1000 \pm 20 \mathrm{MPa}(145 \pm 3 \mathrm{ksi})$ \\
\end{tabular} \\
\hline
- in vertical direction (Z) & \begin{tabular}{l}
typ. $1070 \pm 50 \mathrm{MPa}$ \\
typ. $155 \pm 7 \mathrm{ksi}$ \\
\end{tabular} & \begin{tabular}{c}
$\min .860 \mathrm{MPa}(124.7 \mathrm{ksi})$ \\
typ. $1000 \pm 20 \mathrm{MPa}(145 \pm 3 \mathrm{ksi})$ \\
\end{tabular} \\
\hline
\multicolumn{3}{|l|}{Elongation at break [5]} \\
\hline
- in horizontal direction (XY) & typ. $(10 \pm 2) \%$ & \begin{tabular}{r}
$\min .10 \%$ \\
typ. $(14 \pm 1 \%)$ \\
\end{tabular} \\
\hline
- in vertical direction (Z) & typ. $(11 \pm 3) \%$ & \begin{tabular}{r}
$\min .10 \%$ \\
typ. $(15 \pm 1 \%)$ \\
\end{tabular} \\
\hline
\multicolumn{3}{|l|}{Modulus of elasticity [5]} \\
\hline
- in horizontal direction (XY) & \begin{tabular}{l}
typ. $110 \pm 10 \mathrm{GPa}$ \\
typ. $16 \pm 1.5 \mathrm{Msi}$ \\
\end{tabular} & \begin{tabular}{l}
typ. $116 \pm 10 \mathrm{GPa}$ \\
typ. $17 \pm 1.5 \mathrm{Msi}$ \\
\end{tabular} \\
\hline
- in vertical direction (Z) & \begin{tabular}{l}
typ. $110 \pm 10 \mathrm{GPa}$ \\
typ. $16 \pm 1.5 \mathrm{Msi}$ \\
\end{tabular} & \begin{tabular}{l}
typ. $114 \pm 10 \mathrm{GPa}$ \\
typ. $17 \pm 1.5 \mathrm{Msi}$ \\
\end{tabular} \\
\hline
Hardness [7] & typ. $320 \pm 12 \mathrm{HV} 5$ &  \\
\hline
\end{tabular}
\end{center}

[5] Tensile testing according to ISO 6892-1:2009 (B) Annex D, proportional test pieces, diameter of the neck area $5 \mathrm{~mm}$ ( 0.2 inch), original gauge length $25 \mathrm{~mm}$ (1 inch).

[6] Specimens were treated at $800^{\circ} \mathrm{C}\left(1470^{\circ} \mathrm{F}\right)$ for 4 hours in argon inert atmosphere. Mechanical properties are expressed as minimum values to indicate that mechanical properties exceed the minimum requirements of material specification standards. ASTM F1472-08. By fulfilling these minimum values, also the specifications of standards ASTM B348-09 and ISO 5832-3:2000 are meet.

[7] Vickers hardness measurement (HV) according to EN ISO 6507-1 on polished surface. Note that measured hardness can vary significantly depending on how the specimen has been prepared.

\section*{Material data sheet}
\section*{Thermal properties of parts}
\begin{center}
\begin{tabular}{ll}
\hline
Maximum long-term operating temperature & approx. $350^{\circ} \mathrm{C}$ \\
 & approx. $660^{\circ} \mathrm{F}$ \\
\end{tabular}
\end{center}

\section*{Abbreviations}
typ. typical

min. minimum

wt. weight

approx. approximately

\section*{Notes}
The data are valid for the combinations of powder material machine and parameter sets referred to on page 1 , when used in accordance with the relevant Operating Instructions (including Installation Requirements and Maintenance) and Parameter Sheet. Part properties are measured using defined test procedures. Further details of the test procedures used by EOS are available on request.

The data correspond to our knowledge and experience at the time of publication. They do not on their own provide a sufficient basis for designing parts. Neither do they provide any agreement or guarantee about the specific properties of a part or the suitability of a part for a specific application. The producer or the purchaser of a part is responsible for checking the properties and the suitability of a part for a particular application. This also applies regarding any rights of protection as well as laws and regulations. The data are subject to change without notice as part of EOS' continuous development and improvement processes.

EOS, EOSINTand DMLS are registered trademarks of EOS GmbH.

2011 EOS GmbH - Electro Optical Systems. All rights reserved.

\section*{EOS NickelAlloy HX}
EOS NickelAlloy HX is a heat and corrosion resistant metal alloy powder intended for processing on EOS M 290 systems.

This document provides information and data for parts built using EOS NickelAlloy HX powder (EOS art.-no. 9011-0023) on the following system specifications:

\begin{itemize}
  \item EOS M 290 400W with EOSPRINT 1.x and EOS Parameter set HX\_Performance 2.0
\end{itemize}

\section*{Description, application}
EOS NickelAlloy HX raw material is a nickel-chromium-iron-molybdenum alloy in fine powder form. Its composition corresponds to UNS N06002. While the wrought and cast versions of the alloy generally are solution annealed, the laser melted material manufactured of this powder has a high strength and good elongation already in the as- built condition. Solution annealing of the laser sintered material will homogenize the microstructure, relax internal stresses and increase the elongation, while slightly decreasing the strength.

This type of alloy is characterized by having high strength and oxidation resistance also at elevated temperatures, and is often used up to $1200^{\circ} \mathrm{C}$. Therefore its applications can be found in aerospace technology, gas turbine parts, etc.

Standard laser processing parameters results in full melting of the entire geometry, typically with $40 \mu \mathrm{m}$ layer thickness. Parts built from EOS NickelAlloy HX can be heat treated and material properties can be varied within specified range. In both as-built and solution heat treated states the parts can be machined, spark-eroded, welded, micro shot-peened, polished, and coated if required. Unexposed powder can be reused.

\section*{EOS - Material data sheet}
\section*{Technical data}
General process data

\begin{center}
\begin{tabular}{|c|c|}
\hline
 & EOS NickelAlloy HX \\
\hline
\multicolumn{2}{|l|}{Typical achievable part accuracy [1]} \\
\hline
- small parts & \begin{tabular}{c}
approx. $\pm 50-80 \mu \mathrm{m}$ \\
$( \pm 0.0020-0.0031$ inch $)$ \\
\end{tabular} \\
\hline
- large parts & approx. $\pm 0.2 \%$ \\
\hline
Min. wall thickness [2] & \begin{tabular}{c}
typ. $0.4-0.5 \mathrm{~mm}$ \\
$(0.016-0.020$ inch) \\
\end{tabular} \\
\hline
Layer thickness & $40 \mu \mathrm{m}$ \\
\hline
\multicolumn{2}{|l|}{Surface roughness [3]} \\
\hline
- after shot-peening (horizontal / vertical) & \begin{tabular}{l}
$R_{a} 3-6.5 \mu \mathrm{m}, \mathrm{Rz}_{2} 10-30 \mu \mathrm{m}$ \\
$\mathrm{Ra}_{\mathrm{a}} 0.12-0.25 \times 10^{-3}$ inch, \\
$\mathrm{R}_{z} 0.39-1.18 \times 10^{-3}$ inch \\
\end{tabular} \\
\hline
- after polishing & \begin{tabular}{l}
$\quad R_{z}$ up to $<0.5 \mu \mathrm{m}$ \\
$\mathrm{R}_{z}$ up to $<0.02 \times 10^{3}$ inch \\
(can be very finely polished) \\
\end{tabular} \\
\hline
Volume rate [4] & \begin{tabular}{c}
$4.2 \mathrm{~mm}^{3} / \mathrm{s}\left(15.2 \mathrm{~cm}^{3} / \mathrm{h}\right)$ \\
$0.93 \mathrm{in}^{3} / \mathrm{h}$ \\
\end{tabular} \\
\hline
\end{tabular}
\end{center}

[1] Based on users' experience of dimensional accuracy for typical geometries, e.g. $\pm 50 \mu \mathrm{m}$ when parameters can be optimized for a certain class of parts or $\pm 80 \mu \mathrm{m}$ when building a new kind of geometry for the first time. Part accuracy is subject to appropriate data preparation and postprocessing.

[2] Mechanical stability is dependent on geometry (wall height etc.) and application

[3] Due to the layerwise building, the surface structure depends strongly on the orientation of the surface, for example sloping and curved surfaces exhibit a stair-step effect. The values also depend on the measurement method used. The values quoted here given an indication of what can be expected for horizontal (up-facing) or vertical surfaces.

[4] Volume rate is a measure of build speed during laser exposure. The total build speed depends on the average volume rate, the recoating time (related to the number of layers) and other factors such as DMLS- Start settings.

\section*{EOS - Material data sheet}
Physical and chemical properties of parts

\begin{center}
\begin{tabular}{lccc}
\hline
 & \multicolumn{3}{c}{EOS NickelAlloy HX} \\
\hline
Material composition & Element & Min & Max \\
\cline { 2 - 5 }
 & $\mathrm{Ni}$ & \multicolumn{2}{c}{balance} \\
 & $\mathrm{Cr}$ & 20.5 & 23.0 \\
 & $\mathrm{Fe}$ & 17.0 & 20.0 \\
 & $\mathrm{Mo}$ & 8.0 & 10.0 \\
 & $\mathrm{~W}$ & 0.2 & 1.0 \\
 & $\mathrm{Co}$ & 0.5 & 2.5 \\
 & $\mathrm{C}$ & -- & 0.1 \\
 & $\mathrm{Si}$ & -- & 1.0 \\
 & $\mathrm{Mn}$ & -- & 1.0 \\
 & $\mathrm{~S}$ & -- & 0.03 \\
 & $\mathrm{P}$ & -- & 0.04 \\
 & $\mathrm{~B}$ & -- & 0.01 \\
 & $\mathrm{Se}$ & -- & 0.0050 \\
 & $\mathrm{Cu}$ & -- & 0.5 \\
 & $\mathrm{Al}$ & -- & 0.5 \\
\hline
Relative density with standard parameters & Ti & -- & 0.15 \\
\hline
\end{tabular}
\end{center}

\section*{EOS - Material data sheet}
Mechanical properties of parts (at room temperature)

\begin{center}
\begin{tabular}{lc}
\hline
 & \multicolumn{1}{c}{As built} \\
\hline
Ultimate tensile strength [5] &  \\
\cline { 2 - 2 }
- in horizontal direction (XY) & typ. $820 \pm 50 \mathrm{MPa}$ \\
\cline { 2 - 2 }
- in vertical direction (Z) & typ. $675 \pm 50 \mathrm{MPa}$ \\
\hline
Yield strength, Rp0.2\% [5] &  \\
\cline { 2 - 2 }
- in horizontal direction (XY) & typ. $630 \pm 50 \mathrm{MPa}$ \\
\cline { 2 - 2 }
- in vertical direction (Z) & typ. $545 \pm 50 \mathrm{MPa}$ \\
\hline
Young's modulus [5] &  \\
\cline { 2 - 2 }
- in horizontal direction (XY) & typ. $195 \pm 20 \mathrm{GPa}$ \\
\hline
- in vertical direction (Z) & typ. $175 \pm 20 \mathrm{GPa}$ \\
\hline
Elongation at break [5] &  \\
\hline
\end{tabular}
\end{center}

[5] Tensile testing according to ISO 6892-1:2009 (B) Annex D, proportional test pieces, diameter of the neck area $5 \mathrm{~mm}$ ( 0.2 inch), original gauge length $25 \mathrm{~mm}$ (1 inch).

\section*{Abbreviations}
typ. typical

min. minimum

approx. approximately

wt weight

\section*{EOS - Material data sheet}
The quoted values refer to the use of these materials with EOS M 290 systems according to current specifications (including the latest released process software PSW and any hardware specified for the relevant material) and operating instructions. All values are approximate. Unless otherwise stated, the quoted mechanical and physical properties refer to standard building parameters and test samples built in vertical orientation. They depend on the building parameters and strategies used, which can be varied by the user according to the application.

The data are based on our latest knowledge and are subject to changes without notice. They are provided as an indication and not as a guarantee of suitability for any specific application.

EOS ${ }^{\square}$ EOSINT $T^{\square}$, DM LS $^{\square}$, DirectTool ${ }^{\square}$ and DirectPart ${ }^{\square}$ are registered trademarks of EOS GmbH.

प 2015 EOS GmbH - Electro Optical Systems. All rights reserved.

\section*{Material data sheet}
\section*{EOS NickelAlloy IN625}
EOS NickelAlloy IN625 is a heat and corrosion resistant nickel alloy powder which has been optimized especially for processing on EOSINT M systems.

This document provides information and data for parts built using EOS NickelAlloy IN625 powder (EOS art.-no. 9011-0022) on the following system specifications:

\begin{itemize}
  \item EOS DMLS ${ }^{\text {TM }}$ system: EOS M290
  \item HSS recoating blade (2200-4073)
  \item Argon atmosphere
  \item Grid nozzle (2200-5501)
  \item IPCM M sieving module with 63 4 m mesh recommended (9044-0032)
  \item Software: EOSYSTEM v.2.5 or newer
  \item EOS Parameter set IN625 Performance 2.0
\end{itemize}

\section*{Description}
Parts built from EOS NickelAlloy IN625 have chemical composition corresponding to UNS N06625, AMS 5666F, AMS 5599G, \href{http://W.Nr}{W.Nr} 2.4856, DIN NiCr22Mo9Nb. This type of alloy is characterized by having high tensile, creep and rupture strength. Conventionally cast or wrought components in this type of nickel alloy have typically excellent fatigue and thermal-fatigue properties combined with good oxidation resistance. EOS NickelAlloy IN625 is expected to have good corrosion resistance in various corrosive environments. Especially sea-water applications require high pitting and crevice corrosion resistance, stress-corrosion resistance against chloride-ions, high tensile and corrosion-fatigue strength. However, corrosion resistance has not been verified yet and therefore it is recommended to conduct relevant corrosion tests and studies prior to use in specific corrosive environment.

Parts built from EOS NickelAlloy IN625 can be heat treated and material properties can be varied within specified range. Parts can be machined, spark-eroded, welded, micro shot-peened, polished and coated in both as-built and in heat-treated conditions. Due to the layerwise building method, the parts have certain anisotropy.

\section*{Material data sheet}
\section*{Quality Assurance}
The quality of the EOS NickelAlloy IN625 powder lots is ensured by the Quality Assurance procedures. The procedures include sampling (ASTM B215), PSD analysis (ISO 13320), chemical analyses (ASTM E2371, ASTM E1409, ASTM E1941, ASTM E1447), and mechanical testing (IS0 6892-1).

The results of the quality assurance tests are given in the lot specific Mill Test Certificates (MTC) according to EN 10204 type 3.1.

\section*{Material data sheet}
\section*{Technical Data}
\section*{Powder properties}
\begin{center}
\begin{tabular}{|c|c|c|c|}
\hline
Material composition [wt.\%] & Element & Min & Max \\
\hline
 & $\mathrm{Cr}$ & 20.00 & 23.00 \\
\hline
 & Mo & 8.00 & 10.00 \\
\hline
 & $\mathrm{Nb}$ & 3.15 & 4.15 \\
\hline
 & $\mathrm{Fe}$ & - & 5.00 \\
\hline
 & $\mathrm{Ti}$ & - & 0.40 \\
\hline
 & $\mathrm{Al}$ & - & 0.40 \\
\hline
 & Co & - & 1.00 \\
\hline
 & Si & - & 0.50 \\
\hline
 & $\mathrm{Mn}$ & - & 0.50 \\
\hline
 & C & - & 0.10 \\
\hline
 & Ta & - & 0.05 \\
\hline
 & P & - & 0.015 \\
\hline
 & S & - & 0.015 \\
\hline
 & $\mathrm{Ni}$ & 58.00 & bal. \\
\hline
\end{tabular}
\end{center}

\section*{Particle size}
d50 [1]

$35 \pm 6 \mu \mathrm{m}$

[1] Particle size distribution analysis according to ISO 13320

\section*{Material data sheet}
\section*{General process data}
\begin{center}
\begin{tabular}{lc}
\hline
Layer thickness & $40 \mu \mathrm{m}$ \\
\hline
Volume rate [2] & $4.2 \mathrm{~mm}^{3} / \mathrm{s}\left(15.2 \mathrm{~cm}^{3} / \mathrm{h}\right)$ \\
\hline
\end{tabular}
\end{center}

[2] The volume rate is a measure of build speed during laser exposure of the skin area. The total build speed depends on this volume rate and many other factors such as exposure parameters of contours, supports, up and downskin, recoating time, Home-In or LPM settings.

\section*{Physical properties of parts}
\begin{center}
\begin{tabular}{lc}
\hline
Part density [3] & $8.4 \mathrm{~g} / \mathrm{cm}^{3}$ \\
\hline
Surface roughness after shot peening [4] & typ. Ra $1-5 \mu \mathrm{m} ;$ Rz $3-10 \mu \mathrm{m}$ \\
\hline
Hardness as built [5] & typ. $27 \mathrm{HRC}$ \\
\hline
\end{tabular}
\end{center}

[3] Weighing in air and water according to ISO 3369.

[4] The numbers were measured at the horizontal (up-facing) and all vertical surfaces of test cubes. Due to the layerwise building the roughness strongly depends on the orientation of the surface, for example sloping and curved surfaces exhibit a stair-step effect.

[5] Hardness measurement according to standard EN ISO 6508-1:2005

\section*{Material data sheet}
\section*{Tensile data at room temperature $[6,7]$}
\begin{center}
\begin{tabular}{lcc}
\hline
 & As built [9] & Heat treated [8,9] \\
\hline
Ultimate tensile strength, Rm &  &  \\
\hline
- in horizontal direction (XY) & Mean $980 \mathrm{MPa}$ & Mean $1000 \mathrm{MPa}$ \\
 & StDev. $5 \mathrm{MPa}$ & StDev. $10 \mathrm{MPa}$ \\
\hline
- in vertical direction (Z) & Mean 870 MPa & StDev. $10 \mathrm{MPa}$ \\
\hline
Yield strength, Rp0.2\% & StDev. $10 \mathrm{MPa}$ &  \\
\hline
- in horizontal direction (XY) &  & Mean 680 MPa \\
 & Mean 720 MPa & StDev. $5 \mathrm{MPa}$ \\
\hline
- in vertical direction (Z) & StDev. $5 \mathrm{MPa}$ & Mean 640 MPa \\
\hline
Elongation at break, A & Mean 630 MPa & StDev. $\mathrm{MPa}$ \\
\hline
- in horizontal direction (XY) & StDev. $5 \mathrm{MPa}$ & Mean 34\% \\
 &  & StDev. 2\% \\
\hline
- in vertical direction (Z) & Mean 33 \% & Mean 49 \% \\
\hline
\end{tabular}
\end{center}

[6] The numbers are average values and are determined from samples with horizontal and vertical orientation.

[7] Tensile testing according to ISO 6892-1 B10, proportional test pieces, diameter of the neck area $5 \mathrm{~mm}(0.2$ inch), original gauge length $20 \mathrm{~mm}(0,79$ inch).

[8] Heat treatment procedure: anneal at $870^{\circ} \mathrm{C}\left(1600^{\circ} \mathrm{F}\right)$ for 1 hour, rapid cooling.

[9] The values are subject to variations depending on samples orientation on a building platform.

\section*{Material data sheet}
\section*{Abbreviations}
min. minimum

max. maximum

wt. weight

typ. typical

StDev. standard deviation

The quoted values refer to the use of this material with above specified EOS DMLS system, EOSYSTEM software version, parameter set and operation in compliance with parameter sheet and operating instructions. All measured values are average numbers. Part properties are measured with specified measurement methods using defined test geometries and procedures and. Further details of the test procedures used by EOS are available on request. Any deviation from these standard settings may affect the measured properties.

The data correspond to EOS knowledge and experience at the time of publication and they are subject to change without notice as part of EOS' continuous development and improvement processes.

EOS does not warrant any properties or fitness for a specific purpose, unless explicitly agreed upon. This also applies regarding any rights of protection as well as laws and regulations.

EOS $^{\circledR}$, EOSINT $^{\circledR}$, DMLS $^{\circledR}$, DirectTool $^{\circledR}$ and DirectPart ${ }^{\circledR}$ are registered trademarks of EOS $\mathrm{GmbH}$.

(C2017 EOS GmbH - Electro Optical Systems. All rights reserved.

\section*{Typical Material Properties}
\begin{center}
\begin{tabular}{|c|c|c|}
\hline
Material Properties & Test Method & 17-4PH H900 \\
\hline
\multicolumn{3}{|l|}{Tensile Strength} \\
\hline
Ultimate Strength & \multirow{4}{*}{ASTM E8} & \begin{tabular}{l}
X \& Y: 1070 - $1310 \mathrm{MPa}$ \\
Z: $1070-1310 \mathrm{MPa}$ \\
\end{tabular} \\
\hline
Yield Strength ( $0.2 \%$ offset) &  & \begin{tabular}{l}
X\& Y: $970-1030 \mathrm{MPa}$ \\
Z: $970-1020 \mathrm{MPa}$ \\
\end{tabular} \\
\hline
Elongation &  & \begin{tabular}{l}
$X \& Y: 4 \%-12 \%$ \\
Z: $4 \%-11 \%$ \\
\end{tabular} \\
\hline
Elastic Modulus &  & \begin{tabular}{l}
X \& Y: $180-190 \mathrm{GPa}$ \\
$\mathrm{Z}: 180-200 \mathrm{GPa}$ \\
\end{tabular} \\
\hline
Hardness & ASTM E18 & $35-41 \mathrm{HRC}$ \\
\hline
Impact & ASTM E23 & $55-75$ \\
\hline
Poisson's Ratio &  & $0.28-0.30$ \\
\hline
Relative Density &  & $96-99 \%$ \\
\hline
Density &  & $7.5-7.7 \mathrm{~g} / \mathrm{cc}$ \\
\hline
Surface Roughness &  & $3-12 \mu \mathrm{m} \mathrm{Ra}$ \\
\hline
\end{tabular}
\end{center}

\begin{center}
\includegraphics[max width=\textwidth]{2024_03_11_add5292786349c475313g-057}
\end{center}

17-4PH Printed Part

\begin{center}
\begin{tabular}{|l|l|l|l|}
\hline
Material Composition &  &  &  \\
\hline
Iron & bal & Niobium + Tantalum & $0.15-0.45 \%$ \\
\hline
Nickel & $3-5 \%$ & Manganese & $1.0 \%$ max \\
\hline
Chromium & $15.5-17.5 \%$ & Silicon & $1.0 \%$ max \\
\hline
Carbon & $0.07 \%$ max &  &  \\
\hline
Copper & $3-5 \%$ &  &  \\
\hline
\end{tabular}
\end{center}

\begin{center}
\includegraphics[max width=\textwidth]{2024_03_11_add5292786349c475313g-057(1)}
\end{center}

ExOne disclaims all warranties and liabilities for the content hereof and makes no representations as to its accuracy or fitness for use for any purpose. Any tradenames, trademarks, or service marks of others appearing herein are used strictly nominatively and are not to be construed as implying any affiliation, connection, association, sponsorship, or approval of the owners thereof for ExOne, its products, or the content hereof.

\section*{Typical Material Properties}
\begin{center}
\begin{tabular}{|c|c|c|}
\hline
Material Properties & Test Method & $316 \mathrm{~L}$ \\
\hline
\multicolumn{3}{|l|}{Tensile Strength} \\
\hline
Ultimate Strength & \multirow{4}{*}{ASTM E8} & \begin{tabular}{l}
$X \& Y: 450-580 \mathrm{MPa}$ \\
Z: $450-520 \mathrm{MPa}$ \\
\end{tabular} \\
\hline
Yield Strength ( $0.2 \%$ offset) &  & \begin{tabular}{l}
X \& Y: $140-220 \mathrm{MPa}$ \\
Z: $140-220 \mathrm{MPa}$ \\
\end{tabular} \\
\hline
Elongation &  & \begin{tabular}{l}
$X \& Y: 40-55 \%$ \\
Z: $40-50 \%$ \\
\end{tabular} \\
\hline
Elastic Modulus &  & \begin{tabular}{l}
X\& Y: $190-220 \mathrm{GPa}$ \\
Z: $180-190 \mathrm{GPa}$ \\
\end{tabular} \\
\hline
Hardness & ASTM E18 & $67-71$ HRB \\
\hline
Impact & ASTM E23 & $55-75 J$ \\
\hline
Poisson's Ratio &  & $0.28-0.30$ \\
\hline
Relative Density &  & $96-99 \%$ \\
\hline
Density &  & $7.6-7.9 \mathrm{~g} / \mathrm{cc}$ \\
\hline
Surface Roughness &  & $3-12 \mu \mathrm{m} \mathrm{Ra}$ \\
\hline
\end{tabular}
\end{center}

\begin{center}
\includegraphics[max width=\textwidth]{2024_03_11_add5292786349c475313g-058(1)}
\end{center}

316L Printed Part

\begin{center}
\includegraphics[max width=\textwidth]{2024_03_11_add5292786349c475313g-058}
\end{center}

ExOne disclaims all warranties and liabilities for the content hereof and makes no representations as to its accuracy or fitness for use for any purpose. Any tradenames, trademarks, or service marks of others appearing herein are used strictly nominatively and are not to be construed as implying any affiliation, connection, association, sponsorship, or approval of the owners thereof for ExOne, its products, or the content hereof.

\begin{center}
\begin{tabular}{|c|c|}
\hline
\multicolumn{2}{|c|}{}\begin{tabular}{l}
Material Properties \\
Table displays the average tested values \\
\end{tabular} \\
\hline
\multicolumn{2}{|l|}{Tensile Data} \\
\hline
Ultimate Strength & \begin{tabular}{l}
X \& Y: $1310 \mathrm{MPa}$ \\
Z: $1331 \mathrm{MPa}$ \\
\end{tabular} \\
\hline
Yield Strength & \begin{tabular}{l}
$X \& Y: 1069 \mathrm{MPa}$ \\
$Z: 1076 \mathrm{MPa}$ \\
\end{tabular} \\
\hline
Elongation & \begin{tabular}{l}
X \& Y: $15 \%$ \\
Z: $15 \%$ \\
\end{tabular} \\
\hline
Elastic Modulus & \begin{tabular}{l}
$X \& Y: 193 \mathrm{GPa}$ \\
Z: $193 \mathrm{GPa}$ \\
\end{tabular} \\
\hline
Hardness & $41 \mathrm{HRC}$ \\
\hline
Impact & $12.2 \mathrm{~J}$ \\
\hline
Poisson's Ratio & 0.29 \\
\hline
Relative Density & $99.2 \%$ \\
\hline
Density & $8.13 \mathrm{~g} / \mathrm{cc}$ \\
\hline
Surface Roughness & $3.6 \mu \mathrm{m} \mathrm{Ra}$ \\
\hline
\end{tabular}
\end{center}

\begin{center}
\includegraphics[max width=\textwidth]{2024_03_11_add5292786349c475313g-059}
\end{center}

Inconel 718 Printed Parts

Table displays the average tested values. Material properties tested after solution annealing and aging heat treatments. Heat treating profile in accordance with AMS 5662.

\begin{center}
\begin{tabular}{|c|c|}
\hline
Material Composition & \begin{tabular}{c}
UNS N07718 \\
ASTM B637-18 (wt\%) \\
\end{tabular} \\
\hline
Nickel & $50.0-55.0$ \\
\hline
Chromium & $17.0-21.0$ \\
\hline
Iron & Balance \\
\hline
Niobium+ Tantalum & $4.75-5.50$ \\
\hline
Molybdenum & $2.80-3.30$ \\
\hline
Titanium & $0.65-1.15$ \\
\hline
Aluminum & $0.20-0.80$ \\
\hline
Carbon & $0.04-0.13^{*}$ \\
\hline
Cobalt & $1.00 \mathrm{max}$ \\
\hline
Manganese & 0.35 max \\
\hline
Silicon & $0.35 \mathrm{max}$ \\
\hline
Copper & 0.30 max \\
\hline
Phosphorus & 0.015 max \\
\hline
Sulfur & 0.015 max \\
\hline
Boron & $0.006 \max$ \\
\hline
\end{tabular}
\end{center}

\begin{center}
\includegraphics[max width=\textwidth]{2024_03_11_add5292786349c475313g-059(1)}
\end{center}

Microstructure

\begin{itemize}
  \item The wt\% of carbon in the as-sintered state is $<0.08 \%$. Final carbon results are dependent upon the secondary heat treating conditions
\end{itemize}

ExOne disclaims all warranties and liabilities for the content hereof and makes no representations as to its accuracy or fitness for use for any purpose. Any tradenames, trademarks, or service marks of others appearing herein are used strictly nominatively and are not to be construed as implying any affiliation, connection, association, sponsorship, or approval of the owners thereof for ExOne, its products, or the content hereof.

\section*{Typical Material Properties}
\begin{center}
\begin{tabular}{|c|c|c|}
\hline
Material Properties & Test Method & 304L Stainless Steel \\
\hline
\multicolumn{3}{|l|}{Tensile Strength} \\
\hline
Ultimate Strength & \multirow{4}{*}{ASTM E8} & \begin{tabular}{l}
$X \& Y: 450-570 \mathrm{MPa}$ \\
$Z: 450-570 \mathrm{MPa}$ \\
\end{tabular} \\
\hline
Yield Strength ( $0.2 \%$ offset) &  & \begin{tabular}{l}
X \& Y: $140-200 \mathrm{MPa}$ \\
$Z: 140-200 \mathrm{MPa}$ \\
\end{tabular} \\
\hline
Elongation &  & \begin{tabular}{l}
$X \& Y: 40-60 \%$ \\
Z: $40-55 \%$ \\
\end{tabular} \\
\hline
Elastic Modulus &  & \begin{tabular}{l}
X \& Y: $190-220 \mathrm{GPa}$ \\
Z: $190-200 \mathrm{GPa}$ \\
\end{tabular} \\
\hline
Hardness & ASTM E18 & $67-75$ HRB \\
\hline
Impact & \multirow[t]{4}{*}{ASTM E23} & $35-45 J$ \\
\hline
Relative Density &  & $96-99 \%$ \\
\hline
Density &  & $7.6-7.9 \mathrm{~g} / \mathrm{cc}$ \\
\hline
Surface Roughness &  & $3-12 \mu \mathrm{m} \mathrm{Ra}$ \\
\hline
\end{tabular}
\end{center}

\begin{center}
\includegraphics[max width=\textwidth]{2024_03_11_add5292786349c475313g-060}
\end{center}

304L Stainless Steel Printed Part

\begin{center}
\includegraphics[max width=\textwidth]{2024_03_11_add5292786349c475313g-060(1)}
\end{center}

304L Stainless Steel Microstructure

ExOne disclaims all warranties and liabilities for the content hereof and makes no representations as to its accuracy or fitness for use for any purpose. Any tradenames, trademarks, or service marks of others appearing herein are used strictly nominatively and are not to be construed as implying any affiliation, connection, association, sponsorship, or approval of the owners thereof for ExOne, its products, or the content hereof.

\begin{center}
\includegraphics[max width=\textwidth]{2024_03_11_add5292786349c475313g-061}
\end{center}

Review in Advance first posted online on April 21, 2016. (Changes may still occur before final publication online and in print.)

\section*{Metal Additive Manufacturing: A Review of Mechanical Properties }
Annu. Rev. Mater. Res. 2016. 46:14.1-14.36

The Annual Review of Materials Research is online at \href{http://matsci.annualreviews.org}{matsci.annualreviews.org}

This article's doi:

10.1146/annurev-matsci-070115-032024

Copyright (c) 2016 by Annual Reviews. All rights reserved

\section*{Keywords}
metal additive manufacturing, mechanical properties, fracture, fatigue, toughness, tensile

\begin{abstract}
This article reviews published data on the mechanical properties of additively manufactured metallic materials. The additive manufacturing techniques utilized to generate samples covered in this review include powder bed fusion (e.g., EBM, SLM, DMLS) and directed energy deposition (e.g., LENS, $\left.\mathrm{EBF}^{3}\right)$. Although only a limited number of metallic alloy systems are currently available for additive manufacturing (e.g., Ti-6Al-4V, TiAl, stainless steel, Inconel 625/718, and Al-Si-10Mg), the bulk of the published mechanical properties information has been generated on Ti-6Al-4V. However, summary tables for published mechanical properties and/or key figures are included for each of the alloys listed above, grouped by the additive technique used to generate the data. Published values for mechanical properties obtained from hardness, tension/compression, fracture toughness, fatigue crack growth, and high cycle fatigue are included for as-built, heat-treated, and/or HIP conditions, when available. The effects of test orientation/build direction on properties, when available, are also provided, along with discussion of the potential source(s) (e.g., texture, microstructure changes, defects) of anisotropy in properties. Recommendations for additional work are also provided.
\end{abstract}

\section*{INTRODUCTION}
A number of metal additive manufacturing (AM) processes are currently available (1), depending on the heat source $(2,3)$, such as electron beam $(2-7)$, laser, or arc $(2,3,8,9)$, and on how the raw material is supplied. Materials supply can occur via powder or wire feed, whereby selected regions are melted at different combinations of absorbed power $(P)$ and beam velocity $(V)(10)$, as shown in Figure 1, and then solidified. Cooling rates during and after solidification are affected and controlled by the $P-V$ combinations utilized and by any preheating of the substrate. These variables, along with the subsequent thermal cycles that occur during such layered manufacturing as well as any postprocessing (e.g., heat treatment, HIP), affect the resulting microstructures, as reviewed previously (10) and in another contribution to this journal's keynote topic on AM (11). Nonequilibrium microstructures and defects can result in as-built materials, depending on the processing conditions and materials employed, whereas postprocessing via heat treatment and/or HIP can be used to change some of the microstructural features and to reduce or eliminate defects and any residual stresses. These changes affect both the orientation dependence of mechanical properties and their magnitude, as this article documents. A broader review on materials qualification needs for metal AM is provided elsewhere (10).

The recent reviews of the metal AM processes by Frazier (2) and Dutta \& Froes (3) highlight some of the differences between the various processes. Figure 2 provides the two major metal AM process categories reviewed in this article, powder bed fusion (PBF) and directed energy deposition (DED). Figure 2 also includes designations for the technologies currently available within each major process category [e.g., direct metal laser melting (DMLM), selective laser melting (SLM), selective laser sintering (SLS), direct metal deposition (DMD)] and current commercial machine\\
\includegraphics[max width=\textwidth, center]{2024_03_11_add5292786349c475313g-062}

Figure 1

Typical combinations of power $(P)$ and velocity $(V)$ in various metal AM processes. Abbreviations: $\mathrm{EBF}^{3}$, electron beam freeform fabrication; LENS, laser-engineered net shaping; SLM, selective laser melting. Adapted with permission from Reference 10.

\begin{center}
\includegraphics[max width=\textwidth]{2024_03_11_add5292786349c475313g-063}
\end{center}

Figure 2

Summary of metal additive manufacturing processes.

Table 1 Summary of various alloys used in different process categories, along with references

\begin{center}
\includegraphics[max width=\textwidth]{2024_03_11_add5292786349c475313g-064}
\end{center}

suppliers. To systematically review the published values for mechanical properties obtained for materials manufactured by these different techniques, this review begins with a compilation of the most widely utilized AM alloy systems along with the process category (e.g., PBF, DED) and energy source for fusion [e.g., electron beam melting (EBM), laser, and wire and arc additive manufacturing (WAAM)]. Individual tables and/or figures for each alloy, energy source, and/or mechanical property are then provided for as-built, heat-treated, and/or HIP conditions and document the test orientation and build direction when available.

\section*{ADDITIVELY MANUFACTURED ALLOY SYSTEMS}
As indicated in previous reviews $(2,3)$, at present there is only a limited number of alloy systems for which mechanical properties are published. Table 1 summarizes the existing alloy classes and references to published data, along with the process category and source of fusion. These categories provide the basis for the remainder of this review.

\section*{MECHANICAL PROPERTIES OF ADDITIVELY MANUFACTURED METALLIC MATERIALS}
Although most of the published mechanical property measurements have been reported for Ti$6 \mathrm{Al}-4 \mathrm{~V}$, tables and/or figures summarizing data for each of the alloy classes shown in Table $\mathbf{1}$ are provided, when available. Review of the literature also reveals that most of the published work has focused on tension/compression testing, with more recent work on fracture-critical properties. In the tables, the effects of specimen or build orientation on tensile properties are documented using the $X, Y, Z$ designation according to the ASTM standard (130) shown in Figure 3, when documented in the published work. Rectangular and nonsymmetric test coupons thus require three letters $(X, Y, Z)$ to provide a complete orientation designation. In this terminology, $Z$ designates the build direction. The $X$ axis is parallel to the front of the machine and is perpendicular to $Z$. The $Y$ axis is perpendicular to both the $Z$ and $X$ axes, with a positive direction defined to make a righthand set of coordinates. The first letter designates the axis parallel to the longest overall dimension. The second letter designates the second-longest overall dimension, followed by the third letter, which designates the third-longest overall dimension of the coupon. For example, a specimen with $X Y Z$ designation has its longest dimension parallel to $X$, its second-longest dimension parallel to $Y$, and its shortest overall dimension parallel to $Z$. Figure 3 also illustrates that only one letter\\
\includegraphics[max width=\textwidth, center]{2024_03_11_add5292786349c475313g-065}

Figure 3

Orientation designations for AM materials.

is required for cylindrically symmetric samples. Unfortunately, not all of the published works reviewed herein followed these ASTM/ISO rules. In some cases, only one letter was used for nonsymmetric samples. The tables also document any postprocessing (e.g., heat treatment, HIP) that was used.

ASTM committee F42 (131) is reviewing potential modifications to the orientation designation scheme for fracture toughness and fatigue crack growth, as shown in Figure 4 (48). These modifications to the evolving ASTM standards for AM materials may be necessary to document the unique orientation- and location-dependent properties that can be produced both within and between builds in AM-processed materials $(10,48)$ due to differences in the microstructure, texture, residual stresses, and/or defects. These types of samples could also serve as witness samples deposited along with components in the same build to provide input on part/component quality in different locations and orientations.

\begin{center}
\includegraphics[max width=\textwidth]{2024_03_11_add5292786349c475313g-066}
\end{center}

Figure 4

Possible designations for AM mechanical testing based on existing ASTM standards. There are eight different orientation and direction combinations. Abbreviations: L, longitudinal; S, short; T, transverse. Adapted from References 48 and 131 with permission.

\section*{QUASI-STATIC PROPERTIES: TENSILE AND FRACTURE TOUGHNESS}
Tables 2 and 3 compile published tensile properties for Ti-6Al-4V produced via PBF EBM and laser techniques, respectively, and include hardness data when available along with literature references. Tables 2 and $\mathbf{3}$ show the machine type and powder [e.g., conventional versus ELI (extra low interstitial)] utilized; show the specimen orientation using the $X, Y, Z$ scheme presented in Figure 3; and specify whether the material was tested in the as-built, heat-treated, and/or HIP condition. Figure 5 captures some of the key early observations on $A M$ tensile properties for Ti-6Al-4V, and the tables and figures contained herein provide updated details along with many additional references.

The summary in Table 2 for PBF (EBM) reveals orientation-dependent values for yield strength, ultimate tensile strength (UTS), and elongation to failure that are also affected by postprocessing heat treatments and/or HIP. Because of differences in the sample gauge lengths between the different investigations, the reported elongations to failure are difficult to compare directly. However, properties reported for all of the conditions (as built, heat treated, and/or HIP) often approach and exceed those reported for conventionally processed Ti-6Al-4V (3). EBM machine type (e.g., Arcam A1, A2, A2xx, S12, S400) appears to affect the reported properties, although property variations have also been documented on samples manufactured within one machine type. The source(s) of these variations could be explored by conducting round robin activities like those organized by NIST/ASTM $(132,133)$ for PBF (laser), and the source(s) of the orientation-dependent properties is starting to receive additional attention from the AM community. Orientation-dependent differences in the microstructure, texture, and defects contribute to some of these tensile property differences but become more important in the fracture-critical properties (e.g., HCF, fatigue crack growth, fracture toughness) reviewed below. Postprocessing (e.g., heat treatment and/or HIP) can be used to produce more desirable microstructures and to reduce or eliminate process-induced defects [e.g., lack of fusion (LoF), isolated porosity] but affects the cost-effectiveness of the process. In general, the use of preheated powder beds in the EBM process reduces the cooling rate during and after the AM process, typically producing

Table 2 Summary of EBM PBF AM Ti-6Al-4V tensile properties

\begin{center}
\begin{tabular}{|c|c|c|c|c|c|c|c|c|}
\hline
\begin{tabular}{l}
Machine \\
type \\
\end{tabular} & Condition & \begin{tabular}{c}
Specimen \\
orientation \\
\end{tabular} & $E(\mathrm{GPa})$ & \begin{tabular}{c}
Yield \\
strength \\
$(\mathrm{MPa})$ \\
\end{tabular} & \begin{tabular}{c}
Ultimate \\
tensile \\
strength \\
$(\mathrm{MPa})$ \\
\end{tabular} & \begin{tabular}{c}
Elongation \\
$(\%)$ \\
\end{tabular} & \begin{tabular}{c}
Hardness \\
$(\mathrm{Hv})$ \\
\end{tabular} & Reference \\
\hline
Arcam & \begin{tabular}{l}
Heat \\
treated \\
\end{tabular} & $Z X$ & $\mathrm{NA}$ & $869 \pm 7$ & $965 \pm 5$ & $6 \pm 0$ & NA & 141 \\
\hline
\multirow{2}{*}{}\begin{tabular}{l}
Arcam \\
A1 \\
\end{tabular} & \multirow[t]{2}{*}{As built} & $X Y$ & \multirow[t]{2}{*}{NA} & $783 \pm 15$ & $833 \pm 22$ & $2.7 \pm 0.4$ & \multirow[t]{2}{*}{$\mathrm{NA}$} & \multirow[t]{2}{*}{142} \\
\hline
 &  & $Z X$ &  & $812 \pm 12$ & $851 \pm 19$ & $3.6 \pm 0.9$ &  &  \\
\hline
\multirow[t]{4}{*}{Arcam} & \multirow[t]{2}{*}{As built} & $X Y$ & \multirow[t]{4}{*}{$\mathrm{NA}$} & $870 \pm 8.1$ & $971 \pm 3.1$ & $12.1 \pm 0.9$ & \multirow[t]{4}{*}{NA} & \multirow[t]{8}{*}{12} \\
\hline
 &  & $Z$ &  & $879 \pm 12.5$ & $953 \pm 8.8$ & $13.8 \pm 0.9$ &  &  \\
\hline
 & \multirow[t]{2}{*}{HIP} & $X Y$ &  & $866 \pm 6.4$ & $959 \pm 8.2$ & $13.6 \pm 0.6$ &  &  \\
\hline
 &  & $Z$ &  & $868 \pm 2.9$ & $942 \pm 2.6$ & $12.9 \pm 0.8$ &  &  \\
\hline
\multirow{4}{*}{}\begin{tabular}{c}
Arcam \\
ELI $^{\mathrm{a}}$ \\
\end{tabular} & \multirow[t]{2}{*}{As built} & $X Y$ & \multirow[t]{4}{*}{NA} & $817 \pm 4.3$ & $918 \pm 1.0$ & $12.6 \pm 0.8$ & \multirow[t]{4}{*}{NA} &  \\
\hline
 &  & $Z$ &  & $802 \pm 7.9$ & $904 \pm 6.0$ & $13.8 \pm 0.9$ &  &  \\
\hline
 & \multirow[t]{2}{*}{HIP} & $X Y$ &  & $814 \pm 2.4$ & $916 \pm 2.5$ & $13.6 \pm 1.2$ &  &  \\
\hline
 &  & $Z$ &  & $807 \pm 8.4$ & $902 \pm 8.7$ & $14.8 \pm 0.5$ &  &  \\
\hline
\begin{tabular}{r}
Arcam \\
A2xx \\
$\mathrm{ELI}^{\mathrm{a}}$ \\
\end{tabular} & As built & $X Y$ & $\mathrm{NA}$ & $851.8 \pm 5.8$ & $964 \pm 0.3$ & $16.3 \pm 0.8$ & $\mathrm{NA}$ & 143 \\
\hline
\multirow{2}{*}{}\begin{tabular}{l}
Arcam \\
A2 \\
ELI $^{\mathrm{a}}$ \\
\end{tabular} & As built & $Z$ & $\mathrm{NA}$ & $928 \pm 13.3$ & $1,011 \pm 14.8$ & $13.6 \pm 1.4$ & $\mathrm{NA}$ & \multirow[t]{2}{*}{31} \\
\hline
 & HIP & $Z$ & $\mathrm{NA}$ & $813 \pm 14.3$ & $908 \pm 3.2$ & $17.7 \pm 0.9$ & $\mathrm{NA}$ &  \\
\hline
\begin{tabular}{l}
Arcam \\
S12 \\
\end{tabular} & As built & $X Y$ & $\mathrm{NA}$ & 975 & 1,033 & 16.78 & $\mathrm{NA}$ & 144 \\
\hline
\multirow[t]{2}{*}{Arcam} & As built & $X Y$ & NA & $881 \pm 12.5$ & $978 \pm 11.5$ & $10.7 \pm 1.5$ & NA & \multirow[t]{2}{*}{33} \\
\hline
 & HIP & $X Y$ & $\mathrm{NA}$ & $876 \pm 12.5$ & $978 \pm 9.5$ & $13.5 \pm 1.5$ & NA &  \\
\hline
\multirow{2}{*}{}\begin{tabular}{l}
Arcam \\
S12 \\
\end{tabular} & \multirow[t]{2}{*}{As built} & $X Y$ & $\mathrm{NA}$ & $982 \pm 5.7$ & $1,029 \pm 7$ & $12.2 \pm 0.8$ & $372 \pm 7.2$ & \multirow[t]{2}{*}{145} \\
\hline
 &  & $Z$ & $\mathrm{NA}$ & $984 \pm 8.5$ & $1,032 \pm 12.9$ & $9 \pm 2.9$ & $367 \pm 8.3$ &  \\
\hline
\multirow{2}{*}{}\begin{tabular}{c}
Arcam \\
S400 \\
\end{tabular} & \multirow[t]{2}{*}{As built} & $X Y$ & \multirow[t]{2}{*}{NA} & $899 \pm 4.7$ & $978 \pm 3.2$ & $9.5 \pm 1.2$ & \multirow[t]{2}{*}{$\mathrm{NA}$} & \multirow[t]{2}{*}{39} \\
\hline
 &  & $Z X$ &  & $869 \pm 7.2$ & $928 \pm 9.8$ & $9.9 \pm 1.7$ &  &  \\
\hline
\multirow{2}{*}{}\begin{tabular}{c}
Arcam \\
S400 \\
\end{tabular} & \multirow[t]{2}{*}{As built} & $X Y$ & $104 \pm 2.3$ & $844 \pm 21.6$ & $917 \pm 30.53$ & $8.8 \pm 1.42$ & NA & \multirow[t]{2}{*}{40} \\
\hline
 &  & $Z$ & $101 \pm 2.5$ & $782 \pm 5.1$ & $842 \pm 13.84$ & $9.9 \pm 1.02$ & $\mathrm{NA}$ &  \\
\hline
\begin{tabular}{r}
Arcam \\
S400 \\
ELI $^{\mathrm{a}}$ \\
\end{tabular} & As built & $Z$ & $\mathrm{NA}$ & 1,150 & 1,200 & 16 & 380 & 146 \\
\hline
Arcam & As built & $\mathrm{NA}$ & $118 \pm 5$ & $830 \pm 5$ & $915 \pm 10$ & $13.1 \pm 0.4$ & $\mathrm{NA}$ & 16 \\
\hline
 & HIP & $\mathrm{NA}$ & $117 \pm 4$ & $795 \pm 10$ & $870 \pm 10$ & $13.7 \pm 1$ & NA &  \\
\hline
\begin{tabular}{l}
Arcam \\
A2 \\
ELI $^{\mathrm{a}}$ \\
\end{tabular} & As built & $Z$ & $93 \pm 2$ & $735 \pm 28$ & $775 \pm 26$ & $2.3 \pm 0.8$ & $369 \pm 2$ & 29 \\
\hline
\end{tabular}
\end{center}

Table 2 (Continued)

\begin{center}
\begin{tabular}{|c|c|c|c|c|c|c|c|c|}
\hline
\begin{tabular}{l}
Machine \\
type \\
\end{tabular} & Condition & \begin{tabular}{l}
Specimen \\
orientation \\
\end{tabular} & $E(\mathrm{GPa})$ & \begin{tabular}{c}
Yield \\
strength \\
$(\mathrm{MPa})$ \\
\end{tabular} & \begin{tabular}{l}
Ultimate \\
tensile \\
strength \\
$(\mathrm{MPa})$ \\
\end{tabular} & \begin{tabular}{c}
Elongation \\
$(\%)$ \\
\end{tabular} & \begin{tabular}{c}
Hardness \\
$(\mathbf{H v})$ \\
\end{tabular} & Reference \\
\hline
\multirow{3}{*}{}\begin{tabular}{l}
Arcam \\
$\mathrm{ELI}^{\mathrm{a}}$ \\
\end{tabular} & HIP & $X Y$ & $\mathrm{NA}$ & 841 & 938 & 20 & NA & \multirow[t]{3}{*}{14} \\
\hline
 & As built & $Z$ & $\mathrm{NA}$ & 856 & 924 & 15 & $\mathrm{NA}$ &  \\
\hline
 & HIP & $Z$ & NA & 800 & 876 & 16 & $\mathrm{NA}$ &  \\
\hline
Arcam & As built & NA & $114 \pm 6$ & $1,135 \pm 12$ & NA & NA & NA & 147 \\
\hline
\begin{tabular}{c}
Arcam \\
S400 \\
\end{tabular} & As built & $Z$ & $109 \pm 2.1$ & $1,098 \pm 15$ & $1,237 \pm 13$ & $8.8 \pm 0.6$ & NA & 148 \\
\hline
Arcam & As built & $\mathrm{NA}$ & 128 & 880 & 930 & $>10 \%$ & $\mathrm{NA}$ & 149 \\
\hline
\end{tabular}
\end{center}

NA denotes data not available.

${ }^{\text {a ELI (extra low interstitial) powder was used. }}$

$\alpha+\beta$ lamellar microstructures [with prior $\beta$ grain sizes that can be affected or controlled by the combinations of $P$ and $V$ utilized in the process (134-140)]. EBM AM materials typically possess lower levels of residual stresses in the as-built condition than do materials made by laser-based techniques that typically use no preheat; the faster cooling rates typically produce highly nonequilibrium microstructures [e.g., martensite in PBF (laser) Ti-6Al-4V] and much higher levels of residual stress that require subsequent stress relief treatments, as described below. Chemistry control in the PBF (EBM) process can also become an issue in Ti-6Al-4V due to the preferential loss of $\mathrm{Al}$ during EBM of powders in high vacuum.

Table 3 illustrates that PBF (laser) of Ti-6Al-4V exhibited features (e.g. orientation-dependent properties, machine effects $(132,133)$, postprocessing improvements to properties) similar to those shown for PBF (EBM) in Table 2. Highly nonequilibrium microstructures (e.g., martensite), along with substantial residual stresses that increase the strength and decrease the elongation values, are possible in as-built Ti-6Al-4V. Postprocessing has been used to increase the elongation to failure while reducing the yield strength, UTS, and residual stress values.

Table 4, which summarizes DED (laser) tensile properties, shows similar general characteristics of orientation- and machine-dependent properties, with values for as-built yield strength, UTS, and elongation to failure generally between the values exhibited in Tables 2 and $\mathbf{3}$ for PBF (EBM) and PBF (laser), respectively. The combinations of $P$ and $V$ shown in Figure 1 for DED generally produce $\alpha+\beta$ lamellar microstructures, with prior $\beta$ grain sizes somewhat larger than those obtained from PBF (laser) due to the slower cooling rates typically present in $\operatorname{DED}(57,59$, 134-137). Table 4 also shows that HIP can result in significant increases to the elongation-tofailure values via the elimination of process-induced defects. These process-induced defects are particularly detrimental to the high cycle fatigue behavior, as discussed below.

Table 5 summarizes the more limited published work on tensile properties for all of the other alloys in Table 1 manufactured using PBF techniques. The limited published tensile properties for 316L PBF (laser) reveal properties in the range of commercially produced 316L. HIP of 316L produced via SLM increased elongation, likely due to the elimination of process-induced defects.

\includegraphics[max width=\textwidth, center]{2024_03_11_add5292786349c475313g-068}\\
Both $\mathrm{Al}-12 \mathrm{Si}$ and $\mathrm{Al}-\mathrm{Si}-10 \mathrm{Mg}$ are alloys typically processed via commercial casting techniques (e.g., sand, gravity die). Strength levels in PBF (laser)-processed versions of Al-12Si and Al-Si$10 \mathrm{Mg}$ are in the range of data produced via sand and die casting techniques, whereas the somewhat higher elongation values arise due to the microstructure refinement provided by the faster cooling

Table 3 Summary of laser-melted PBF AM Ti-6Al-4V tensile properties

\begin{center}
\includegraphics[max width=\textwidth]{2024_03_11_add5292786349c475313g-069}
\end{center}

Table 3 (Continued)

\begin{center}
\begin{tabular}{|c|c|c|c|c|c|c|c|c|}
\hline
\begin{tabular}{l}
Machine \\
type \\
\end{tabular} & Condition & \begin{tabular}{c}
Specimen \\
orienta- \\
tion \\
\end{tabular} & $E(\mathrm{GPa})$ & \begin{tabular}{c}
Yield \\
strength \\
$(\mathrm{MPa})$ \\
\end{tabular} & \begin{tabular}{l}
Ultimate \\
tensile \\
strength \\
$(\mathrm{MPa})$ \\
\end{tabular} & \begin{tabular}{c}
Elongation \\
$(\%)$ \\
\end{tabular} & Hardness & Reference \\
\hline
$\overline{S L M}$ & As built & $X Y$ & \begin{tabular}{c}
$109.2 \pm$ \\
3.1 \\
\end{tabular} & $1,110 \pm 9$ & $1,267 \pm 5$ & $7.28 \pm 1.12$ & $\mathrm{NA}$ & 156 \\
\hline
\multirow[t]{2}{*}{EOS M270} & As built & \multirow[t]{2}{*}{NA} & $110 \pm 5$ & $990 \pm 5$ & $1,095 \pm 10$ & $8.1 \pm 0.3$ & NA & \multirow[t]{2}{*}{51} \\
\hline
 & \begin{tabular}{l}
Heat \\
treated \\
\end{tabular} &  & NA & $1,040 \pm 10$ & $1,140 \pm 10$ & $8.2 \pm 0.3$ & NA &  \\
\hline
EOS M270 & As built & $Z X$ & 111 & 1,120 & 1,257 & 8.0 & 37 (HRC) & 148 \\
\hline
EOS M27 & As built & $Z$ & NA & 1,333 & 1,407 & 4.54 & $\mathrm{NA}$ & 157 \\
\hline
\multirow{4}{*}{}\begin{tabular}{l}
SLM \\
(Trumpf) \\
\end{tabular} & \multirow[t]{2}{*}{As built} & $X Y$ & $105 \pm 5$ & $1,137 \pm 20$ & $1,206 \pm 8$ & $7.6 \pm 2$ & \multirow[t]{4}{*}{NA} & \multirow[t]{4}{*}{158} \\
\hline
 &  & $Z X$ & $102 \pm 7$ & $962 \pm 47$ & $1,166 \pm 25$ & $1.7 \pm 0.3$ &  &  \\
\hline
 & \multirow[t]{2}{*}{}\begin{tabular}{l}
Heat \\
treated \\
\end{tabular} & $X Y$ & \begin{tabular}{c}
$103 \pm$ \\
11 \\
\end{tabular} & $944 \pm 8$ & $1,036 \pm 30$ & $8.5 \pm 1$ &  &  \\
\hline
 &  & $Z X$ & $98 \pm 3$ & $925 \pm 14$ & $1,040 \pm 40$ & $7.5 \pm 2$ &  &  \\
\hline
SLM & As built & NA & 94 & 1,125 & 1,250 & 6 & NA & 159 \\
\hline
\begin{tabular}{l}
Renishaw \\
MTT \\
\end{tabular} & As built & $X$ & NA & $1,166 \pm 6$ & $1,321 \pm 6$ & $2.0 \pm 0.7$ & NA & 160 \\
\hline
$\overline{\mathrm{DLF}}$ & As built & $X$ & \begin{tabular}{c}
$118 \pm$ \\
2.3 \\
\end{tabular} & $1,100 \pm 12$ & $1,211 \pm 31$ & $6.5 \pm 0.6$ & NA & 161 \\
\hline
\multirow{2}{*}{}\begin{tabular}{l}
Concept \\
Laser M2 \\
\end{tabular} & \multirow[t]{2}{*}{As built} & $X$ & \multirow[t]{2}{*}{$\mathrm{NA}$} & $1,070 \pm 50$ & $1,250 \pm 50$ & $5.5 \pm 1$ & \multirow[t]{2}{*}{NA} & \multirow[t]{2}{*}{162} \\
\hline
 &  & $Z$ &  & $1,050 \pm 40$ & $1,180 \pm 30$ & $8.5 \pm 1.5$ &  &  \\
\hline
\end{tabular}
\end{center}

NA denotes data not available.

rates in the PBF (laser) processes used. The very limited published tension data on CoCrMo reveal that PBF (laser) exhibits somewhat higher strengths and lower elongation to failure than does PBF (EBM) in the as-built condition; these data also indicate orientation-dependent properties in the PBF (EBM) material. HIP and heat treatment of PBF (EBM) CoCrMo removed the orientation effects on properties and significantly increased the elongation to failure, consistent with HIP elimination of process-induced defects. There are not enough published data on the other alloy systems listed in Table 5 to make sensible comparisons at this time.

Table 6 summarizes the evolving database for DED, focusing on Inconel 718 (IN718). Properties are shown for both as-built and heat-treated conditions for a variety of machine types using either laser melting or EBM. Although there are not enough data reported within one machine type to make sensible comparisons, Table 6 shows significant differences between the properties obtained on IN718 processed across the different machine types and energy sources used for melting.

Whereas there has been extensive research to determine the range of uniaxial tensile properties possible for PBF (EBM, laser) and DED (laser) reported in Tables 2-6, much less published research is available for the fracture-critical properties (e.g., toughness, fatigue) of Ti-6Al-4V, and very few published data exist for the fracture properties of the other alloy systems listed in Table 1. Table 7 summarizes fracture toughness properties of Ti-6Al-4V for both PBF (laser)- and PBF (EBM)-processed materials, again using the $X, Y, Z$ nomenclature shown in Figure 3. Candidate fracture toughness numbers, $K_{\mathrm{q}}$, are provided in Table 7 because thickness requirements for valid fracture toughness (i.e., $K_{\mathrm{Ic}}$ ) measurements are not met in PBF (EBM) Ti-6Al-4V (48) and are not

\begin{center}
\includegraphics[max width=\textwidth]{2024_03_11_add5292786349c475313g-071(1)}
\end{center}

Figure 5

Summary of Ti-6Al-4V AM tensile properties. Abbreviations: DMD, direct metal deposition; DMLS, direct metal laser sintering; EBM, electron beam melting; HT, heat treated; LENS, laser-engineered net shaping; UTS, ultimate tensile strength; YS, yield stress. Adapted from Reference 3.

reported for PBF (laser) Ti-6Al-4V (151). Again, directly comparing toughness numbers between various works is difficult because few of the published values were obtained on samples sufficiently thick to provide valid $K_{\mathrm{Ic}}$ (i.e., plane strain) measurements and because non-plane strain conditions (i.e., thinner samples) inflate toughness numbers due to plane stress conditions. Nonetheless, PBF (laser)-processed Ti-6Al-4V exhibits toughness values well below those of conventionally processed Ti-6Al-4V (48, 140, 196) and exhibits orientation-dependent values and significant effects of machine type and postprocessing conditions. In general, the as-built PBF (laser) Ti-6Al$4 \mathrm{~V}$ exhibits the lowest toughness in the as-built condition, likely due to a combination of highly nonequilibrium microstructures, significant residual stresses, and process-induced defects. Stress relief, heat treatment, and/or HIP postprocessing appear to improve the toughness values reported in Table 7 by reducing harmful residual stresses, by generating more favorable microstructures, and by minimizing defects in the as-built PBF (laser) material. The highest published toughness for the PBF (laser) Ti-6Al-4V reported in Table 7 belongs to the EOS M280 processed material $\left(K_{\mathrm{q}}=86.3 \mathrm{MPa} \sqrt{ } \mathrm{m}\right)$ after a postprocessing heat treatment $(194)$, whereas the as-built and/or HIP versions exhibited significantly lower toughness values.

In contrast, Table 7 reveals significantly higher toughness values from preliminary studies on PBF (EBM) Ti-6Al-4V $(48,140)$ in both as-built and HIP-processed material, although machine-, orientation-, and location-dependent toughness values were exhibited. More recent work (10) suggested that both microstructure and texture variations and defect population vary with different build orientations, locations, and machines, thereby affecting the magnitude of toughness. Figure 6 shows the fracture surface of an as-built PBF (EBM) Ti-6Al-4V toughness sample tested in the LT-BOTH orientation (10) shown in Figure 4. The various LoF defects that are evident perpendicular to the build direction in this LT-BOTH sample likely contribute to the orientationdependent toughness values, although microstructural differences along and perpendicular to the build can also contribute (10), as suggested in Figure 7.

\begin{center}
\includegraphics[max width=\textwidth]{2024_03_11_add5292786349c475313g-071}
\end{center}

Table 4 Summary of laser-fusion DED AM Ti-6Al-4V tensile properties

\begin{center}
\begin{tabular}{|c|c|c|c|c|c|c|c|c|}
\hline
\begin{tabular}{l}
Machine \\
type \\
\end{tabular} & Condition & \begin{tabular}{c}
Specimen \\
orientation \\
\end{tabular} & $E(\mathrm{GPa})$ & \begin{tabular}{c}
Yield \\
strength \\
$(\mathrm{MPa})$ \\
\end{tabular} & \begin{tabular}{l}
Ultimate \\
tensile \\
strength \\
$(\mathrm{MPa})$ \\
\end{tabular} & \begin{tabular}{c}
Elongation \\
$(\%)$ \\
\end{tabular} & Hardness & Reference \\
\hline
\multirow{6}{*}{}\begin{tabular}{l}
LENS \\
$(\mathrm{Op-}$ \\
tomec) \\
\end{tabular} & \multirow{3}{*}{}\begin{tabular}{l}
Stress \\
relieved \\
\end{tabular} & $X$ & 116 & 1,065 & 1,109 & 4.9 & \multirow[t]{6}{*}{NA} & \multirow[t]{6}{*}{58} \\
\hline
 &  & $Y$ & 116 & 1,066 & 1,112 & 5.5 &  &  \\
\hline
 &  & $Z$ & 112 & 832 & 832 & 0.8 &  &  \\
\hline
 & \multirow[t]{3}{*}{HIP} & $X$ & 118 & 946 & 1,005 & 13.1 &  &  \\
\hline
 &  & $Y$ & 118 & 952 & 1,007 & 13.0 &  &  \\
\hline
 &  & $Z$ & 114 & 899 & 1,002 & 11.8 &  &  \\
\hline
\multirow{3}{*}{}\begin{tabular}{l}
DLD \\
(Trumpf) \\
\end{tabular} & \multirow[t]{2}{*}{As built} & $X$ & \multirow[t]{3}{*}{$\mathrm{NA}$} & $950 \pm 2$ & $1,025 \pm 10$ & $12 \pm 1$ & \multirow[t]{3}{*}{NA} & \multirow[t]{3}{*}{72} \\
\hline
 &  & $Z$ &  & $950 \pm 2$ & $1,025 \pm 2$ & $5 \pm 1$ &  &  \\
\hline
 & HIP & NA &  & $850 \pm 2$ & $920 \pm 1$ & $17 \pm 2$ &  &  \\
\hline
LMD & As built & $X$ & NA & $976 \pm 24$ & $1,099 \pm 2$ & $4.9 \pm 0.1$ & NA & 163 \\
\hline
$\overline{\mathrm{LSF}}$ & As built & $Z$ & $\mathrm{NA}$ & 1,070 & 1,140 & 6 & NA & 164 \\
\hline
\multirow[t]{8}{*}{$\overline{\mathrm{LF}^{3}}$} & As built & $X$ & \multirow[t]{8}{*}{$\mathrm{NA}$} & $892 \pm 10$ & $911 \pm 10$ & $6.4 \pm 0.6$ & \multirow[t]{8}{*}{NA} & \multirow[t]{8}{*}{165} \\
\hline
 & As built & $Z$ &  & 522 & $797 \pm 27$ & $1.7 \pm 0.3$ &  &  \\
\hline
 & \begin{tabular}{l}
As built \\
(ma- \\
chined) \\
\end{tabular} & $X$ &  & $984 \pm 25$ & $1,069 \pm 19$ & $5.4 \pm 1$ &  &  \\
\hline
 & \begin{tabular}{l}
As built \\
(ma- \\
chined) \\
\end{tabular} & $Z$ &  & $958 \pm 14$ & $1,026 \pm 17$ & $3.8 \pm 0.9$ &  &  \\
\hline
 & \begin{tabular}{l}
Heat \\
treated \\
\end{tabular} & $X$ &  & $681 \pm 35$ & $750 \pm 20$ & $4.8 \pm 1.6$ &  &  \\
\hline
 & \begin{tabular}{l}
Heat \\
treated \\
\end{tabular} & $Z$ &  & $637 \pm 13$ & $717 \pm 12$ & $3.4 \pm 1.0$ &  &  \\
\hline
 & \begin{tabular}{l}
Heat \\
treated \\
(ma- \\
chined) \\
\end{tabular} & $X$ &  & $870 \pm 37$ & $953 \pm 18$ & $11.8 \pm 1.3$ &  &  \\
\hline
 & \begin{tabular}{l}
Heat \\
treated \\
(ma- \\
chined) \\
\end{tabular} & $Z$ &  & $930 \pm 15$ & $942 \pm 13$ & $9.7 \pm 2.2$ &  &  \\
\hline
\multirow[t]{2}{*}{$\overline{\mathrm{DMD}}$} & As built & $X$ & \multirow[t]{2}{*}{$\mathrm{NA}$} & $1,105 \pm 19$ & $1,163 \pm 22$ & $4 \pm 1$ & \multirow[t]{2}{*}{NA} & \multirow[t]{2}{*}{166} \\
\hline
 & \begin{tabular}{l}
Heat \\
treated \\
\end{tabular} & $X$ &  & $975 \pm 15$ & $1,053 \pm 18$ & $7.5 \pm 1$ &  &  \\
\hline
\multirow[t]{2}{*}{$\overline{\text { IPG YLR }}$} & \multirow[t]{2}{*}{As built} & $X$ & \multirow[t]{2}{*}{NA} & $960 \pm 26$ & $1,063 \pm 20$ & $10.9 \pm 1.4$ & \multirow[t]{2}{*}{NA} & 80 \\
\hline
 &  & $Z$ &  & $958 \pm 19$ & $1,064 \pm 26$ & $14 \pm 1$ &  &  \\
\hline
\end{tabular}
\end{center}

\begin{center}
\includegraphics[max width=\textwidth]{2024_03_11_add5292786349c475313g-072}
\end{center}

Table 4 (Continued)

\begin{center}
\begin{tabular}{|c|c|c|c|c|c|c|c|c|}
\hline
\begin{tabular}{l}
Machine \\
type \\
\end{tabular} & Condition & \begin{tabular}{l}
Specimen \\
orientation \\
\end{tabular} & $E(\mathrm{GPa})$ & \begin{tabular}{c}
Yield \\
strength \\
$(\mathrm{MPa})$ \\
\end{tabular} & \begin{tabular}{c}
Ultimate \\
tensile \\
strength \\
$(\mathrm{MPa})$ \\
\end{tabular} & \begin{tabular}{c}
Elongation \\
$(\%)$ \\
\end{tabular} & Hardness & Reference \\
\hline
\multirow[t]{4}{*}{}\begin{tabular}{l}
LENS \\
$(\mathrm{Op-}$ \\
tomec) \\
\end{tabular} & \begin{tabular}{l}
As built \\
(low \\
power) \\
\end{tabular} & \multirow[t]{4}{*}{$X$} & \multirow[t]{4}{*}{NA} & 1,005 & 1,103 & 4 & \multirow[t]{4}{*}{NA} & \multirow[t]{4}{*}{167} \\
\hline
 & \begin{tabular}{l}
Heat \\
treated \\
(low \\
power) \\
\end{tabular} &  &  & 1,000 & 1,073 & 9 &  &  \\
\hline
 & \begin{tabular}{l}
As built \\
(high \\
power) \\
\end{tabular} &  &  & 990 & 1,042 & 7 &  &  \\
\hline
 & \begin{tabular}{l}
Heat \\
treated \\
(high \\
power) \\
\end{tabular} &  &  & 991 & 1,044 & 10 &  &  \\
\hline
\begin{tabular}{l}
Laser \\
form- \\
ing \\
\end{tabular} & \begin{tabular}{l}
Heat \\
treated \\
\end{tabular} & $\mathrm{NA}$ & NA & 839 & 900 & 12.3 & NA & 168 \\
\hline
$\overline{\mathrm{DLF}}$ & \begin{tabular}{l}
Heat \\
treated \\
\end{tabular} & NA & NA & 958 & 1,027 & 6.2 & $\mathrm{NA}$ & 169 \\
\hline
LENS & \begin{tabular}{l}
Heat \\
treated \\
\end{tabular} & NA & NA & $827-965$ & $896-1,000$ & $1-16$ & NA & 170 \\
\hline
\multirow[t]{3}{*}{LENS} & As built & \multirow[t]{3}{*}{$Z$} & 119 & 908 & 1,038 & 3.8 & \multirow[t]{3}{*}{NA} & \multirow[t]{3}{*}{226} \\
\hline
 & Annealed &  & 112 & 959 & 1,049 & 3.7 &  &  \\
\hline
 & \begin{tabular}{l}
Heat \\
treated \\
\end{tabular} &  & 118 & 957 & 1,097 & 3.4 &  &  \\
\hline
\end{tabular}
\end{center}

NA denotes data not available. Other abbreviations: DLD, direct laser fabrication; DMD, direct metal deposition; LMD, laser metal deposition; LSF, laser solid forming; $\mathrm{LF}^{3}$, laser freeform fabrication.

Figure 8 shows $\mu \mathrm{CT}$ analyses revealing process-induced defects in a large (i.e., $10 \times 20 \times 100 \mathrm{~mm}$ ) as-built PBF (EBM) LT-BOTH Ti-6Al-4V fracture toughness sample that was tested to failure. Interestingly, although $\mu \mathrm{CT}$ analyses showed that HIP minimized or eliminated the defects present in Figure 8, Table 7 and recently published work (10) report lower toughness values for HIP PBF (EBM) Ti-6Al-4V in comparison to the as-built material. Ongoing work is examining the details of the crack path and microstructure interactions to determine the source(s) of this reduction in toughness observed in the defect-free HIP versions (197).

The competition between microstructure-dominated (e.g., Figure 7) and defect-dominated (e.g., Figures 6 and 8) contributions to toughness may be responsible for these apparently conflicting observations in which defect-free PBF (EBM) HIP Ti-6Al-4V samples exhibit lower toughness than do their defect-containing as-built counterparts. Figure 9 summarizes location-dependent toughness values along a tall, as-built PBF (EBM) Ti-6Al-4V sample, in addition to microstructure variations and significant differences in defect density along the build. Although HIP eliminates process-induced defects, location-dependent toughness values remain,

\begin{center}
\includegraphics[max width=\textwidth]{2024_03_11_add5292786349c475313g-074(1)}
\end{center}

\begin{center}
\includegraphics[max width=\textwidth]{2024_03_11_add5292786349c475313g-074}
\end{center}

\begin{center}
\includegraphics[max width=\textwidth]{2024_03_11_add5292786349c475313g-075(1)}
\end{center}

\href{http://www.annualreviews.org}{www.annualreviews.org} $} \cdot$ \text{Metal Additive Manufacturing}

\begin{center}
\includegraphics[max width=\textwidth]{2024_03_11_add5292786349c475313g-075}
\end{center}

Table 6 Summary of tensile properties of other alloys additively manufactured by DED

\begin{center}
\begin{tabular}{|c|c|c|c|c|c|c|c|c|c|}
\hline
\begin{tabular}{l}
Machine \\
type \\
\end{tabular} & Alloy & \begin{tabular}{c}
Condition \\
(as built, \\
HIP, or \\
heat \\
treated) \\
\end{tabular} & \begin{tabular}{c}
Specimen \\
orienta- \\
tion \\
\end{tabular} & $E(\mathrm{GPa})$ & \begin{tabular}{c}
Yield \\
strength \\
$(\mathrm{MPa})$ \\
\end{tabular} & \begin{tabular}{c}
Ultimate \\
tensile \\
strength \\
$(\mathrm{MPa})$ \\
\end{tabular} & \begin{tabular}{c}
Elongation \\
$(\%)$ \\
\end{tabular} & Hardness & Reference \\
\hline
\multirow{2}{*}{GTAW} & \multirow{2}{*}{TiAl} & \multirow{2}{*}{As built} & $Y$ & \multirow{2}{*}{NA} & $474 \pm 17$ & $549 \pm 23$ & 0.5 & \multirow[t]{2}{*}{NA} & \multirow[t]{2}{*}{185} \\
\hline
 &  &  & $Z$ &  & $424 \pm 30$ & $488 \pm 50$ & 0.5 &  &  \\
\hline
$\overline{\mathrm{DMD}}$ & 4340 & \begin{tabular}{l}
Stress \\
relieved \\
\end{tabular} & $X Y$ & NA & $1,398.65$ & NA & 1.665 & NA & 95 \\
\hline
\multirow{2}{*}{}\begin{tabular}{l}
DLD \\
$(\mathrm{LENS})$ \\
\end{tabular} & \multirow[t]{2}{*}{$316 \mathrm{~L}$} & As built & $Z$ & NA & $405-415$ & $620-660$ & $34-40$ & NA & \multirow[t]{2}{*}{225} \\
\hline
 &  & \begin{tabular}{l}
Heat \\
treated \\
\end{tabular} & $Z$ & NA & $325-355$ & $600-620$ & $42-43$ & NA &  \\
\hline
SMD & IN718 & As built & $X Y$ & NA & $473 \pm 6$ & $828 \pm 8$ & $28 \pm 2$ & NA & 186 \\
\hline
\multirow[t]{2}{*}{$\overline{\mathrm{DLD}}$} & \multirow[t]{2}{*}{IN718} & As built & \multirow[t]{2}{*}{$Z$} & \multirow[t]{2}{*}{NA} & 650 & 1,000 & \multirow[t]{2}{*}{NA} & \multirow[t]{2}{*}{$\mathrm{NA}$} & \multirow[t]{2}{*}{187} \\
\hline
 &  & \begin{tabular}{l}
Heat \\
treated \\
\end{tabular} &  &  & 1,257 & 1,436 &  &  &  \\
\hline
\multirow[t]{2}{*}{Laser} & \multirow[t]{2}{*}{IN718} & As built & \multirow[t]{2}{*}{NA} & \multirow[t]{2}{*}{NA} & 590 & 845 & 11 & \multirow[t]{2}{*}{NA} & \multirow[t]{2}{*}{188} \\
\hline
 &  & \begin{tabular}{l}
Heat \\
treated \\
\end{tabular} &  &  & 1,133 & 1,240 & 9 &  &  \\
\hline
$\mathrm{EBF}^{3}$ & IN718 & As built & $X Y$ & 159 & 580 & 910 & 22 & $\mathrm{NA}$ & 189 \\
\hline
\multirow[t]{4}{*}{$\overline{\mathrm{EBF}^{3}}$} & \multirow[t]{4}{*}{IN718} & \multirow[t]{2}{*}{As built} & $X Y$ & 138 & 655 & 978 & \multirow[t]{4}{*}{NA} & \multirow[t]{4}{*}{NA} & \multirow[t]{4}{*}{99} \\
\hline
 &  &  & $Y X$ & 194 & 699 & 936 &  &  &  \\
\hline
 &  & \multirow{2}{*}{}\begin{tabular}{l}
Heat \\
treated \\
\end{tabular} & $X Y$ & 174 & 986 & 1,114 &  &  &  \\
\hline
 &  &  & $Y X$ & 192 & 998 & 1,162 &  &  &  \\
\hline
$\overline{\text { DLD }}$ & IN718 & \begin{tabular}{l}
Heat \\
treated \\
\end{tabular} & $\mathrm{NA}$ & NA & $1,097.6$ & 1,321 & 9.8 & NA & 111 \\
\hline
$\overline{\text { DLD }}$ & IN718 & \begin{tabular}{l}
Heat \\
treated \\
\end{tabular} & NA & NA & 1,034 & 1,276 & 12 & NA & 190 \\
\hline
Laser/wire & IN718 & \begin{tabular}{l}
Heat \\
treated \\
\end{tabular} & $\mathrm{NA}$ & NA & 1,079 & 1,314 & 20.4 & NA & 191 \\
\hline
\multirow[t]{2}{*}{WAAM} & \multirow[t]{2}{*}{AA2319} & \multirow[t]{2}{*}{As built} & $X$ & \multirow[t]{2}{*}{NA} & $114 \pm 4.8$ & $263 \pm 0.5$ & $18 \pm 0.5$ & \multirow[t]{2}{*}{$\mathrm{NA}$} & 127 \\
\hline
 &  &  & $Y$ &  & $106 \pm 0.8$ & $258 \pm 2.2$ & $15.5 \pm 1$ &  &  \\
\hline
\end{tabular}
\end{center}

NA denotes data not available.

suggesting that subtle detrimental changes to the microstructure may be responsible. More work is clearly needed to resolve these issues, and testing of much thicker samples is necessary to obtain valid $K_{\text {Ic }}$ measurements.

The relatively high toughness values obtained for the as-built PBF (EBM) Ti-6Al-4V shown in Table 7 appear promising from a damage tolerance perspective. However, the presence of process-induced defects significantly reduces HCF properties, as discussed below.

Table 8 provides $K_{\mathrm{q}}$ toughness values for PBF $(\mathrm{EBM})$ Ti-Al $4822(198,199)$ in both the asbuilt and HIP conditions. Although only very limited data exist, the toughness values were similar to those previously obtained for as-cast Ti-Al 4822 (200). However, the scale and homogeneity of the microstructures of the as-built and HIP PBF (EBM) TiAl were very different from one another and from those of the as-cast TiAl (200). The presence of process-induced defects in the

\begin{center}
\includegraphics[max width=\textwidth]{2024_03_11_add5292786349c475313g-076}
\end{center}

Table 7 Summary of AM Ti-6Al-4V PBF (laser/EBM) fracture toughness

\begin{center}
\begin{tabular}{|c|c|c|c|c|c|c|}
\hline
\begin{tabular}{l}
Process \\
category \\
\end{tabular} & Machine type & Condition & \begin{tabular}{c}
Specimen \\
orienta- \\
tion \\
\end{tabular} & \begin{tabular}{c}
Specimen \\
type \\
\end{tabular} & $K_{\mathrm{q}}(\mathrm{MPa} / \mathrm{m})$ & Reference \\
\hline
\multirow[t]{16}{*}{PBF(laser)} & \multirow[t]{9}{*}{SLM} & \multirow[t]{3}{*}{As built} & $X Y$ & \multirow[t]{22}{*}{CT} & $28 \pm 2$ & \multirow[t]{8}{*}{151} \\
\hline
 &  &  & $X Z$ &  & $23 \pm 1$ &  \\
\hline
 &  &  & $Z X$ &  & $16 \pm 1$ &  \\
\hline
 &  & \multirow[t]{3}{*}{Stress relieved} & $X Y$ &  & $28 \pm 2$ &  \\
\hline
 &  &  & $X Z$ &  & $30 \pm 1$ &  \\
\hline
 &  &  & $Z X$ &  & $31 \pm 2$ &  \\
\hline
 &  & \multirow[t]{3}{*}{Heat treated} & $X Y$ &  & $41 \pm 2$ &  \\
\hline
 &  &  & $X Z$ &  & $49 \pm 2$ &  \\
\hline
 &  &  & $Z X$ &  & $49 \pm 1$ &  \\
\hline
 & \multirow[t]{3}{*}{SLM MTT250} & \multirow[t]{3}{*}{As built} & $X Y$ &  & $66.9 \pm 2.6$ & \multirow[t]{3}{*}{192} \\
\hline
 &  &  & $X Z$ &  & $64.8 \pm 16.9$ &  \\
\hline
 &  &  & $Y Z$ &  & $41.8 \pm 1.7$ &  \\
\hline
 & SLM & As built & $Z X$ &  & $52.4 \pm 3.48$ & 193 \\
\hline
 & \multirow[t]{3}{*}{EOS M280} & As built & \multirow[t]{3}{*}{$X Y$} &  & $37.5 \pm 5$ & \multirow[t]{3}{*}{194} \\
\hline
 &  & HIP &  &  & $57.8 \pm 5$ &  \\
\hline
 &  & Heat treated &  &  & 86.3 &  \\
\hline
\multirow[t]{11}{*}{PBF(EBM)} & \multirow[t]{2}{*}{Arcam A1} & \multirow[t]{2}{*}{As built} & $X Y$ &  & $110 \pm 8.9$ & \multirow[t]{2}{*}{142} \\
\hline
 &  &  & $Z X$ &  & $102 \pm 7.4$ &  \\
\hline
 & \multirow[t]{4}{*}{Arcam} & \multirow[t]{2}{*}{As built} & $X Y$ &  & 96.9 & \multirow[t]{4}{*}{195} \\
\hline
 &  &  & $Z X$ &  & 78.1 &  \\
\hline
 &  & \multirow[t]{2}{*}{HIP} & $X Y$ &  & 99.0 &  \\
\hline
 &  &  & $Z X$ &  & 83.1 &  \\
\hline
 & \multirow[t]{5}{*}{Arcam A2} & \multirow[t]{5}{*}{As built} & $X Y Z$ & \multirow[t]{5}{*}{$3 \mathrm{~PB}$} & 68,80 & \multirow[t]{3}{*}{48} \\
\hline
 &  &  & $X Z Y$ &  & 76 &  \\
\hline
 &  &  & \begin{tabular}{l}
$\mathrm{ZXY}:$ \\
middle \\
\end{tabular} &  & 65,66 &  \\
\hline
 &  &  & \begin{tabular}{r}
$Z X Y:$ \\
near \\
start \\
\end{tabular} &  & 79 & \multirow[t]{2}{*}{140} \\
\hline
 &  &  & \begin{tabular}{c}
$Z X Y:$ \\
near \\
end \\
\end{tabular} &  & 100 &  \\
\hline
\end{tabular}
\end{center}

as-built PBF (EBM) TiAl was confirmed by $\mu \mathrm{CT}$, and these defects were eliminated with HIP. This produced less scatter in the toughness data reported, again suggesting a competition between microstructure-dominated and defect-dominated contributions to toughness.

\section*{HIGH CYCLE FATIGUE AND FATIGUE CRACK GROWTH}
As discussed above, process-induced AM defects (e.g., Figures 6, 8, and 9) and microstructure variation/changes (e.g., Figures 7 and 9) can affect the tensile and toughness properties. However, such defects, along with surface roughness and residual stresses, can dominate the cyclic behavior,

\begin{center}
\includegraphics[max width=\textwidth]{2024_03_11_add5292786349c475313g-078(2)}
\end{center}

Figure 6

Lack-of-fusion (LoF) defects evident on the fracture surface of a PBF (EBM) as-built Ti-6Al-4V toughness sample tested in the LT-BOTH orientation shown in Figure 4. LoF defects are perpendicular to the build direction. (a) Low magnification. (b) Higher magnification.

can obscure microstructural effects, and can severely degrade the high cycle fatigue performance by providing potent fatigue initiation sites along with superimposed harmful residual stresses. These features can overwhelm any microstructural effects, as is shown below.

Although the early fatigue work of Kobryn \& Semiatin (58) on LENS-processed (i.e., DED) Ti6Al-4V exhibited HCF behavior that exceeded cast properties and was in the scatter band for cast + HIP and wrought-annealed Ti-6Al-4V (58), that work also revealed orientation-dependent fatigue life. More recent work (81) on LENS-processed Ti-6Al-4V documented defect-dominated fatigue behavior with fracture initiation from surface cracks and unmelted particles at the surface, as well as subsurface fatigue initiation from unmelted particles. Unmelted particles at the surface reduced the fatigue lifetime by an order of magnitude in comparison to subsurface crack initiation from unmelted particles in the bulk. However, when surface defects were suppressed by optimization\\
\includegraphics[max width=\textwidth, center]{2024_03_11_add5292786349c475313g-078(1)}

Figure 7

\begin{center}
\includegraphics[max width=\textwidth]{2024_03_11_add5292786349c475313g-078}
\end{center}

Large-area EBSD of an as-built PBF (EBM) Ti-6Al-4V sample showing crack growth across versus along reconstructed $\beta$ grains. Adapted with permission from Reference 10.

I4.I8 Lewandowski - Seifi

\begin{center}
\includegraphics[max width=\textwidth]{2024_03_11_add5292786349c475313g-079(1)}
\end{center}

Figure 8

$\mu$ CT images of a $10 \times 20 \times 100 \mathrm{~mm}$ as-built LT-BOTH PBF (EBM) Ti-6Al-4V toughness sample tested to failure in bending. Isolated defects (dark spots) are evident throughout the sample. Notch, fatigue precrack, cracked regions, and direct current potential drop lead holes are also shown. Build direction is out of page.

of the LENS process parameters (81), both as-deposited and simulated repair conditions could produce a fatigue life that exceeded the lower bound for wrought, annealed Ti-6Al-4V and that was in the upper-bound regions of cast + HIP material.

A very recent review (201) summarized the stress-controlled fatigue behavior of PBF (laser)processed, PBF (EBM)-processed, and DED-processed Ti-6Al-4V, along with the effects of surface roughness (e.g., as built versus machined) and defects (e.g., as built versus HIP) in comparison

\begin{center}
\includegraphics[max width=\textwidth]{2024_03_11_add5292786349c475313g-079}
\end{center}

\begin{center}
\includegraphics[max width=\textwidth]{2024_03_11_add5292786349c475313g-080}
\end{center}

$\overline{20} \mathrm{~mm}$\\
Microstructure variation along the build\\
\includegraphics[max width=\textwidth, center]{2024_03_11_add5292786349c475313g-080(2)}

Figure 9

Illustration of location-dependent toughness values in an as-built PBF (EBM) Ti-6Al-4V sample. Variations in microstructure (prior $\beta$ grains and $\alpha+\beta$ microstructure) and defect density were detected along the same as-built sample.

to as-cast and wrought Ti-6Al-4V tested with machined surfaces. Figure 10 includes data from that work (201) in addition to a recent similar study on PBF (EBM) Ti-6Al-4V (202) conducted under strain control to capture the essence of the observations to date (201).

Figure 10 shows summary data replotted (from Reference 201) of PBF (laser) $S-N$ fatigue behavior for Ti-6Al-4V tested at $R=0.1$, along with Metallic Materials Properties Development and Standardization (MMPDS) data obtained for cast (3-inch-thick) and wrought (annealed and aged) material with machined surfaces in addition to data obtained from Reference 202. Although orientation-dependent fatigue behavior was found and some property improvements were achieved with machined and polished surfaces, the very poor performance in comparison to the other data summarized in this plot was assumed to result from process-induced defects. Reference 201 indicates that the significantly improved fatigue data that were obtained by machining the asbuilt surfaces after optimization of the PBF (laser) process for Ti-6Al-4V support this hypothesis. However, this review (201) indicated that variations in laser process parameters created either a martensitic microstructure or a fine $\alpha$ microstructure. Fine $\alpha$ microstructure resulted in fatigue

Table 8 Summary of AM $\gamma$ Ti-Al 4822 fracture toughness for EBM PBF

\begin{center}
\includegraphics[max width=\textwidth]{2024_03_11_add5292786349c475313g-080(1)}
\end{center}

\begin{center}
\includegraphics[max width=\textwidth]{2024_03_11_add5292786349c475313g-081(1)}
\end{center}

\textbackslash author\{

\begin{itemize}
  \item Laser: HIP at $920^{\circ} \mathrm{C} / 2 \mathrm{~h} / 100 \mathrm{MPa}$, machined surface \\
 — Laser: no treatment, as-built surface \\
 L Laser: $3 \mathrm{~h}$ at $650^{\circ} \mathrm{C}$, as-built surface, $R=0.1$ \\
 - Laser: no treatment, machined surface \\
 - Laser: $4 \mathrm{~h}$ at $650^{\circ} \mathrm{C}$, machined surface \\
 - - - MMPDS 2010: 3-inch casting, machined surface, $R=0.1$ \\
 ....... MMPDS 2010: wrought, annealed, machined surface, $R=0.1$ \\
 --- MMPDS 2010: wrought, aged, machined surface, $R=0.1$ \\
 E-beam: no treatment, machined surface \\
 E-beam: no treatment, as-built surface \\
 E-beam: HIP at $920^{\circ} \mathrm{C} / 2 \mathrm{~h}$, machined surface \\
 Directed energy deposition, laser wire feed: heat treated and no treatment \\
 E-beam: optimized, no treatment, as-built surface finish, $R=0.1$ \\
 - E-beam: optimized, surface treated, $R=0.1$ \\
 E-beam: no treatment, machined surface, $R=0.1$\\
\}
\end{itemize}

Figure 10

Summary of stress $(S)$ versus cycles to failure $(N)(S-N)$ data for PBF (laser), PBF (EBM), and wire (DED) at $R=0.1$. Metallic Materials Properties Development and Standardization (MMPDS) data for cast, wrought machined data are shown for comparison. Data were obtained from a variety of sources, including References 12, 201, and 202. Adapted from Reference 201.

performance superior to that of MMPDS reference data for cast + HIP material, whereas the martensitic microstructure's performance was below that of the MMPDS reference data. Heat treatment at $650^{\circ} \mathrm{C} / 3 \mathrm{~h}$ marginally improved fatigue behavior with the as-built rough surfaces, and cracking again appeared to initiate from the (rough) surfaces. Heat treatment at $650^{\circ} \mathrm{C} / 4 \mathrm{~h}$ along with machined/EDM/shot-peened/sand-blasted surfaces produced fatigue performance that approached the fatigue performance of cast + HIP MMPDS data.

HIP of the $\mathrm{PBF}$ (laser) Ti-6Al- $4 \mathrm{~V}$ at $920^{\circ} \mathrm{C} / 2 \mathrm{~h} / 100 \mathrm{MPa}$, combined with surface machining, produced further improvements to the fatigue data in Figure 10 via elimination of process-induced defects (201). However, although HIP at $1,050^{\circ} \mathrm{C} / 4 \mathrm{~h} / 100 \mathrm{MPa}$ similarly eliminated processinduced defects, the associated microstructure coarsening at this HIP temperature reduced the fatigue performance. These results again emphasize the competition between defect-dominated

\includegraphics[max width=\textwidth, center]{2024_03_11_add5292786349c475313g-081}\\
and microstructure-dominated contributions to properties. Although the presence of processinduced defects dominates high cycle fatigue performance and obscure microstructural effects, the removal of these defects via HIP or process optimization requires optimization of microstructural features to continue to improve performance.

Figure 10 also includes data (from References 201 and 202) of PBF (EBM) and wire (DED) $S-N$ fatigue behavior for Ti-6Al-4V tested at $R=0.1$. The PBF (EBM)-processed samples tested with as-built (i.e., rough) surfaces reveals performance only slightly better than that of the worstperforming PBF (laser) samples shown. Some improvement in performance is provided by machining as-built surfaces (12), with more significant improvement reported recently on machined samples (202). However, the high cycle fatigue performance of the early work (12) on machined samples is well below that of the MMPDS cast Ti-6Al-4V data and is likely compromised by premature crack initiation at process-induced defects such as porosity. However, machining + $\mathrm{HIP}$ at $920^{\circ} \mathrm{C} / 2 \mathrm{~h} / 100 \mathrm{MPa}$ (12) produced results comparable to MMPDS wrought data, again consistent with the HIP elimination of process-induced defects.

Figure 10 also summarizes wire-fed DED fatigue performance. The good fatigue performance of DED-processed Ti-6Al-4V is attributed to the general lack of process-induced defects, whereas differences in the fatigue performance of the laser wire-fed and tungsten inert gas wire-fed Ti-6Al-4V were attributed to differences in microstructural scale (201).

Table 9 summarizes the limited fatigue crack growth data reported for PBF (laser) and PBF (EBM) Ti-6Al-4V. Despite the generally low toughness values of the PBF (laser) Ti-6Al-4V summarized in Table 7, Table 9 reveals Paris slope values at $R=0.1$ in the range of 3-6, which is typical for metallic materials, and overload $K_{\mathrm{c}}$ values in the range of the toughness values reported in Table 7 for PBF (laser) Ti-6Al-4V. The higher-toughness PBF (EBM) Ti-6Al-4V exhibits much higher $K_{\mathrm{c}}$ at overload in fatigue and similarly low Paris slope values and fatigue thresholds in the range of conventional Ti-6Al-4V. However, location- and orientation-dependent fatigue crack growth is evident in Table $\mathbf{9}$ and is likely affected by the competition between microstructuredependent and defect-dependent contributions to fatigue crack growth, which is somewhat similar to that proposed in Figure 9 for toughness.

\section*{CONCLUSIONS AND FUTURE RESEARCH PERSPECTIVE}
Figure 11 summarizes the range of mechanical properties typically generated in the mechanical characterization of metallic structural materials depending on their intended application. In that regard, this review summarizes published data currently available for AM metallic materials across the range of currently available PBF and DED process categories. Although the breadth of published mechanical properties has not covered the whole range of those shown in Figure 11, some of the mechanical properties reported for some of the metallic systems approach, and sometimes exceed, properties obtained on similar materials processed conventionally (e.g., casting, extrusion, forging). However, relatively few published data exist on standard samples, and little to no published work exists for low cycle fatigue, fatigue crack growth, fracture toughness, impact, creep, creep fatigue, multiaxial testing, or environmental effects. Furthermore, the current variability of properties (controlled by microstructure, residual stress, defects, etc.) within and between builds in one machine and across different machines and techniques, as well as the presence of process-induced defects and location/orientation-dependent properties, limits the more widespread use of these processing techniques for fracture-critical applications. The source(s), detection, and elimination of process-induced defects remain areas requiring additional focus to determine the microstructural features that will control properties with these processing techniques. These goals can be accomplished only by a better understanding of the fundamental

\begin{center}
\includegraphics[max width=\textwidth]{2024_03_11_add5292786349c475313g-083}
\end{center}

\begin{center}
\includegraphics[max width=\textwidth]{2024_03_11_add5292786349c475313g-084(1)}
\end{center}

Figure 11

The range of mechanical properties typically generated for structural materials. The specific properties of interest depend on the intended application. Abbreviations: LEFM, linear elastic fracture mechanics; SHPB, split-Hopkinson pressure bar.

\begin{center}
\includegraphics[max width=\textwidth]{2024_03_11_add5292786349c475313g-084}
\end{center}

processing-structure-property relationships possible with this emerging technology. A more complete review of the evolving processing-microstructure-property relationships is in progress (203).

One approach that has recently been proposed (10) and that is summarized in Figure 12 is to utilize Integrated Computation Materials (Science) and Engineering [ICM(S)E] to begin to address the multitude of issues that include development strategies for new alloys/microstructures

\begin{center}
\includegraphics[max width=\textwidth]{2024_03_11_add5292786349c475313g-085}
\end{center}

Figure 12

Integrated multiscale approach for the development of additively manufactured alloys for structural applications.

specifically designed to take advantage of AM as well as the lack of detailed process-structureproperty understanding both within and across different machines that do not provide open source access. Some of the various challenges that have been summarized in more detail elsewhere (10) include lack of computationally efficient tools, lack of in situ commercial monitoring systems, lack of material/testing standards, feedstock and recyclability/reusability issues, surface roughness and residual stress management/control, process feedback and control, postprocessing via alternate heat treatments and/or HIP conditions, big data generation and handling issues, and probabilistic modeling of fracture-critical properties. Addressing these challenges in a cost-effective manner will require the integration of fundamental and applied approaches by various science and engineering disciplines at academic, industrial, and government institutions.

\section*{DISCLOSURE STATEMENT}
The authors are not aware of any affiliations, memberships, funding, or financial holdings that might be perceived as affecting the objectivity of this review.

\section*{ACKNOWLEDGMENTS}
The time spent gathering the information and generating some of the data reported in this article was partly supported by America Makes, the National Additive Manufacturing Innovation Institute, under project 4009, "Rapid Qualification Methods for Powder Bed Direct Metal Additive Manufacturing Processes," through contract FA8650-12-2-7230, and such support is highly appreciated. Additional support was provided by two ASTM International scholarship awards (M. Seifi) and by the Arthur P. Armington Professorship (J.J. Lewandowski). Mechanical testing and analyses at CWRU were conducted in the Advanced Manufacturing and Mechanical Reliability Center. M. Seifi appreciates various discussions with ASTM F42/E08/E07 committee members as well as significant interactions with YXLON, a division of Comet technologies, and GE inspection technologies regarding $\mu \mathrm{CT}$ results. Authors also appreciate fruitful collaboration with Materials Resources LLC for microstructural quantification/analyses.

\section*{LITERATURE CITED}
\begin{enumerate}
  \item Kruth J-P, Leu MC, Nakagawa T. 1998. Progress in additive manufacturing and rapid prototyping. CIRP Ann. Manuf. Technol. 47(1):525-40

  \item Frazier WE. 2014. Metal additive manufacturing: a review. 7. Mater. Eng. Perform. 23(6):1917-28

  \item Dutta B, Froes FHS. 2014. Additive manufacturing of titanium alloys. Adv. Mater. Res. 1019(Oct.):19-25

  \item Facchini L, Magalini E, Robotti P, Molinari A. 2009. Microstructure and mechanical properties of Ti-6Al-4V produced by electron beam melting of pre-alloyed powders. Rapid Prototyp. F. 15(3):171-78

  \item Parthasarathy J, Starly B, Raman S, Christensen A. 2010. Mechanical evaluation of porous titanium (Ti6Al4V) structures with electron beam melting (EBM). 7. Mech. Behav. Biomed. Mater. 3(3):249-59

  \item Murr LE, Gaytan SM, Ramirez DA, Martinez E, Hernandez J, et al. 2012. Metal fabrication by additive manufacturing using laser and electron beam melting technologies. 7. Mater. Sci. Technol. 28(1):1-14

  \item Collins PC, Haden CV, Ghamarian I, Hayes BJ, Ales T, et al. 2014. Progress toward an integration of process-structure-property-performance models for 'three-dimensional (3-D) printing' of titanium alloys. FOM 66(7):1299-309

  \item Yu J, Rombouts M, Maes G, Motmans F. 2012. Material properties of Ti6Al4V parts produced by laser metal deposition. Phys. Procedia 39:416-24

  \item Leuders S, Thöne M, Riemer A, Niendorf T, Tröster T, et al. 2013. On the mechanical behaviour of titanium alloy TiAl6V4 manufactured by selective laser melting: fatigue resistance and crack growth performance. Int. F. Fatigue 48:300-7

  \item Seifi M, Salem A, Beuth J, Harrysson O, Lewandowski JJ. 2016. Overview of materials qualification need for metal additive manufacturing. FOM 68(3):747-64

  \item Collins PC, Brice DA, Samimi P, Ghamarian I, Fraser HL. 2016. Microstructural control of additively manufactured materials. Annu. Rev. Mater. Res. 46:in press

  \item Ackelid U, Svensson M. 2009. Additive manufacturing of dense metal parts by electron beam melting. In Proceedings of Materials Science and Technology Conference (MS\&T), pp. 2711-19. Novelty, OH: ASM Int.

  \item Chang Y, McLouth T, Pozuelo M, Chang C, Wooten J. 2015. The micro-mechanical behavior of electron beam melted Ti-6Al-4V alloy. In TMS Proceedings, pp. 211-18. Warrendale, PA/Hoboken, NJ: TMS/Wiley

  \item Christensen A, Kircher R, Lippincott A. 2007. Qualification of electron beam melted (EBM) Ti6Al4VELI for orthopaedic applications. In Proceedings from the Materials \& Processes for Medical Devices Conference, pp. 48-53. Novelty, OH: ASM Int.

  \item Devika D, Dass SS, Kumar Chaudhary S. 2015. Characterization and corrosion behaviour study on biocompatible Ti-6Al-4V component fabricated by electron beam melting. 7. Biomimetics Biomater. Biomed. Eng. 22:63-75

  \item Facchini L, Magalini E, Robotti P, Molinari A. 2009. Microstructure and mechanical properties of Ti-6Al-4V produced by electron beam melting of pre-alloyed powders. Rapid Prototyp. F. 15(3):171-78

  \item Gong H, Rafi K, Gu H, Starr T, Stucker B. 2014. Analysis of defect generation in Ti-6Al-4V parts made using powder bed fusion additive manufacturing processes. Addit. Manuf. 1-4:87-98

  \item Gong X, Anderson T, Chou K. 2012. Review on powder-based electron beam additive manufacturing technology. In Proceedings of the ASME International Symposium on Flexible Automation, pp. 507-15. New York: ASME

  \item Gong X, Lydon J, Cooper K, Chou K. 2013. Microstructural characterization and modeling of beam speed effects on Ti-6Al-4V by electron beam additive manufacturing. In Solid Freeform Fabrication Proceedings, pp. 459-69. Austin: Univ. Tex.

  \item Gong X, Lydon J, Cooper K, Chou K. 2014. Beam speed effects on Ti-6Al-4V microstructures in electron beam additive manufacturing. 7. Mater. Res. 29(17):1-9

  \item Guo C, Ge W, Lin F. 2015. Effects of scanning parameters on material deposition during electron beam selective melting of Ti-6Al-4V powder. 7. Mater. Process. Technol. 217:148-57

  \item Hrabe N, Quinn T. 2013. Effects of processing on microstructure and mechanical properties of a titanium alloy (Ti-6Al-4V) fabricated using electron beam melting (EBM). Part 1. Distance from build plate and part size. Mater. Sci. Eng. A 573:264-70

  \item Ikeo N, Ishimoto T, Serizawa A, Nakano T. 2014. Control of mechanical properties of three-dimensional Ti-6Al-4V products fabricated by electron beam melting with unidirectional elongated pores. Metall. Mater. Trans. A 45(10):4293-301

  \item Jamshidinia M, Kovacevic R. 2015. The influence of heat accumulation on the surface roughness in powder-bed additive manufacturing. Surf. Topogr. Metrol. Prop. 3(1):014003

  \item Juechter V, Scharowsky T, Singer RF, Körner C. 2014. Processing window and evaporation phenomena for Ti-6Al-4V produced by selective electron beam melting. Acta Mater. 76:252-58

  \item Kalinyuk AN, Trigub NP, Semiatin SL. 2003. Microstructure, texture, and mechanical properties of electron-beam melted Ti-6Al-4V. Mater. Sci. Eng. A 346:178-88

  \item Karlsson J, Norell M, Ackelid U, Engqvist H, Lausmaa J. 2015. Surface oxidation behavior of Ti-6Al-4V manufactured by Electron Beam Melting (EBM $\left.{ }^{\circledR}\right)$. F. Manuf. Process. 17:120-26

  \item Koike M, Greer P, Owen K, Lilly G, Murr LE, et al. 2011. Evaluation of titanium alloys fabricated using rapid prototyping technologies-electron beam melting and laser beam melting. Materials 4(12):1776-92

  \item Koike M, Martinez K, Guo L, Chahine G, Kovacevic R, Okabe T. 2011. Evaluation of titanium alloy fabricated using electron beam melting system for dental applications. F. Mater. Process. Technol. 211(8):1400-8

  \item Kok Y, Tan X, Tor S, Chua CK. 2015. Fabrication and microstructural characterisation of additive manufactured Ti-6Al-4V parts by electron beam melting. Virtual Phys. Prototyp. 10(1):13-21

  \item Lu SL, Tang HP, Ning YP, Liu N, StJohn DH, Qian M. 2015. Microstructure and mechanical properties of long Ti-6Al-4V rods additively manufactured by selective electron beam melting out of a deep powder bed and the effect of subsequent hot isostatic pressing. Metall. Mater. Trans. A 46(9):3824-34

  \item Markl M, Ammer R, Rüde U, Körner C. 2015. Numerical investigations on hatching process strategies for powder-bed-based additive manufacturing using an electron beam. Int. F. Adv. Manuf. Technol. 78(14):239-47

  \item Mohammadhosseini A, Fraser D, Masood SH, Jahedi M. 2013. Microstructure and mechanical properties of Ti-6Al-4V manufactured by electron beam melting process. Mater. Res. Innov. 17(Suppl. 2):106-12

  \item Murr LE. 2014. Metallurgy of additive manufacturing: examples from electron beam melting. Addit. Manuf. 5:40-53

  \item Murr LE, Gaytan SM, Lopez MI, Martinez E, Medina F, Wicker RB. 2009. Metallographic characterization of additive-layer manufactured products by electron beam melting of Ti-6Al-4V powder. Pract. Metallogr. 46(9):442-53

  \item Hrabe N, Kircher R, Quinn T. 2012. Effects of processing on microstructure and mechanical properties of Ti-6Al-4V fabricated using electron beam melting (EBM): orientation and location. In Solid Freeform Fabrication Proceedings, pp. 1045-58. Austin: Univ. Tex.

  \item Ponader S, Vairaktaris E, Heinl P, Wilmowsky CV, Rottmair A, et al. 2008. Effects of topographical surface modifications of electron beam melted Ti-6Al-4V titanium on human fetal osteoblasts. 7. Biomed. Mater. Res. A 84(4):1111-19

  \item Puebla K. 2012. Effect of melt scan rate on microstructure and macrostructure for electron beam melting of Ti-6Al-4V. Mater. Sci. Appl. 3(5):259-64

  \item Rafi HK, Karthik NV, Gong H, Starr TL, Stucker BE. 2013. Microstructures and mechanical properties of Ti6Al4V parts fabricated by selective laser melting and electron beam melting. 7. Mater. Eng. Perform. 22(12):3872-83

  \item Rafi K, Karthik N, Starr TL, Stucker BE. 2012. Mechanical property evaluation of Ti-6Al-4V parts made using electron beam melting. In Solid Freeform Fabrication Proceedings, pp. 526-35. Austin: Univ. Tex.

  \item Safdar A, He HZ, Wei L-Y, Snis A, De Paz LEC. 2012. Effect of process parameters settings and thickness on surface roughness of EBM produced Ti-6Al-4V. Rapid Prototyp. F. 18(5):401-8

  \item Safdar A, Wei L-Y, Snis A, Lai Z. 2012. Evaluation of microstructural development in electron beam melted Ti-6Al-4V. Mater. Charact. 65:8-15

  \item Scharowsky T, Juechter V, Singer RF, Körner C. 2015. Influence of the scanning strategy on the microstructure and mechanical properties in selective electron beam melting of Ti-6Al-4V. Adv. Eng. Mater. 17(11):1573-78

  \item Svensson M. 2013. Influence of interstitials on material properties of Ti-6Al-4V fabricated with Electron Beam Melting $\left(\mathrm{EBM}^{\circledR}\right)$. In Proceedings from the Materials and Processes for Medical Devices Conference, pp. 119-24. Novelty, OH: ASM Int.

  \item Svensson M, Ackelid U, Ab A. 2010. Titanium alloys manufactured with electron beam melting mechanical and chemical properties. In Proceedings of Materials \& Processes for Medical Devices Conference, pp. 189-94. Novelty, OH: ASM Int.

  \item Wang X, Gong X, Chou K. 2015. Scanning speed effect on mechanical properties of Ti-6Al-4V alloy processed by electron beam additive manufacturing. Procedia Manuf. 1:287-95

  \item Zhao H, Antonysamy AA, Meyer J, Ciuea O, Williams ST, Prangnell PB. 2015. Automated multi-scale microstructure heterogeneity analysis of selective electron beam melted Ti-6Al-4V components. In TMS Proceedings, pp. 429-36. Warrendale, PA/Hoboken, NJ: TMS/Wiley

  \item Seifi M, Dahar M, Aman R, Harrysson O, Beuth J, Lewandowski JJ. 2015. Evaluation of orientation dependence of fracture toughness and fatigue crack propagation behavior of as-deposited ARCAM EBM Ti-6Al-4V. 7OM 67(3):597-607

  \item Behrendt U, Shellabear M. 1995. The EOS rapid prototyping concept. Comput. Ind. 28(1):57-61

  \item Challis VJ, Xu X, Zhang LC, Roberts AP, Grotowski JF, Sercombe TB. 2014. High specific strength and stiffness structures produced using selective laser melting. Mater. Des. 63:783-88

  \item Facchini L, Magalini E, Robotti P, Molinari A, Höges S, Wissenbach K. 2010. Ductility of a Ti-6Al-4V alloy produced by selective laser melting of prealloyed powders. Rapid Prototyp. F. 16(6):450-59

  \item Fu CH, Guo YB. 2014. Three-dimensional temperature gradient mechanism in selective laser melting of Ti-6Al-4V. F. Manuf. Sci. Eng. 136(6):06 1004

  \item Gong H, Gu H, Dilip JJS, Pal D, Stucker B, et al. 2014. Melt pool characterization for selective laser melting of Ti-6Al-4V pre-alloyed powder. In Solid Freeform Fabrication Proceedings. Austin: Univ. Tex.

  \item Grimm T, Wiora G, Witt G. 2015. Characterization of typical surface effects in additive manufacturing with confocal microscopy. Surf. Topogr. Metrol. Prop. 3(1):014001

  \item Hanzl P, Zetek M, Bakša T, Kroupa T. 2015. The influence of processing parameters on the mechanical properties of SLM parts. Procedia Eng. 100:1405-13

\end{enumerate}

\begin{center}
\includegraphics[max width=\textwidth]{2024_03_11_add5292786349c475313g-088}
\end{center}

\begin{enumerate}
  \setcounter{enumi}{55}
  \item Khaing MW, Fuh JYH, Lu L. 2001. Direct metal laser sintering for rapid tooling: processing and characterisation of EOS parts. 7. Mater. Process. Technol. 113(1-3):269-72

  \item Kobryn PA, Semiatin SL. 2001. The laser additive manufacture of Ti-6Al-4V. FOM 53(9):40-42

  \item Kobryn PA, Semiatin SL. 2001. Mechanical properties of laser-deposited Ti-6Al-4V. In Solid Freeform Fabrication Proceedings, pp. 179-86. Austin: Univ. Tex.

  \item Kobryn P, Semiatin S. 2003. Microstructure and texture evolution during solidification processing of Ti-6Al-4V. 7. Mater. Process. Technol. 135(2-3):330-39

  \item Schnitzer M, Lisý M, Hudák R, Živ J. 2015. Experimental measuring of the roughness of test samples made using DMLS technology from the titanium alloy Ti-6Al-4V. In IEEE International Symposium on Applied Machine Intelligence and Informatics, 13th, pp. 31-36

  \item Simchi A. 2006. Direct laser sintering of metal powders: mechanism, kinetics and microstructural features. Mater. Sci. Eng. A 428(1-2):148-58

  \item Simchi A, Petzoldt F, Pohl H. 2003. On the development of direct metal laser sintering for rapid tooling. 7. Mater. Process. Technol. 141(3):319-28

  \item Simonelli M, Tse YY, Tuck C. 2012. Further understanding of Ti-6Al-4V selective laser melting using texture analysis. In Solid Freeform Fabrication Proceedings, pp. 480-91. Austin: Univ. Tex.

  \item Simonelli M, Tse YY, Tuck C. 2014. The formation of $\alpha+\beta$ microstructure in as-fabricated selective laser melting of Ti-6Al-4V. 7. Mater. Res. 29(17):2028-35

  \item Thijs L, Verhaeghe F, Craeghs T, Van Humbeeck J, Kruth J-P. 2010. A study of the microstructural evolution during selective laser melting of Ti-6Al-4V. Acta Mater. 58(9):3303-12

  \item Thombansen U, Abels P. 2015. Process observation in selective laser melting (SLM). Proc. SPIE 9356:93560R

  \item Wauthle R, Vrancken B, Beynaerts B, Jorissen K, Schrooten J, et al. 2015. Effects of build orientation and heat treatment on the microstructure and mechanical properties of selective laser melted Ti6Al4V lattice structures. Addit. Manuf. 5:77-84

  \item Wu X, Sharman R, Mei J, Voice W. 2004. Microstructure and properties of a laser fabricated burnresistant Ti alloy. Mater. Des. 25(2):103-9

  \item Wu X, Liang J, Mei J, Mitchell C, Goodwin PS, Voice W. 2004. Microstructures of laser-deposited Ti-6Al-4V. Mater. Des. 25(2):137-44

  \item Xu W, Brandt M, Sun S, Elambasseril J, Liu Q, et al. 2015. Additive manufacturing of strong and ductile Ti-6Al-4V by selective laser melting via in situ martensite decomposition. Acta Mater. 85:74-84

  \item 
  \begin{enumerate}
    \setcounter{enumii}{2000}
    \item EOS takes fine approach to laser sintering. Met. Powder Rep. 56(3):18
  \end{enumerate}
  \item Qiu C, Ravi GA, Dance C, Ranson A, Dilworth S, Attallah MM. 2015. Fabrication of large Ti-6Al-4V structures by direct laser deposition. F. Alloys Compd. 629:351-61

  \item Hofmann DC, Roberts S, Otis R, Kolodziejska J, Dillon RP, et al. 2014. Developing gradient metal alloys through radial deposition additive manufacturing. Sci. Rep. 4:5357

  \item Wang F, Williams S, Colegrove P, Antonysamy AA. 2012. Microstructure and mechanical properties of wire and arc additive manufactured Ti-6Al-4V. Metall. Mater. Trans. A 44(2):968-77

  \item Brandl E, Baufeld B, Leyens C, Gault R. 2010. Additive manufactured Ti-6A1-4V using welding wire: comparison of laser and arc beam deposition and evaluation with respect to aerospace material specifications. Phys. Procedia 5:595-606

  \item Qian L, Mei J, Liang J, Wu X. 2005. Influence of position and laser power on thermal history and microstructure of direct laser fabricated Ti-6Al-4V samples. Mater. Sci. Technol. 21(5):597-605

  \item Kobryn PA, Moore EH, Semiatin SL. 2000. Effect of laser power and traverse speed on microstructure, porosity, and build height in laser-deposited Ti-6Al-4V. Scr. Mater. 43(4):299-305

  \item Brandl E, Leyens C, Palm F. 2011. Mechanical properties of additive manufactured Ti-6Al-4V using wire and powder based processes. IOP Conf. Ser. Mater. Sci. Eng. 26:012004

  \item Kottman M, Zhang S, McGuffin-Cawley J, Denney P, Narayanan BK. 2015. Laser hot wire process: a novel process for near-net shape fabrication for high-throughput applications. FOM 67(3):622-28

  \item Carroll BE, Palmer TA, Beese AM. 2015. Anisotropic tensile behavior of Ti-6Al-4V components fabricated with directed energy deposition additive manufacturing. Acta Mater. 87:309-20

  \item Prabhu AW, Chaudhary A, Zhang W, Babu SS. 2015. Effect of microstructure and defects on fatigue behaviour of directed energy deposited Ti-6Al-4V. Sci. Technol. Weld. Foin. 20(8):659-69

  \item Kelly SM, Kampe SL. 2004. Microstructural evolution in laser-deposited multilayer Ti6Al-4V build. Part I. Microstructural characterization. Metall. Mater. Trans. A 35(June):1861-67

  \item Kelly SM, Kampe SL. 2004. Microstructural evolution in laser-deposited multilayer Ti-6Al-4V builds: Part II. Thermal modeling. Metall. Mater. Trans. A 35(6):1869-79

\end{enumerate}

\begin{center}
\includegraphics[max width=\textwidth]{2024_03_11_add5292786349c475313g-089}
\end{center}

\begin{enumerate}
  \setcounter{enumi}{83}
  \item Buican GR, Oancea G, Lancea C, Pop MA. 2015. Some considerations regarding micro hardness of parts manufactured from 316-L Steel using SLM technology. Appl. Mech. Mater. 760:515-20

  \item Zhao X, Wei Q, Song B, Liu Y, Luo X, et al. 2015. Fabrication and characterization of AISI 420 stainless steel using selective laser melting. Mater. Manuf. Process. 30(11):1283-89

  \item Jelis E, Clemente M, Kerwien S, Ravindra NM, Hespos MR. 2015. Metallurgical and mechanical evaluation of 4340 steel produced by direct metal laser sintering. FOM 67(3):582-89

  \item King WE, Barth HD, Castillo VM, Gallegos GF, Gibbs JW, et al. 2014. Observation of keyhole-mode laser melting in laser powder-bed fusion additive manufacturing. F. Mater. Process. Technol. 214(12):291525

  \item Jägle EA, Choi P. 2014. Precipitation and austenite reversion behavior of a maraging steel produced by selective laser melting. 7. Mater. Res. 29(17):2072-79

  \item Kempen K, Vrancken B, Buls S, Thijs L, Van Humbeeck J, Kruth J-P. 2014. Selective laser melting of crack-free high density M2 high speed steel parts by baseplate preheating. 7. Manuf. Sci. Eng. 136(6):061026

  \item Lebrun T, Tanigaki K, Horikawa K, Kobayashi H. 2014. Strain rate sensitivity and mechanical anisotropy of selective laser melted 17-4 PH stainless steel. Mech. Eng. F. 1(5):SMM0049

  \item Abele E, Stoffregen HA, Kniepkamp M, Lang S, Hampe M. 2014. Selective laser melting for manufacturing of thin-walled porous elements. 7. Mater. Process. Technol. 215:114-22

  \item Riemer A, Leuders S, Thöne M, Richard HA, Tröster T, Niendorf T. 2014. On the fatigue crack growth behavior in 316L stainless steel manufactured by selective laser melting. Eng. Fract. Mech. 120:15-25

  \item Tolosa I, Garciandía F, Zubiri F, Zapirain F, Esnaola A. 2010. Study of mechanical properties of AISI 316 stainless steel processed by 'selective laser melting', following different manufacturing strategies. Int. 7. Adv. Manuf. Technol. 51(5-8):639-47

  \item Wanjara P, Brochu M, Jahazi M. 2007. Electron beam freeforming of stainless steel using solid wire feed. Mater. Des. 28(8):2278-86

  \item Sun G, Zhou R, Lu J, Mazumder J. 2015. Evaluation of defect density, microstructure, residual stress, elastic modulus, hardness and strength of laser-deposited AISI 4340 steel. Acta Mater. 84:172-89

  \item Colegrove PA, Coules HE, Fairman J, Martina F, Kashoob T, et al. 2013. Microstructure and residual stress improvement in wire and arc additively manufactured parts through high-pressure rolling. 7. Mater. Process. Technol. 213(10):1782-91

  \item You X, Tan Y, Li J, Li P, Dong C, et al. 2015. Effects of solution heat treatment on the microstructure and hardness of Inconel 740 superalloy prepared by electron beam smelting. F. Alloys Compd. 638:239-48

  \item List FA, Dehoff RR, Lowe LE, Sames WJ. 2014. Properties of Inconel 625 mesh structures grown by electron beam additive manufacturing. Mater. Sci. Eng. A 615:191-97

  \item Tayon WA, Shenoy RN, Redding MR, Bird RK, Hafley RA. 2014. Correlation between microstructure and mechanical properties in an Inconel 718 deposit produced via electron beam freeform fabrication. 7. Manuf. Sci. Eng. 136(6):061005

  \item Martinez E, Murr LE, Hernandez J, Pan X, Amato K, et al. 2013. Microstructures of niobium components fabricated by electron beam melting. Metallogr. Microstruct. Anal. 2(3):183-89

  \item Murr LE, Martinez E, Pan XM, Gaytan SM, Castro JA, et al. 2013. Microstructures of Rene 142 nickelbased superalloy fabricated by electron beam melting. Acta Mater. 61(11):4289-96

  \item Murr LE, Martinez E, Gaytan SM, Ramirez DA, Machado BI, et al. 2011. Microstructural architecture, microstructures, and mechanical properties for a nickel-base superalloy fabricated by electron beam melting. Metall. Mater. Trans. A 42(11):3491-508

  \item Li S, Wei Q, Shi Y, Zhu Z, Zhang D. 2015. Microstructure characteristics of Inconel 625 superalloy manufactured by selective laser melting. 7. Mater. Sci. Technol. 31(9):946-52

  \item Strößner J, Terock M, Glatzel U. 2015. Mechanical and microstructural investigation of nickel-based superalloy IN718 manufactured by selective laser melting (SLM). Adv. Eng. Mater. 17(8):1099-105

\end{enumerate}

\begin{center}
\includegraphics[max width=\textwidth]{2024_03_11_add5292786349c475313g-090}
\end{center}

\begin{enumerate}
  \setcounter{enumi}{104}
  \item Scott-Emuakpor O, Schwartz J, George T, Holycross C, Cross C, Slater J. 2015. Bending fatigue life characterisation of direct metal laser sintering nickel alloy 718. Fatigue Fract. Eng. Mater. Struct. 38(9):1105-17

  \item Harrison NJ, Todd I, Mumtaz K. 2015. Reduction of micro-cracking in nickel superalloys processed by Selective Laser Melting: a fundamental alloy design approach. Acta Mater. 94:59-68

\end{enumerate}

Lewandowski $\cdot$ Seifi

\begin{enumerate}
  \setcounter{enumi}{106}
  \item Kunze K, Etter T, Grässlin J, Shklover V. 2015. Texture, anisotropy in microstructure and mechanical properties of IN738LC alloy processed by selective laser melting (SLM). Mater. Sci. Eng. A 620:213-22

  \item Carter LN, Martin C, Withers PJ, Attallah MM. 2014. The influence of the laser scan strategy on grain structure and cracking behaviour in SLM powder-bed fabricated nickel superalloy. 7. Alloys Compd. 615:338-47

  \item Kanagarajah P, Brenne F, Niendorf T, Maier HJ. 2013. Inconel 939 processed by selective laser melting: effect of microstructure and temperature on the mechanical properties under static and cyclic loading. Mater. Sci. Eng. A 588:188-95

  \item Benn RC, Salva RP, Engineering P. 2010. Additively manufactured Inconel alloy 718. In TMS Superalloy 718 and Derivatives Proceedings, 7th, pp. 455-69. Warrendale, PA/Hoboken, NJ: TMS/Wiley

  \item Zhang Y-N, Cao X, Wanjara P, Medraj M. 2014. Tensile properties of laser additive manufactured Inconel 718 using filler wire. 7. Mater. Res. 29(17):2006-20

  \item Dehoff RR, Sarosi PM, Collins PC, Fraser HL, Mills MJ. 2003. Microstructural evaluation of LENSTM deposited Nb-Ti-Si-Cr alloys. MRS Proc. 753:BB2.6

  \item Gu D, Wang H, Dai D, Yuan P, Meiners W, Poprawe R. 2015. Rapid fabrication of Al-based bulk-form nanocomposites with novel reinforcement and enhanced performance by selective laser melting. Scr. Mater. 96:25-28

  \item Weingarten C, Buchbinder D, Pirch N, Meiners W, Wissenbach K, Poprawe R. 2015. Formation and reduction of hydrogen porosity during selective laser melting of AlSi10Mg. F. Mater. Process. Technol. 221:112-20

  \item Siddique S, Imran M, Wycisk E, Emmelmann C, Walther F. 2015. Influence of process-induced microstructure and imperfections on mechanical properties of AlSi12 processed by selective laser melting. 7. Mater. Process. Technol. 221:205-13

  \item Yan C, Hao L, Hussein A, Young P, Huang J, Zhu W. 2015. Microstructure and mechanical properties of aluminium alloy cellular lattice structures manufactured by direct metal laser sintering. Mater. Sci. Eng. A 628:238-46

  \item Olakanmi EO, Cochrane RF, Dalgarno KW. 2015. A review on selective laser sintering/melting (SLS/SLM) of aluminium alloy powders: processing, microstructure, and properties. Prog. Mater. Sci. 74:401-77

  \item Kempen K, Thijs L, Van Humbeeck J, Kruth J-P. 2015. Processing AlSi10Mg by selective laser melting: parameter optimisation and material characterisation. Mater. Sci. Technol. 31(8):917-23

  \item Krishnan M, Atzeni E, Canali R, Calignano F, Manfredi D, et al. 2014. On the effect of process parameters on properties of AlSi10Mg parts produced by DMLS. Rapid Prototyp. 7. 20(6):449-58

  \item Rometsch PA, Zhong H, Nairn KM, Jarvis T, Wu X. 2014. Characterization of a laser-fabricated hypereutectic Al-Sc alloy bar. Scr. Mater. 87:13-16

  \item Li Y, Gu D. 2014. Parametric analysis of thermal behavior during selective laser melting additive manufacturing of aluminum alloy powder. Mater. Des. 63:856-67

  \item Mertens R, Clijsters S, Kempen K, Kruth J-P. 2014. Optimization of scan strategies in selective laser melting of aluminum parts with downfacing areas. 7. Manuf. Sci. Eng. 136(6):06 1012

  \item Aboulkhair NT, Everitt NM, Ashcroft I, Tuck C. 2014. Reducing porosity in AlSi10Mg parts processed by selective laser melting. Addit. Manuf. 1-4:77-86

  \item Rosenthal I, Stern A, Frage N. 2014. Microstructure and mechanical properties of AlSi10Mg parts produced by the laser beam additive manufacturing (AM) technology. Metallogr. Microstruct. Anal. 3(6):44853

  \item Taminger K, Hafley R. 2003. Electron beam freeform fabrication: a rapid metal deposition process. In Proc. Annu. Automot. Compos. Conf., 3rd, pp. 9-10

  \item Li X, Reynolds AP, Cong B, Ding J, Williams S. 2015. Production and properties of a wire-arc additive manufacturing part made with friction extruded wire. In TMS Proceedings, pp. 445-52. Warrendale, PA/Hoboken, NJ: TMS/Wiley

  \item Gu J, Cong B, Ding J, Williams SW, Zhai Y. 2014. Wire+arc additive manufacturing of aluminum. In Solid Freeform Fabrication Proceedings, pp. 451-58. Austin: Univ. Tex.

\end{enumerate}

\begin{center}
\includegraphics[max width=\textwidth]{2024_03_11_add5292786349c475313g-091}
\end{center}

\begin{center}
\includegraphics[max width=\textwidth]{2024_03_11_add5292786349c475313g-092}
\end{center}

\begin{enumerate}
  \setcounter{enumi}{127}
  \item Fujieda T, Shiratori H, Kuwabara K, Kato T, Yamanaka K, et al. 2015. First demonstration of promising selective electron beam melting method for utilizing high-entropy alloys as engineering materials. Mater. Lett. 159(15):12-15

  \item Brif Y, Thomas M, Todd I. 2015. The use of high-entropy alloys in additive manufacturing. Scr. Mater. 99:93-96

  \item ISO/ASTM. 2013. Standard terminology for additive manufacturing-coordinate systems and test methodologies. ASTM/ISO Stand. 52921

  \item ASTM. 2015. Anisotropy effects in mechanical properties of AM parts. ASTM Work Item WK49229

  \item Moylan S, Land J, Possolo A. 2015. Additive manufacturing round robin protocols: a pilot study. In Solid Freeform Fabrication Proceedings, pp. 1504-12. Austin: Univ. Tex.

  \item Moylan S, Slotwinski J. 2014. Assessment of guidelines for conducting round robin studies in additive manufacturing. In Proceedings of ASPE Spring Topical Meeting—Dimensional Accuracy and Surface Finish in Additive Manufacturing, Vol. 57, pp. 82-85. Berkeley, CA: NIST

  \item Gockel J, Beuth J, Taminger K. 2014. Integrated control of solidification microstructure and melt pool dimensions in electron beam wire feed additive manufacturing of Ti-6Al-4V. Addit. Manuf. 1-4:119-26

  \item Beuth J, Fox J, Gockel J, Montgomery C, Yang R, et al. 2013. Process mapping for qualification across multiple direct metal additive manufacturing processes. In Solid Freeform Fabrication Proceedings, pp. $655-$ 65. Austin: Univ. Tex.

  \item Gockel J, Beuth J. 2013. Understanding Ti-6Al-4V microstructure control in additive manufacturing via process maps. In Solid Freeform Fabrication Proceedings, pp. 666-74. Austin: Univ. Tex.

  \item Nassar AR, Keist JS, Reutzel EW, Spurgeon TJ. 2015. Intra-layer closed-loop control of build plan during directed energy additive manufacturing of Ti-6Al-4V. Addit. Manuf. 6:39-52

  \item Soylemez E, Beuth JL, Taminger K. 2010. Controlling melt pool dimensions over a wide range of material deposition rates in electron beam additive manufacturing. In Solid Freeform Fabrication Proceedings, pp. 571-82. Austin: Univ. Tex.

  \item Montgomery C, Beuth J, Sheridan L, Klingbeil N. 2015. Process mapping of Inconel 625 in laser powder bed additive manufacturing. In Solid Freeform Fabrication Proceedings, pp. 1195-204. Austin: Univ. Tex.

  \item Seifi M, Christiansen D, Beuth JL, Harrysson O, Lewandowski JJ. 2016. Process mapping, fracture and fatigue behavior of Ti-6Al-4V produced by EBM additive manufacturing. In Proceedings of World Conference on Titanium, 13th, pp. 1373-77. Warrendale, PA/Hoboken, NJ: TMS/Wiley

  \item Greitemeier D, Dalle Donne C, Syassen F, Eufinger J, Melz T. 2016. Effect of surface roughness on fatigue performance of additive manufactured Ti-6Al-4V. Mater. Sci. Technol. In press

  \item Edwards P, O'Conner A, Ramulu M. 2013. Electron beam additive manufacturing of titanium components: properties and performance. F. Manuf. Sci. Eng. 135(6):061016

  \item Tan X, Kok Y, Tan YJ, Descoins M, Mangelinck D, et al. 2015. Graded microstructure and mechanical properties of additive manufactured Ti-6Al-4V via electron beam melting. Acta Mater. 97:1-16

  \item Rodriguez OL, Allison PG, Whittington WR, Francis DK, Rivera OG, et al. 2015. Dynamic tensile behavior of electron beam additive manufactured Ti6Al4V. Mater. Sci. Eng. A 641:323-27

  \item Hrabe N, Quinn T. 2013. Effects of processing on microstructure and mechanical properties of a titanium alloy (Ti-6Al-4V) fabricated using electron beam melting (EBM). Part 2. Energy input, orientation, and location. Mater. Sci. Eng. A 573:271-77

  \item Murr LE, Esquivel EV, Quinones SA, Gaytan SM, Lopez MI, et al. 2009. Microstructures and mechanical properties of electron beam-rapid manufactured Ti-6Al-4V biomedical prototypes compared to wrought Ti-6Al-4V. Mater. Charact. 60(2):96-105

  \item McLouth T, Chang Y, Wooten J, Yang J. 2015. The effects of electron beam melting on the microstructure and mechanical properties of Ti-6Al-4V and $\gamma$-TiAl. Microsc. Microanal. 21(588):1177-78

  \item Gong H, Rafi K, Gu H, Janaki Ram GD, Starr T, Stucker B. 2015. Influence of defects on mechanical properties of Ti-6Al-4V components produced by selective laser melting and electron beam melting. Mater. Des. 86:545-54

  \item Morgan L, Lindhe U, Harrysson O. 2003. Rapid manufacturing with electron beam melting (EBM)-a manufacturing revolution? In Solid Freeform Fabrication Proceedings, pp. 433-38. Austin: Univ. Tex.

  \item Rekedal KD, Liu D. 2015. Fatigue life of selective laser melted and hot isostatically pressed Ti-6Al-4V absent of surface machining. Presented at AIAA/ASCE/AHS/ASC Structures, Structural Dynamics, and Materials Conference, 56 th

  \item Cain V, Thijs L, Van Humbeeck J, Van Hooreweder B, Knutsen R. 2015. Crack propagation and fracture toughness of Ti6Al4V alloy produced by selective laser melting. Addit. Manuf. 5(1):68-76

  \item Kasperovich G, Hausmann J. 2015. Improvement of fatigue resistance and ductility of TiAl6V4 processed by selective laser melting. 7. Mater. Process. Technol. 220:202-14

  \item Edwards P, Ramulu M. 2014. Fatigue performance evaluation of selective laser melted Ti-6Al-4V. Mater. Sci. Eng. A 598:327-37

  \item Leuders S, Lieneke T, Lammers S, Tröster T, Niendorf T. 2014. On the fatigue properties of metals manufactured by selective laser melting-the role of ductility. F. Mater. Res. 29(17):1911-19

  \item Simonelli M, Tse YY, Tuck C. 2014. Effect of the build orientation on the mechanical properties and fracture modes of SLM Ti-6Al-4V. Mater. Sci. Eng. A 616(10):1-11

  \item Vrancken B, Thijs L, Kruth J-P, Van Humbeeck J. 2012. Heat treatment of Ti6Al4V produced by Selective Laser Melting: microstructure and mechanical properties. 7. Alloys Compd. 541:177-85

  \item Murr LE, Quinones SA, Gaytan SM, Lopez MI, Rodela A, et al. 2009. Microstructure and mechanical behavior of Ti-6Al-4V produced by rapid-layer manufacturing, for biomedical applications. 7. Mech. Behav. Biomed. Mater. 2(1):20-32

  \item Vilaro T, Colin C, Bartout JD. 2011. As-fabricated and heat-treated microstructures of the Ti-6Al-4V alloy processed by selective laser melting. Metall. Mater. Trans. A 42(10):3190-99

  \item Vandenbroucke B, Kruth JP. 2007. Selective laser melting of biocompatible metals for rapid manufacturing of medical parts. Rapid Prototyp. 7. 13(4):196-203

  \item Mertens A, Reginster S, Paydas H, Contrepois Q, Dormal T, et al. 2014. Mechanical properties of alloy Ti-6Al-4V and of stainless steel 316L processed by selective laser melting: influence of out-of-equilibrium microstructures. Powder Metall. 57(3):184-89

  \item Hollander DA, von Walter M, Wirtz T, Sellei R, Schmidt-Rohlfing B, et al. 2006. Structural, mechanical and in vitro characterization of individually structured Ti-6Al-4V produced by direct laser forming. Biomaterials 27(7):955-63

  \item Qiu C, Adkins NJE, Attallah MM. 2013. Microstructure and tensile properties of selectively laser-melted and of HIPed laser-melted Ti-6Al-4V. Mater. Sci. Eng. A 578:230-39

  \item Yu J, Rombouts M, Maes G, Motmans F. 2012. Material properties of Ti6Al4V parts produced by laser metal deposition. Phys. Procedia 39:416-24

  \item Zhang S, Lin X, Chen J, Huang W. 2009. Heat-treated microstructure and mechanical properties of laser solid forming Ti-6Al-4V alloy. Rare Met. 28(6):537-44

  \item Alcisto J, Enriquez A, Garcia H, Hinkson S, Steelman T, et al. 2011. Tensile properties and microstructures of laser-formed Ti-6Al-4V. F. Mater. Eng. Perform. 20(2):203-12

  \item Dinda GP, Song L, Mazumder J. 2008. Fabrication of Ti-6Al-4V scaffolds by direct metal deposition. Metall. Mater. Trans. A 39(12):2914-22

  \item Zhai Y, Galarraga H, Lados DA. 2015. Microstructure evolution, tensile properties, fatigue damage mechanisms in Ti-6Al-4V alloys fabricated by two additive manufacturing techniques. Procedia Eng. 114:658-66

  \item Arcella FG, Froes FH. 2000. Producing titanium aerospace components from powder using laser forming. 7OM 52(5):28-30

  \item Lewis GK, Schlienger E. 2000. Practical considerations and capabilities for laser assisted direct metal deposition. Mater. Des. 21(4):417-23

  \item Griffith ML, Ensz MT, Puskar JD, Robino CV, Brooks JA, et al. 2000. Understanding the microstructure and properties of components fabricated by Laser Engineered Net Shaping (LENS). MRS Proc. 625:9; doi: 10.1557/PROC-625-9

  \item Löber L, Schimansky FP, Kühn U, Pyczak F, Eckert J. 2014. Selective laser melting of a beta-solidifying TNM-B1 titanium aluminide alloy. 7. Mater. Process. Technol. 214(9):1852-60

  \item Löber L, Biamino S, Ackelid U, Sabbadini S, Epicoco P, et al. 2011. Comparison of selective laser and electron beam melted titanium aluminides. In Solid Freeform Fabrication Proceedings, pp. 547-56. Austin: Univ. Tex.

  \item Spierings AB, Starr TL, Ag I. 2013. Fatigue performance of additive manufactured metallic parts. Rapid Prototyp. 7. 19(2):88-94

  \item Sercombe TB, Li X. 2016. Selective laser melting of aluminium and aluminium metal matrix composites: review. Mater. Technol. In press

  \item Wang XJ, Zhang LC, Fang MH, Sercombe TB. 2014. The effect of atmosphere on the structure and properties of a selective laser melted Al-12Si alloy. Mater. Sci. Eng. A 597:370-75

  \item Li XP, Wang XJ, Saunders M, Suvorova A, Zhang LC, et al. 2015. A selective laser melting and solution heat treatment refined Al-12Si alloy with a controllable ultrafine eutectic microstructure and $25 \%$ tensile ductility. Acta Mater. 95:74-82

  \item Read N, Wang W, Essa K, Attallah MM. 2015. Selective laser melting of AlSi10Mg alloy: process optimisation and mechanical properties development. Mater. Des. 65:417-24

  \item Manfredi D, Calignano F, Krishnan M, Canali R, Ambrosio EP, Atzeni E. 2013. From powders to dense metal parts: characterization of a commercial AlSiMg alloy processed through direct metal laser sintering. Materials 6(3):856-69

  \item Kempen K, Thijs L, Van Humbeeck J, Kruth J-P. 2012. Mechanical properties of AlSi10Mg produced by selective laser melting. Phys. Procedia 39:439-46

  \item Song C, Yang Y, Wang Y, Wang D, Yu J. 2014. Research on rapid manufacturing of CoCrMo alloy femoral component based on selective laser melting. Int. F. Adv. Manuf. Technol. 75(1-4):445-53

  \item Kircher R, Christensen A, Wurth K. 2009. Electron Beam Melted (EBM) Co-Cr-Mo alloy for orthopaedic implant applications. In Solid Freeform Fabrication Proceedings, pp. 428-36. Austin: Univ. Tex.

  \item Tarasova TV, Nazarov AP, Prokof ev MV. 2015. Effect of the regimes of selective laser melting on the structure and physicomechanical properties of cobalt-base superalloys. Phys. Met. Metallogr. 116(6):601-5

  \item Terrazas CA, Mireles J, Gaytan SM, Morton PA, Hinojos A, et al. 2016. Fabrication and characterization of high-purity niobium using electron beam melting additive manufacturing technology. Int. F. Adv. Manuf. Technol. In press

  \item Wei K, Gao M, Wang Z, Zeng X. 2014. Effect of energy input on formability, microstructure and mechanical properties of selective laser melted AZ91D magnesium alloy. Mater. Sci. Eng. A 611:212-22

  \item Ma Y, Cuiuri D, Hoye N, Li H, Pan Z. 2015. The effect of location on the microstructure and mechanical properties of titanium aluminides produced by additive layer manufacturing using in-situ alloying and gas tungsten arc welding. Mater. Sci. Eng. A 631:230-40

  \item Baufeld B. 2012. Mechanical properties of INCONEL 718 parts manufactured by shaped metal deposition (SMD). 7. Mater. Eng. Perform. 21(7):1416-21

  \item Blackwell PL. 2005. The mechanical and microstructural characteristics of laser-deposited IN718. 7. Mater. Process. Technol. 170(1-2):240-46

  \item Zhao X, Chen J, Lin X, Huang W. 2008. Study on microstructure and mechanical properties of laser rapid forming Inconel 718. Mater. Sci. Eng. A 478(1-2):119-24

  \item Bird RK, Hibberd J. 2009. Tensile properties and microstructure of Inconel 718 fabricated with electron beam freeform fabrication $\left(E B F^{3}\right)$. Tech. Rep., NASA

  \item Qi H, Azer M, Ritter A. 2009. Studies of standard heat treatment effects on microstructure and mechanical properties of laser net shape manufactured INCONEL 718. Metall. Mater. Trans. A 40(10):2410-22

  \item Cao X, Rivaux B, Jahazi M, Cuddy J, Birur A. 2009. Effect of pre- and post-weld heat treatment on metallurgical and tensile properties of Inconel 718 alloy butt joints welded using $4 \mathrm{~kW}$ Nd:YAG laser. 7. Mater. Sci. 44(17):4557-71

  \item Edwards P, Ramulu M. 2015. Effect of build direction on the fracture toughness and fatigue crack growth in selective laser melted Ti-6Al-4V. Fatigue Fract. Eng. Mater. Struct. 38(10):1228-36

  \item Van Hooreweder B, Moens D, Boonen R, Kruth J-P, Sas P. 2012. Analysis of fracture toughness and crack propagation of Ti6Al4V produced by selective laser melting. Adv. Eng. Mater. 14(1-2):92-97

\end{enumerate}

\begin{center}
\includegraphics[max width=\textwidth]{2024_03_11_add5292786349c475313g-094}
\end{center}

\begin{enumerate}
  \setcounter{enumi}{193}
  \item Becker TH, Beck M, Scheffer C. 2015. Microstructure and mechanical properties of direct metal laser sintered Ti-6Al-4V. S. Afr. 7. Ind. 7. 26:1-10

  \item Svensson M. 2009. Ti6Al4V manufactured with electron beam melting (EBM): mechanical and chemical properties. In Proceedings from the Materials \& Processes for Medical Devices Conference, pp. 189-94. Novelty, $\mathrm{OH}$ : ASM Int.

\end{enumerate}

Lewandowski $\cdot$ Seifi

\begin{enumerate}
  \setcounter{enumi}{195}
  \item Boyer R, Welsch G, Collings EW. 1994. Materials Properties Handbook: Titanium Alloys. Novelty, OH: ASM Int.

  \item Seifi M, Salem A, Satko D, Lewandowski JJ. 2016. Role of defect and microstructure on fracture resistance and fatigue crack growth behavior of AM Ti-6Al-4V alloy. Int. F. Fatigue 91(10):in press

  \item Seifi M, Ghamarian I, Samimi P, Collins PC, Lewandowski JJ. 2016. Microstructure and mechanical properties of Ti-48Al-2 Cr-2Nb manufactured via electron beam melting. In Proceedings of World Conference on Titanium, 13th, pp. 1317-22. Warrendale, PA/Hoboken, NJ: TMS/Wiley

  \item Seifi M, Salem A, Satko D, Ackelid U, Lewandowski JJ. 2016. Microstructural inhomogeneity and post processing effects on mechanical properties of Ti-48Al-2 $\mathrm{Cr}-2 \mathrm{Nb}$ manufactured by $\mathrm{EBM}$ additive manufacturing. Intermetallics. Under review

  \item Dahar MS, Seifi SM, Bewlay BP, Lewandowski JJ. 2015. Effects of test orientation on fracture and fatigue crack growth behavior of third generation as-cast Ti-48Al-2Nb-2Cr. Intermetallics 57:73-82

  \item Li P, Warner DH, Fatemi A, Phan N. 2015. Critical assessment of the fatigue performance of additively manufactured Ti-6Al-4V and perspective for future research. Int. 7. Fatigue 85:130-43

  \item Fodran E, Walker K. 2015. Surface finish enhancement for the electron beam direct digital manufacturing of Ti-6Al-4V alloy structural components. Tech. Rep., Armament Research, Development and Engineering Center, Weapons Software Engineering Center, Benét Lab.

  \item Seifi M, Lewandowski JJ. 2016. Microstructure and mechanical properties of additive manufactured alloys. Prog. Mater. Sci. In preparation

  \item Filippini M, Beretta S, Içöz C, Patriarca L. 2015. Effect of the microstructure on the fatigue strength of a TiAl intermetallic alloy produced by additive manufacturing. Mater. Res. Soc. Symp. Proc. 1:3-8

  \item McLouth T, Chang Y, Wooten J, Yang J. 2015. The effects of electron beam melting on the microstructure and mechanical properties of Ti-6Al-4V and $\gamma$-TiAl. Microsc. Microanal. 21(588):1177-78

  \item Seikh A, Mohammad A, Sherif E-S, Al-Ahmari A. 2015. Corrosion behavior in $3.5 \% \mathrm{NaCl}$ solutions of $\gamma$-TiAl processed by electron beam melting process. Metals 5(4):2289-302

  \item Tang HP, Yang GY, Jia WP, He WW, Lu SL, Qian M. 2015. Additive manufacturing of a high niobiumcontaining titanium aluminide alloy by selective electron beam melting. Mater. Sci. Eng. A 636:103-7

  \item Biamino S, Klöden B, Weißgärber T, Kieback B, Ackelid U. 2014. Titanium aluminides for automotive applications processed by electron beam melting. In Proceedings of Metal Powder Industries Federation (MPIF), pp. 96-103. Princeton, NJ: MPIF

  \item Filippini M, Beretta S, Patriarca L, Sabbadini S. 2014. Effect of the microstructure on the deformation and fatigue damage in a gamma TiAl produced by additive manufacturing. In TMS Proceedings, pp. 18993. Warrendale, PA/Hoboken, NJ: TMS/Wiley

  \item Ge W, Guo C, Lin F. 2014. Effect of process parameters on microstructure of TiAl alloy produced by electron beam selective melting. Procedia Eng. 81:1192-97

  \item Ge W, Lin F, Guo C. 2014. The effect of scan pattern on microstructure evolution and mechanical properties in electron beam melting Ti47A12Cr2Nb. In Solid Freeform Fabrication Proceedings, pp. 50113. Austin: Univ. Tex.

  \item Schwerdtfeger J, Körner C. 2014. Selective electron beam melting of Ti-48Al-2Nb-2Cr: microstructure and aluminium loss. Intermetallics 49:29-35

  \item Filippini M, Beretta S, Patriarca L, Pasquero G, Sabbadini S. 2012. Fatigue sensitivity to small defects of a gamma-titanium-aluminide alloy. 7. ASTM Int. 9(5):104293

  \item Terner M, Biamino S, Ugues D, Sabbadini S, Fino P, et al. 2013. Phase transitions assessment on $\gamma$-TiAl by thermo mechanical analysis. Intermetallics $37: 7-10$

  \item Hernandez J, Murr LE, Gaytan SM, Martinez E, Medina F, Wicker RB. 2012. Microstructures for two-phase gamma titanium aluminide fabricated by electron beam melting. Metallogr. Microstruct. Anal. $1(1): 14-27$

  \item Biamino S, Penna A, Ackelid U, Sabbadini S, Tassa O, et al. 2011. Electron beam melting of Ti-48Al$2 \mathrm{Cr}-2 \mathrm{Nb}$ alloy: microstructure and mechanical properties investigation. Intermetallics 19(6):776-81

  \item Filippini M, Beretta S, Patriarca L, Pasquero G, Sabbadini S. 2011. Defect tolerance of a gamma titanium aluminide alloy. Procedia Eng. 10:3677-82

\end{enumerate}

\begin{center}
\includegraphics[max width=\textwidth]{2024_03_11_add5292786349c475313g-095}
\end{center}

\begin{enumerate}
  \setcounter{enumi}{217}
  \item Franzen SF, Karlsson J, Dehoff R, Ackelid U, Rios O, et al. 2011. Microstructural properties of gamma titanium aluminide manufactured by electron beam melting. In TMS Proceedings, pp. 455-62. Warrendale, PA/Hoboken, NJ: TMS/Wiley

  \item Murr LE, Gaytan SM, Ceylan A, Martinez E, Martinez JL, et al. 2010. Characterization of titanium aluminide alloy components fabricated by additive manufacturing using electron beam melting. Acta Mater. 58(5):1887-94

  \item Patriarca L. 2010. Fatigue crack growth of a gamma titanium aluminide alloy. In Youth Symposium on Experimental Solid Mechanics, 9th, pp. 36-39

  \item Sabbadini S, Tassa O, Gennaro P, Ackelid U. 2010. Additive manufacturing of gamma titanium aluminide parts by electron beam melting. In TMS Proceedings, pp. 267-74. Warrendale, PA/Hoboken, NJ: TMS/Wiley

  \item Cormier D, Harrysson O, Mahale T, West H. 2007. Freeform fabrication of titanium aluminide via electron beam melting using prealloyed and blended powders. Res. Lett. Mater. Sci. 2007:1-4

  \item Li W, Liu J, Wen W, Wei Q, Yan C, Shi Y. 2016. Crystal orientation, crystallographic texture and phase evolution in the Ti-45Al-2Cr-5Nb alloy processed by selective laser melting. Mater. Charact. 113:125-33

  \item Gussone J, Hagedorn Y-C, Gherekhloo H, Kasperovich G, Merzouk T, Hausmann J. 2015. Microstructure of $\gamma$-titanium aluminide processed by selected laser melting at elevated temperatures. Intermetallics 66:133-40

  \item Yadollahi A, Shamsaei N, Thompson SM, Seely DW. 2015. Effects of process time interval and heat treatment on the mechanical and microstructural properties of direct laser deposited 316L stainless steel. Mater. Sci. Eng. A 644:171-83

  \item Sterling A, Torries B, Shamsaei N, Thompson SM, Seely DW. 2016. Fatigue behavior and failure mechanisms of direct laser deposited Ti-6Al-4V. Mater. Sci. Eng. A 655:100-12

\end{enumerate}

\begin{center}
\includegraphics[max width=\textwidth]{2024_03_11_add5292786349c475313g-096}
\end{center}

\section*{Spectra H Titanium Ti6Al4V Grade 5}
\section*{Base Parameter for Arcam EBM Spectra H}
Data in this material datasheet represents material built with $50 \mu m$ layer thickness and in an Vacuum atmosphere on a Arcam EBM Spectra H - V1.0. Values listed are typical.

\begin{center}
\includegraphics[max width=\textwidth]{2024_03_11_add5292786349c475313g-097}
\end{center}

\section*{Titanium Ti6Al4V}
The high strength, low weight ratio and outstanding corrosion resistance inherent to titanium and its alloys has led to a wide and diversified range of successful applications which demand high levels of reliable performance in surgery and medicine as well as in aerospace, automotive, chemical plant, power generation, oil and gas extraction, sports, and other major industries. The biocompatibility of Ti6AI4V is excellent, especially when direct contact with tissue or bone is required. In the majority of other engineering applications, titanium replaces heavier, less serviceable or less cost-effective materials.

\section*{Spectra H - Ti6AI4V - Grade 5}
In the Arcam Spectra H-machine the Ti6Al4V Grade 5 process runs at a powder bed temperature of around $700^{\circ} \mathrm{C}$. The elevated powder bed temperature eliminates the need for post-build stress-relieving and allows the building of parts in multiple layers in the Zdirection which further increases productivity. The chosen layer thickness enables both high build speed and high resolution.

\begin{center}
\includegraphics[max width=\textwidth]{2024_03_11_add5292786349c475313g-097(1)}
\end{center}

\section*{Spectra H Ti6Al4V Grade 5}
This material is developed as a standard material according to Production level (P-material). It has been fully verified according to Arcam EBM Process Verification, with data available for full build envelope. The process parameters can be further optimized to suit the specific needs for individual applications. The parameters set fulfils ASTM F2924 standard both regards to chemical composition as well as mechanical properties.

\section*{POWDER INFORMATION}
AP\&C Ti6AI4V - Grade 5 powder 45-106 $\mu$ m particle size distribution. For more information on Ti6Al4V powder, visit \href{https://www.advancedpowders.com/powders/titanium/ti-6al-4v-5}{https://www.advancedpowders.com/powders/titanium/ti-6al-4v-5},

\section*{MACHINE CONFIGURATION}
\begin{itemize}
  \item Spectra H-V1.0
  \item EBM Control 5.5
\end{itemize}

\section*{AVAILABLE PARAMETERS}
\begin{center}
\includegraphics[max width=\textwidth]{2024_03_11_add5292786349c475313g-098}
\end{center}

\section*{THERMAL TREATMENTS}
\begin{enumerate}
  \item As-Built

  \item HIP $-920^{\circ} \mathrm{C} \pm 10^{\circ} \mathrm{C}, 1000$ bar $\pm 50$ bar, $120 \mathrm{~min} \pm 30 \mathrm{~min}$, cool under inert atmosphere to below $425^{\circ} \mathrm{C}$

\end{enumerate}

\section*{THERMAL TREATMENT COMPARISON}
\begin{center}
\includegraphics[max width=\textwidth]{2024_03_11_add5292786349c475313g-098(1)}
\end{center}

Spider Plot is generated by normalizing typical material data (containing only vertical data) against a range defined for each material family. For Titanium Alloys, the ranges are as follows:

UTS: 0-1300 MPa

$0.2 \%$ YS: 0-1200 MPa

Elongation: 0-20\%

Reduction of Area: 0-60\%

Surface Quality (all): 25-40 $\mu \mathrm{m}$

Surface Roughness

$(\mu \mathrm{m})$

Upskin

Downksin

\begin{center}
\begin{tabular}{l|l|}
 & 32 \\
 & 28 \\
 &  \\
\end{tabular}
\end{center}

\begin{center}
\begin{tabular}{lcccccc}
 &  & \multicolumn{2}{c}{}\begin{tabular}{c}
Porosity \\
(\% Density) \\
\end{tabular} & \multicolumn{2}{c}{Hardness} & \multicolumn{2}{c}{Poisson's Ratio} \\
Thermal State & $\mathrm{H}$ & $\mathrm{V}$ & $\mathrm{H}$ & $\mathrm{V}(\mathrm{XZ}-\mathrm{YZ})$ & $\mathrm{H}$ &  &  \\
\hline
As-Built & 99.8 & 99.8 & 350 & $327-326$ & $\mathrm{~V}$ &  &  \\
2 HIP & -- & -- & -- & -- & -- &  &  \\
\hline
\end{tabular}
\end{center}

Temperature: RT

1 As-Built

2 HIP

\begin{center}
\begin{tabular}{|c|c|c|c|c|c|c|c|c|c|}
\hline
\multicolumn{2}{|c|}{}\begin{tabular}{l}
Modulus of Elasticity \\
(GPa) \\
\end{tabular} & \multicolumn{2}{|c|}{}\begin{tabular}{c}
$0.2 \%$ YS \\
$(\mathrm{MPa})$ \\
\end{tabular} & \multicolumn{2}{|c|}{}\begin{tabular}{l}
UTS \\
(MPa) \\
\end{tabular} & \multicolumn{2}{|c|}{}\begin{tabular}{c}
Elongation \\
$(\%)$ \\
\end{tabular} & \multicolumn{2}{|c|}{}\begin{tabular}{l}
Reduction of Area \\
(\%) \\
\end{tabular} \\
\hline
$\mathrm{H}$ & V & $\mathrm{H}$ & V & $\mathrm{H}$ & V & $\mathrm{H}$ & V & $\mathrm{H}$ & V \\
\hline
-- & -- & 915 & 910 & 1005 & 1005 & 14.7 & 16.4 & 35.7 & 43.0 \\
\hline
-- & -- & -- & 850 & -- & 950 & -- & 18.9 & -- & 50.0 \\
\hline
\end{tabular}
\end{center}

\section*{SEM IMAGES}
1 As-Built

1 HIP

H: HORIZONTAL (XY) orientation $\mathrm{V}$ : VERTICAL (Z) orientation

\begin{itemize}
  \item All of the figures contained herein are approximate only. The figures provided are dependent on a number of factors, including but not limited to, process and machine parameters, and the approval is brand specific and/or application specific. The information provided on this material data sheet is illustrative only and cannot be relied on as binding.
\end{itemize}

\section*{Material data sheet - FlexLine}
\section*{EOS StainlessSteel 316L}
EOS StainlessSteel $316 \mathrm{~L}$ is a stainless steel powder intended for processing on EOS DMLS ${ }^{T M}$ machines.

\begin{itemize}
  \item EOS DMLSTMEOS M100 system
  \item HSS blade (300006274)
  \item Argon atmosphere
  \item IPCM extra sieving module with $63 \mu \mathrm{m}$ mesh
  \item Hand sieve with $63 \mu \mathrm{m}$ mesh (300013590) recommended
  \item EOSYSTEM
  \item EOSPRINT v 1.6 or higher
  \item Software: HCS 1.8 or higher
  \item EOS Parameter set: 316L\_020\_FlexM100\_200
\end{itemize}

\section*{Description}
EOS StainlessSteel $316 \mathrm{~L}$ is a corrosion resistant iron based alloy which has been optimized for processing on EOS DMLS systems. EOS StainlessSteel 316L have chemical composition corresponding to ASTM F138 "Standard Specification for Wrought 18Cr-14Ni-2.5Mo Stainless Steel Bar and Wire for Surgical Implants (UNS S31673)". This kind of stainless steel is characterized having a good corrosion resistance and evidence that there are no leachable substances in cytotoxic concentrations.

This material is ideal in

\begin{itemize}
  \item Lifestyle/Consumer, e.g. watches, other jewellery, spectacle frames, decorations
  \item Automotive/Industrial, e.g. non-corroding common material, food and chemical plants
  \item Aerospace/Turbine industry
  \item Entry-level material for Laser Sintering Technology, e.g. mounting parts, heat exchangers, functional elements in electronic housing and accessories
\end{itemize}

Parts built from EOS StainlessSteel $316 \mathrm{~L}$ can be machined, shot-peened and polished in as-built or stress relieved (AMS2759) states if required. Solution annealing is not necessary because the mechanical properties of as-built state are showing desired values (ASTM A403). Parts are not ideal in temperature range $427^{\circ} \mathrm{C}-816^{\circ} \mathrm{C}$ where precipitation of chromium carbides occurs. Due to layer-wise building method, the parts have a certain anisotropy which could be seen from mechanical properties.

\section*{Material data sheet - FlexLine}
\section*{Technical Data}
\section*{Powder properties}
The chemical composition of the powder (wt-\%):

\begin{center}
\begin{tabular}{|c|c|c|c|}
\hline
Material composition & Element & Min & Max \\
\hline
 & $\mathrm{Fe}$ & \multicolumn{2}{|c|}{Balance} \\
\hline
 & $\mathrm{Cr}$ & 17.00 & 19.00 \\
\hline
 & $\mathrm{Ni}$ & 13.00 & 15.00 \\
\hline
 & Mo & 2.25 & 3.00 \\
\hline
 & C &  & 0.030 \\
\hline
 & $\mathrm{Mn}$ &  & 2.00 \\
\hline
 & $\mathrm{Cu}$ &  & 0.50 \\
\hline
 & $P$ &  & 0.025 \\
\hline
 & $S$ &  & 0.010 \\
\hline
 & Si &  & 0.75 \\
\hline
 & $\mathrm{N}$ &  & 0.10 \\
\hline
\end{tabular}
\end{center}

Max. particle size

$\geq 63 \mu \mathrm{m}[1] \quad$ Max. 1,0 wt\%

[1] analysis according to ASTM B214.

\section*{General process data}
\begin{center}
\begin{tabular}{lc}
\hline
Layer thickness & $20 \mu \mathrm{m}$ \\
\hline
Volume rate [2] & $1,16 \mathrm{~mm}^{3} / \mathrm{s}\left(4,17 \mathrm{~cm}^{3} / \mathrm{h}\right)$ \\
$0,25 \mathrm{in}^{3} / \mathrm{h}$ &  \\
\end{tabular}
\end{center}

[2] The volume rate is a measure of build speed during laser exposure of the skin area. The total build speed depends on this volume rate and many other factors such as exposure parameters of contours, supports, up and downskin, recoating time, Home-In or LPM settings.

\section*{Material data sheet - FlexLine}
\section*{Physical and chemical properties of parts}
\begin{center}
\begin{tabular}{lr}
\hline
Part density [3] & $\min .7,98 \mathrm{~g} / \mathrm{cm}^{3}$ \\
\hline
Surface roughness after shot peening [4] & Ra $<12 \mu \mathrm{m} ; \mathrm{Rz}<62 \mu \mathrm{m}$ \\
\hline
\end{tabular}
\end{center}

[3] Weighing in air and water according to ISO 3369.

[4] Measurement according to ISO 4287. The numbers were measured at the horizontal (up-facing) and all vertical surfaces of test cubes. Due to the layerwise building the roughness strongly depends on the orientation of the surface, for example sloping and curved surfaces exhibit a stair-step effect.

\section*{Tensile data at room temperature $[5,6]$}
As built

\begin{center}
\begin{tabular}{lcc}
\hline
 & Horizontal & Vertical \\
\hline
Ultimate tensile strength, $\mathrm{Rm}$ & $650 \mathrm{MPa}$ & $590 \mathrm{MPa}$ \\
\hline
Yield strength, Rpo.2 & $535 \mathrm{MPa}$ & $490 \mathrm{MPa}$ \\
\hline
Elongation at break, A & $35 \%$ & $45 \%$ \\
\hline
\end{tabular}
\end{center}

[5] The numbers are average values of vertical and horizontal orientation.

[6] Tensile testing according to ISO 6892 \& ASTM E8M, proportional test pieces, diameter of the neck area $4 \mathrm{~mm}$, original gauge length 4D (16 $\mathrm{mm})$. Tensile test parameters: Stress rate $10 \mathrm{MPa} / \mathrm{s}$ in elastic range, strain speed in plastic region 0,375 1/min. Results are derived from the validation data made with EOS M100 system and two powder LOTs

\section*{Material data sheet - FlexLine}
\section*{Abbreviations}
min. minimum

max. maximum

wt. weight

\section*{Legal notes}
The quoted values refer to the use of this material with above specified EOS DMLS system, EOSYSTEM software version, parameter set and operation in compliance with parameter sheet and operating instructions. All measured values are average numbers. Part properties are measured with specified measurement methods using defined test geometries and procedures. Further details of the test procedures used by EOS are available on request. Any deviation from these standard settings may affect the measured properties.

The data correspond to EOS knowledge and experience at the time of publication and they are subject to change without notice as part of EOS' continuous development and improvement processes.

EOS does not warrant any properties or fitness for a specific purpose, unless explicitly agreed upon. This also applies regarding any rights of protection as well as laws and regulations.

EOS ${ }^{\circledR}$, EOSINT ${ }^{\circledR}, \mathrm{DMLS}{ }^{\circledR}$, DirectTool ${ }^{\circledR}$ and DirectPart ${ }^{\circledR}$ are registered trademarks of EOS $\mathrm{GmbH}$.

CC2017 EOS GmbH - Electro Optical Systems. All rights reserved.

\section*{Material data sheet}
\section*{EOS StainlessSteel PH1}
EOS StainlessSteel PH1 is a metal alloy powder which has been optimized for processing on EOS M systems.

This document provides information and data for parts built using EOS StainlessSteel PH1 powder (EOS art.-no. 9011-0019) on the following specifications:

\begin{itemize}
  \item DMLS system: M290
  \item Ceramic blade (2200-3013), carbon fibre (2200-4366) compatible
  \item IPCM sieving module with $63 \mu \mathrm{m}$ mesh size (9044-0032) recommended
  \item Nitrogen atmosphere
  \item Internal 804m mesh sieve (1212-0312) possible
  \item Software: EOSYSTEM 2.4 or newer
  \item Parameter set:PH1\_020\_Surface\_M291\_200
\end{itemize}

\section*{Description}
EOS StainlessSteel PH1 is a pre-alloyed stainless steel in fine powder form. The chemistry of EOS StainlessSteel PH1 conforms to the compositions of DIN 1.4540 and UNS S15500.

This kind of steel is characterized by having good corrosion resistance and excellent mechanical properties, especially in the precipitation hardened state. This type of steel is widely used in variety of engineering applications requiring high hardness, strength and corrosion resistance.

This material is ideal for many part-building applications (DirectPart) such as functional metal prototypes, small series products, individualised products or spare parts. Standard processing parameters use full melting of the entire geometry with $20 \mu \mathrm{m}$ layer thickness. Using standard parameters the mechanical properties are fairly uniform in all directions. Parts made from EOS StainlessSteel PH1 can be machined, spark-eroded, welded, micro shot-peened, polished and coated if required.

Typical applications:

\begin{itemize}
  \item engineering applications including functional prototypes, small series products, individualised products or spare parts.
  \item parts requiring high corrosion resistance, sterilisability, etc.
  \item parts requiring particularly high hardness and strength.
\end{itemize}

\section*{Material data sheet}
\section*{Technical Data}
\section*{Powder properties}
\section*{Material composition}
\begin{center}
\begin{tabular}{ccc}
Element & Min [wt.-\%] & Max [wt.-\%] \\
\hline
$\mathrm{Cr}$ & 14.0 & 15.5 \\
\hline
$\mathrm{Ni}$ & 3.5 & 5.5 \\
\hline
$\mathrm{Cu}$ & 2.5 & 4.5 \\
\hline
$\mathrm{Mn}$ &  & 1.00 \\
\hline
$\mathrm{Si}$ &  & 1.00 \\
\hline
$\mathrm{C}$ &  & 0.07 \\
\hline
$\mathrm{Mo}$ &  & 0.5 \\
\hline
$\mathrm{Nb}$ & 0.15 & 0.45 \\
\hline
\end{tabular}
\end{center}

Max. particle size

Particles $\geq 63 \mu \mathrm{m}[1]$

0.5 wt.-\%

[1] Sieve analysis according to DIN ISO 4497 or ASTM B214.

\section*{Material data sheet}
\section*{General process data}
\begin{center}
\begin{tabular}{lc}
\hline
Layer thickness & $20 \mu \mathrm{m}$ \\
 & $0.8 \mathrm{mil}$. \\
\hline
Typical achievable part accuracy [2] &  \\
- small parts & $\pm 20-50 \mu \mathrm{m}$ \\
 & $0.8-2.0 \mathrm{mil}$ \\
- large parts & $\pm 0.2 \%$ \\
\hline
Volume rate [3] & $\sim 2.0 \mathrm{~mm}^{3} / \mathrm{s}\left(\sim 7.0 \mathrm{~cm}^{3} / \mathrm{h}\right)$ \\
 & $\sim 0.40 \mathrm{in}^{3} / \mathrm{h}$ \\
\hline
Min. wall thickness [4] & $\sim 0.4 \mathrm{~mm}$ \\
\hline
\end{tabular}
\end{center}

[2] Based on users' experience of dimensional accuracy for typical geometries, e.g. $\pm 20 \mu \mathrm{m}$ when parameters can be optimized for a certain class of parts or $\pm 50 \mu \mathrm{m}$ when building a new kind of geometry for the first time.

[3] The volume rate is a measure of build speed during laser exposure of the skin area. The total build speed depends on this volume rate and many other factors such as exposure parameters of contours, supports, up and downskin, recoating time, Home-In or LPM settings.

[4] Mechanical stability is dependent on geometry (wall height etc.) and application

\section*{Material data sheet}
\section*{Physical and chemical properties of parts}
\begin{center}
\begin{tabular}{|c|c|}
\hline
Part density [5] & \begin{tabular}{r}
$\sim 7.7 \mathrm{~g} / \mathrm{cm}^{3}$ \\
$\sim 0.27 \mathrm{lb} / \mathrm{in}^{3}$ \\
\end{tabular} \\
\hline
Relative density with standard parameters & approx. $100 \%$ \\
\hline
\multicolumn{2}{|l|}{Surface roughness after shot peening [6]} \\
\hline
- after shot-peening & \begin{tabular}{c}
Ra $\sim 5.0 \mu \mathrm{m}, \mathrm{Rz} \sim 25.0 \mu \mathrm{m}$ \\
$\mathrm{Ra} \sim 0.2 \mathrm{mil}, \mathrm{Rz} \sim 1.0 \mathrm{mil}$ \\
\end{tabular} \\
\hline
- after polishing & \begin{tabular}{l}
$\quad \mathrm{Rz}$ up to $<0.5 \mu \mathrm{m}$ \\
(can be very finely polished) \\
\end{tabular} \\
\hline
\end{tabular}
\end{center}

[5] Weighing in air and water according to ISO 3369.

[6] Measurement according to ISO 4287. The numbers were measured at the horizontal (up-facing) and all vertical surfaces of test cubes. Due to the layerwise building the roughness strongly depends on the orientation of the surface, for example sloping and curved surfaces exhibit a stair-step effect.

\section*{Material data sheet}
\section*{Tensile data at room temperature [7]}
\begin{center}
\begin{tabular}{|c|c|c|}
\hline
 & As built & \begin{tabular}{c}
Heat treated $[9]$ \\
$(\bmod$ H900 heat treatment $)$ \\
\end{tabular} \\
\hline
\multicolumn{3}{|l|}{Ultimate tensile strength} \\
\hline
- in horizontal direction (XY) & $1200 \pm 50 \mathrm{MPa}$ & \begin{tabular}{c}
min. $1350 \mathrm{MPa}$ \\
(typical $1450 \pm 100 \mathrm{MPa}$ ) \\
\end{tabular} \\
\hline
- in vertical direction (Z) & $1200 \pm 50 \mathrm{MPa}$ & \begin{tabular}{c}
$\min .1340 \mathrm{MPa}$ \\
(typical $1440 \pm 100 \mathrm{MPa}$ ) \\
\end{tabular} \\
\hline
\multicolumn{3}{|l|}{Yield strength (Rp $0.2 \%$ )} \\
\hline
- in horizontal direction (XY) & $1025 \pm 85 \mathrm{MPa}$ & \begin{tabular}{c}
$\min .1250 \mathrm{MPa}$ \\
(typical $1350 \pm 100 \mathrm{MPa}$ ) \\
\end{tabular} \\
\hline
- in vertical direction (Z) & $930 \pm 75 \mathrm{MPa}$ & \begin{tabular}{c}
min. $1200 \mathrm{MPa}$ \\
(typical $1300 \pm 100 \mathrm{MPa}$ ) \\
\end{tabular} \\
\hline
\multicolumn{3}{|l|}{Elongation at break} \\
\hline
- in horizontal direction (XY) & $17 \% \pm 4 \%$ & \begin{tabular}{c}
$\min 10 \%$ \\
(typical $15 \% \pm 3 \%)$ \\
\end{tabular} \\
\hline
- in vertical direction (Z) & $14 \% \pm 4 \%$ & \begin{tabular}{c}
$\min 10 \%$ \\
(typical $13 \% \pm 3 \%$ ) \\
\end{tabular} \\
\hline
\multicolumn{3}{|l|}{Hardness [8]} \\
\hline
- as built & - & \begin{tabular}{c}
$\min 40 \mathrm{HRC}$ \\
(typical $43 \mathrm{HRC}$ ) \\
\end{tabular} \\
\hline
\end{tabular}
\end{center}

[7] Mechanical testing according to IS0 6892:1998(E) Annex C, proportional test pieces, Diameter of the neck area $5 \mathrm{~mm}$, original gauge length $25 \mathrm{~mm}$, test pieces built in $20 \mu \mathrm{m}$ layer-thickness.

[8] Rockwell C (HRC) hardness measurement according to DIN EN ISO 6508-1. Note that depending on the measurement method used, the measured hardness value can be dependent on the surface roughness and can be lower than the real hardness. To avoid inaccurate results, hardness should be measured on a polished surface.

[9] Mechanical properties are expressed as minimum values to indicate that mechanical properties exceed the min requirements of material specification standards such as ASTM A564-04 (XM12), ASTM A693-06 (XM12). Hardening of EOS StainlessSteel PH1 done using modified H900 heat treatment (soaking time at precipitation hardening temperature $525^{\circ} \mathrm{C}$ elongated for 4 hours).

\section*{Material data sheet}
\section*{Abbreviations}
min. minimum

max. maximum

wt. weight

The quoted values refer to the use of this material with above specified type of EOS DMLS system, EOSYSTEM software version, parameter set and operation in compliance with parameter sheet and operating instructions. Part properties are measured with specified measurement methods using defined test geometries and procedures and. Further details of the test procedures used by EOS are available on request. Any deviation from these standard settings may affect the measured properties.

The data correspond to EOS knowledge and experience at the time of publication and they are subject to change without notice as part of EOS' continuous development and improvement processes.

EOS does not warrant any properties or fitness for a specific purpose, unless explicitly agreed upon. This also applies regarding any rights of protection as well as laws and regulations.

EOS $^{\circledR}$, EOSINT $^{\circledast}$, DMLS $^{\circledR}$, DirectTool $^{\circledast}$ and DirectPart ${ }^{\circledR}$ are registered trademarks of EOS GmbH.

(C) 2017 EOS GmbH - Electro Optical Systems. All rights reserved.

\section*{$\forall$ AddUp}
\section*{Ti6-4}
\section*{SPECIFICATIONS}
EU Ti-6Al-4V

USA UNSR56400

\section*{MATERIAL DESCRIPTION}
\begin{itemize}
  \item Titanium alloy with good mechanical properties up to $300^{\circ} \mathrm{C}$. Titanium alloy alpha + beta type.
\end{itemize}

\section*{COMPOSITION}
$\%$ massique

\begin{center}
\begin{tabular}{ccc}
$\mathrm{Ti}$ &  & Balance \\
$\mathrm{Al}$ & - & 6 \\
$\mathrm{~V}$ & 4 &  \\
$\mathrm{O} 2$ & 0,2 &  \\
$\mathrm{C}$ & $<0,08$ &  \\
\end{tabular}
\end{center}

\section*{APPLICATIONS}
\begin{center}
\includegraphics[max width=\textwidth]{2024_03_11_add5292786349c475313g-110}
\end{center}

\section*{MATERIAL SHEET}
\section*{Typical mechanical properties}
The data provided in this document represent typical but not guaranteed values.

\begin{center}
\begin{tabular}{|c|c|c|}
\hline
 & Stress-relieved & Heat treated* \\
\hline
\begin{tabular}{r}
Ultimate Tensile Strength \\
UTS, MPa \\
\end{tabular} & 1150 & 980 \\
\hline
\begin{tabular}{r}
Yield Strength \\
YS, MPa \\
\end{tabular} & 1070 & 860 \\
\hline
\begin{tabular}{r}
Elongation at break \\
E 5D,\% \\
\end{tabular} & 10 & 15 \\
\hline
\end{tabular}
\end{center}

\begin{itemize}
  \item Heat treatment: $920^{\circ} \mathrm{C} / 2 \mathrm{~h}$.
\end{itemize}

The mechanical characteristics along the $Z$ axis are lowered by about 100 MPa after manufacturing.

The anisotropy is significantly reduced, if not eliminated, after heat treatment.

\section*{Physical properties}
\begin{center}
\includegraphics[max width=\textwidth]{2024_03_11_add5292786349c475313g-110(2)}
\end{center}

\section*{Qualitative comparison according to processes}
\begin{center}
\includegraphics[max width=\textwidth]{2024_03_11_add5292786349c475313g-110(1)}
\end{center}

\section*{Technical data}
PARTICULES SIZE

Available in different granulometries.

SUPPLIERS

AddUp will provide support with your choice of powder supplier.

\section*{MATERIAL SHEET}
\section*{Applications, in detail}
\begin{center}
\includegraphics[max width=\textwidth]{2024_03_11_add5292786349c475313g-111}
\end{center}

\section*{INDUSTRY}
Tubing

Ti6-4 has a very good resistance against corrosion. It can be used in corrosive environments such as water, sea water, or natural streams.

Main applications would be in tubing, nozzles, and valve bodies.

\section*{AERONAUTIC}
\begin{center}
\includegraphics[max width=\textwidth]{2024_03_11_add5292786349c475313g-111(1)}
\end{center}

Sub-assembly

Its good mechanical properties and its specific strength (UTS/ density) make Ti6-4 commonly used for aeronautic structures like longeron, external doors safety parts or fasteners in ambient or warm atmospheres $\left(<300^{\circ} \mathrm{C}\right.$ ).

\section*{POWER GENERATION}
Land-based turbine

For land-based turbines, Ti6-4 is used in cold sections $\left(<300^{\circ} \mathrm{C}\right)$ to reduce the mass of rotating parts like disks or blades.

\section*{LUXURY}
Watch

The compromise between lightness and hardness makes Ti6-4 (Titanium grade 5) useful for watch cases and/or wrist strap.

Its low density (4.5) reduces the mass of the watch and while retaining its level of hardness to be scratchresistant.


\end{document}